\documentclass[a4paper,12pt,twoside]{book}

%for appendicies
\usepackage[titletoc]{appendix}

% to check unused references
%\usepackage{refcheck}
\usepackage{slashed}
\usepackage{setspace}
% additional packages
%\usepackage{xcolor}
\usepackage{graphicx}
\usepackage[ngerman,english]{babel}
%\usepackage{subfigure}
\usepackage{subfig}
\usepackage{epigraph}
\usepackage{multirow}
\usepackage{amssymb}
\usepackage{rotating}
\usepackage{wrapfig}
\usepackage{gensymb}
\usepackage{amsmath}
\usepackage[makeroom]{cancel}
\usepackage[bottom]{footmisc}
%to allign text vertically in table cells
\usepackage{pdflscape}
\usepackage{diagbox}
\usepackage{array}
\usepackage{makecell}
%xcolor MUST be loaded BEFORE the footnote, otherwise they conflict
\usepackage{footnote}
\usepackage{float}
\usepackage{amssymb}
\usepackage[percent]{overpic}
%\usepackage{hepunits}
\usepackage{xspace}
\usepackage{cite}
%to include txt files (HERAFITTER steering files)
\usepackage{verbatim}
\usepackage[retain-explicit-plus]{siunitx}
\usepackage{lineno}
%\linenumbers % numbering the lines 
\usepackage[hidelinks]{hyperref}
\usepackage{booktabs}
\usepackage{multirow}
\usepackage{cleveref}

\crefname{figure}{Figure}{Figures}
\crefname{table}{Table}{Tables}
\usepackage{comment} %Added by Andrea, useful when correcting to keep text in tex file and exclude from pdf
\usepackage{lipsum}                     % Dummytext
\usepackage{xargs}                      % Use more than one optional parameter in a new commands
\usepackage[pdftex,dvipsnames]{xcolor}  % Coloured text etc.
% To make a special notes
\usepackage[colorinlistoftodos,prependcaption,textsize=tiny]{todonotes}
\newcommandx{\unsure}[2][1=]{\todo[linecolor=red,backgroundcolor=red!25,bordercolor=red,#1]{#2}}
\newcommandx{\change}[2][1=]{\todo[linecolor=blue,backgroundcolor=blue!25,bordercolor=blue,#1]{#2}}
\newcommandx{\info}[2][1=]{\todo[linecolor=OliveGreen,backgroundcolor=OliveGreen!25,bordercolor=OliveGreen,#1]{#2}}
\newcommandx{\improvement}[2][1=]{\todo[linecolor=Plum,backgroundcolor=Plum!25,bordercolor=Plum,#1]{#2}}
\newcommandx{\thiswillnotshow}[2][1=]{\todo[disable,#1]{#2}}
\newcommandx{\reminder}[2][1=]{\todo[linecolor=red,backgroundcolor=OliveGreen!25,bordercolor=red,#1]{#2}}

\usepackage{physics}
\usepackage[rightcaption]{sidecap}

\usepackage{calrsfs}
\DeclareMathAlphabet{\pazocal}{OMS}{zplm}{m}{n}

% To make nice first letter for the chapters:
\usepackage{lettrine}
\usepackage{type1cm}
\LettrineRealHeighttrue
\newcommand{\colorlettrine}[2]{\lettrine[realheight=true]{\color{blue}\textbf{#1}}{#2}}

%to make feynman diagrams

%add top left logo
%\usepackage{eso-pic}% http://ctan.org/pkg/eso-pic
%\usepackage{graphicx}% http://ctan.org/pkg/graphicx
\usepackage{xcolor}

% page settings
\textwidth=16cm
\oddsidemargin=5mm
\evensidemargin=-5mm

\usepackage{fancyhdr}
\pagestyle{fancy}
\fancyhf{}
% \fancyhead[LE,RO]{\leftmark}
\fancyhead[RE,LO]{\leftmark}
\fancyfoot[C]{\thepage}
% \renewcommand{\headrulewidth}{0pt}

% include the file with command definitions
%%%%%%%%CP%%%%%%%%
\newcommand{\phicp}{\ensuremath{\phi_{\textit{CP}}}\xspace}
\newcommand{\mixa}{\ensuremath{\alpha^{\text{H}\tau\tau}}\xspace}

%%%%%%%%Theory%%%%%%%%
\newcommand{\Lagr}{\mathcal{L}}
\newcommand{\AH}{\ensuremath{A/H}\xspace}

%%%%%%Kinematic %%%%%%%%
\newcommand{\pt}{\ensuremath{p_{\text{T}}}\xspace}
\newcommand{\met}{\ensuremath{\vec{p}_\text{T}^{\text{miss}}}\xspace}
\newcommand{\HT}{\ensuremath{H_{\text{T}}}\xspace}
\newcommand{\dEta}{\ensuremath{\Delta \eta}\xspace}
\newcommand{\dPhi}{\ensuremath{\Delta \phi}\xspace}
\newcommand{\dR}{\ensuremath{\Delta R}\xspace}
\newcommand{\ABSdEta}{\ensuremath{|\Delta \eta|}\xspace}
\newcommand{\ztt}{\ensuremath{Z \rightarrow \tau\tau}\xspace}
\newcommand{\htt}{\ensuremath{\text{H} \rightarrow \tau\tau}\xspace}

%%%%%% Particles %%%%%%%%
\newcommand{\piz}{\ensuremath{\pi^0}\xspace}
\newcommand{\tauh}{\ensuremath{\tau_h}\xspace}
\newcommand{\h}{\ensuremath{\text{h}^\pm}\xspace}


%%%%%%%%%%%%%%%%%%%%%%%%%%

\newcommand{\MA}{\ensuremath{m_{\text{A}}}\xspace}
\newcommand{\mA}{\ensuremath{m_{\text{A}}}\xspace}
\newcommand{\Mh}{\ensuremath{m_{\text{h}}}\xspace}
\newcommand{\mh}{\ensuremath{m_{\text{h}}}\xspace}
\newcommand{\MH}{\ensuremath{m_{\text{H}}}\xspace}
\newcommand{\mH}{\ensuremath{m_{\text{H}}}\xspace}
\newcommand{\MC}{\ensuremath{m_{\text{H}^{\pm}}}\xspace}
\newcommand{\mC}{\ensuremath{m_{\text{H}^{\pm}}}\xspace}
\newcommand{\Mphi}{\ensuremath{m_{\phi}}\xspace}
\newcommand{\Mbb}{\ensuremath{M_{12}}\xspace}
\newcommand{\hSM}{\ensuremath{\text{h}(125)}\xspace}
\newcommand{\MAH}{\ensuremath{m_{\mathrm{\AH}}}\xspace}
\newcommand{\mAH}{\ensuremath{m_{\mathrm{\AH}}}\xspace}
\newcommand{\cosBA}{\text{cos}\ensuremath{(\beta - \alpha)}\xspace}

%%%%%%%%%%%%%%%%%%%%%%%%%%%
\newcommand{\bbPhi}{\ensuremath{b\bar{b}\phi}\xspace}
\newcommand{\bbAH}{\ensuremath{b\bar{b}\AH}\xspace}
\newcommand{\bb}{\ensuremath{b\bar{b}}\xspace}
\newcommand{\bbmu}{\ensuremath{b\bar{b}\mu}\xspace}
\newcommand{\ttbar}{\ensuremath{t\bar{t}}\xspace}

%%%%%%%
\newcommand{\fbi}{\text{fb}\ensuremath{^{-1}}\xspace}
\newcommand{\pbi}{\text{pb}\ensuremath{^{-1}}\xspace}

%

\newcommand{\sTop}{\ensuremath{\tilde{t}}\xspace}
\newcommand{\sTau}{\ensuremath{\tilde{\tau}}\xspace}
\newcommand{\bbar}{\ensuremath{b\bar{b}}\xspace}
\newcommand{\tanB}{\ensuremath{\tan}\ensuremath{\beta}\xspace}
\newcommand{\tanb}{\ensuremath{\tan}\ensuremath{\beta}\xspace}
\newcommand{\cosba}{\ensuremath{\cos}\ensuremath{(\beta - \alpha)}\xspace}
\newcommand{\mSUSY}{\ensuremath{M_{SUSY}}\xspace}
%\newcommand{\M2}{\ensuremath{M_{2}}\xspace}
\newcommand{\CLNF}{\ensuremath{95\%\ \text{C.L.}}\xspace}

% results interpretation: MSSM

%\newcommand{\mhmodp}{\ensuremath{m_{\Phi}^{\text{mod+}}}\xspace}
%\newcommand{\mhmodm}{\ensuremath{m_{\Phi}^{\text{mod--}}}\xspace}
\newcommand{\mhmod}{\ensuremath{m_{\text{h}}^{\text{mod}}}\xspace}
%\newcommand{\mhmax}{\ensuremath{m_{\Phi}^{\text{max}}}\xspace}

\newcommand{\mhmodp}{\ensuremath{m_{\text{h}}^{\text{mod+}}}\xspace}
\newcommand{\mhmodm}{\ensuremath{m_{\text{h}}^{\text{mod--}}}\xspace}
\newcommand{\mhmax}{\ensuremath{m_{\text{h}}^{\text{max}}}\xspace}
\newcommand{\mhonetwofive}{\ensuremath{m_{\text{h}}^{125}}\xspace}
\newcommand{\lightstop}{\text{light-}\ensuremath{\tilde{t}}\xspace}
\newcommand{\lightstau}{\text{light-}\ensuremath{\tilde{\tau}}\xspace}
\newcommand{\tauphobic}{\ensuremath{\tau}\text{-phobic}\xspace}
\newcommand{\typeII}{\ensuremath{type-II}\xspace}
\newcommand{\flipped}{\ensuremath{flipped}\xspace}

% otherr
\newcommand{\red}{\color{red}}
\newcommand{\ifb}{fb$^{-1}$\xspace}

% cos/sin/tan etc

\newcommand*\xbar[1]{%
  \hbox{%
    \vbox{%
      \hrule height 0.5pt % The actual bar
      \kern0.5ex%         % Distance between bar and symbol
      \hbox{%
        \kern-0.1em%      % Shortening on the left side
        \ensuremath{#1}%
        \kern-0.1em%      % Shortening on the right side
      }%
    }%
  }%
}

% for the Feynman diagrams
\usepackage{feynmp}


% %Define DESY-orange
% \definecolor{DESYOrange}{HTML}{ed9a02}
% % Nice chapter style
% \usepackage[Lenny]{fncychap}
% \ChNameUpperCase
% \ChNumVar{\fontsize{60}{62}\usefont{OT1}{ptm}{m}{n}\selectfont\textcolor{DESYOrange}}
% \ChTitleVar{\Huge\bfseries}

% Chapter style
\usepackage{titlesec, blindtext, color}
\definecolor{gray75}{gray}{0.75}
\newcommand{\hsp}{\hspace{20pt}}
\titleformat{\chapter}[hang]{\Huge\bfseries}{\thechapter\hsp\textcolor{gray75}{|}\hsp}{0pt}{\Huge\bfseries}

%\usepackage[nouppercase]{scrpage2}
%Appendix pdf proceedings
\usepackage{pdfpages}

% distance between parahraph breaks 
\setlength{\parskip}{.8em} 
\parindent=15pt

% quotation marks
\usepackage{csquotes}

% links in bibliography
\usepackage{hyperref}
\usepackage{url}

% ref enumerate items
\usepackage{enumitem}

% math 
\usepackage{bm}

% line breaks in tabular cells
\usepackage{makecell}

% line numbers 
\usepackage[switch, modulo]{lineno}
\linenumbers

% large vector symbols
\usepackage{esvect}

% % to show page numbers for fancy package
% \usepackage{afterpage}
% \usepackage{setspace}

%%%%%%%%%%%%%%%%%%%%%%%%%%%%%%%%%%%%%%%%%%%%%%%%
%%%%%%%%%%%%%%%%%%%%%%%%
%%%%%%%%%%%%%%%%%%%%%%%% body of the document
%%%%%%%%%%%%%%%%%%%%%%%%
%%%%%%%%%%%%%%%%%%%%%%%%%%%%%%%%%%%%%%%%%%%%%%%%


\begin{document}

\makeatletter
\renewcommand{\@evenhead}{{\vbox{\hbox to\textwidth{\normalfont\slshape\thepage \hfil \strut \leftmark}\hrule}}}
\renewcommand{\@oddhead}{{\vbox{\hbox to\textwidth{\normalfont\slshape\rightmark \hfil \strut \thepage}\hrule}}}
%for bold hlines:
\def\hlinewd#1{%
\noalign{\ifnum0=`}\fi\hrule \@height #1 %
\futurelet\reserved@a\@xhline} 
\makeatother
\newcommand{\sepspace}{\vspace*{1em}}

% %\pagestyle{scrheadings}
% \makeatletter
% \newcommand{\insertheaderrule}{\rlap{\rule[-.1\normalbaselineskip]{\textwidth}{.4pt}}}
% \let\old@evenhead\@evenhead \let\old@oddhead\@oddhead
% \def\@evenhead{\insertheaderrule\old@evenhead}% Prepend \insertheaderrule
% \def\@oddhead{\insertheaderrule\old@oddhead}% Prepend \insertheaderrule
% \makeatother



%%%%%%%%%% front matters
% \frontmatter
% %----------------------------------------------------------------------------------------
%	TITLE PAGE
%----------------------------------------------------------------------------------------

\begin{titlepage} % Suppresses headers and footers on the title page

%\vspace*{-4cm}
%\hspace*{-1.5cm}
%\includegraphics[height=3cm]{Figures/general/up-uhh-logo.png}
%\hfill
%\includegraphics[height=3cm]{Figures/general/DESY_logo_4C.png}

\vspace{2cm}
    \centering

	%\vspace{2.\baselineskip} % Whitespace above the title
	%\vspace{-2cm}
%	\noindent\makebox[\linewidth]{\rule{1.05\textwidth}{1.5pt}} \\
	%\rule{\textwidth}{1.5pt}\\
	\vspace{0.4cm}
    \parbox[t]{0.96\textwidth}{\rmfamily \setlength\parfillskip{0pt} \def\baselinestretch{1.1}\centering\Huge \bfseries Tau identification algorithms and study of the CP structure of the Yukawa coupling between the Higgs boson and tau leptons in CMS}\\
	\vspace{0.4cm}
%	\noindent\makebox[\linewidth]{\rule{1.05\textwidth}{1.5pt}} \\
    %\rule{\textwidth}{1.5pt}\\
    \vspace{2 cm} % Whitespace below the title
	
	
    \parbox[p]{0.95\textwidth}{\def\baselinestretch{1.4}\centering\scshape\textbf{\LARGE Dissertation} \\ {\large zur Erlangung des Doktorgrades an der Fakult{\"a}t\\ f{\"u}r Mathematik, Informatik und Naturwissenschaften \\ Fachbereich Physik \\ der Universit{\"a}t Hamburg} }
    
    \vspace{2 cm} % Whitespace after this block

    \parbox[b]{0.93\textwidth}{\def\baselinestretch{1.3}\centering\upshape \Large vorgelegt von \\ {\LARGE\scshape \bfseries Oleg Filatov } \\ %aus \\ {\scshape \bfseries XXXXXX}
    }
    
    \vspace{2 cm} % Whitespace after this block
    
    \parbox[b]{0.93\textwidth}{\def\baselinestretch{1.3}\centering\upshape\large Hamburg\\ 2022}

\end{titlepage}

%----------------------------------------------------------------------------------------
%----------------------------------------------------------------------------------------
 
 
\vfill\vspace{-2cm}\par
\thispagestyle{empty}
%\cleardoublepage

\thispagestyle{empty}
\section*{Eidesstattliche Erkl\"arung / Declaration on oath}
\vspace{1cm}
\begin{center}
    
\parbox[b]{0.85\textwidth}{
  
Ich erkl\"are hiermit an Eides statt, dass ich diese Dissertation selbst verfasst und keine anderen als die angegebenen Hilfsmittel oder Quellen benutzt habe.\\

I hereby declare in lieu of oath that I have written this dissertation myself and that I have not used any auxiliary materials or sources other than those indicated.\\\\
\vspace{1cm}

\hspace{0.2cm} Hamburg, \today \hfill 
\vspace{0.5cm}

\hfill Unterschrift des Doktoranden \hspace{1.0cm} 
}

\end{center}
\vfill

\newpage
\thispagestyle{empty}
\cleardoublepage

\newpage
\thispagestyle{empty}
\parbox[b]{\textwidth}{
\renewcommand{\arraystretch}{2.0}
\begin{tabular*}{\linewidth}{p{0.6\linewidth}p{0.4\linewidth}}
 
% (alle Angaben vorl\"aufig) & \\ \\
 
 Gutachter/innen der Dissertation:
  & Dr. Alexei Raspereza\\
  & Prof. Dr. Elisabetta Gallo \\ 
  \\
  
 Zusammensetzung der Pr{\"u}fungskommission: 
  & Prof. Dr. Sven-Olaf Moch \\
  & Prof. Dr. Elisabetta Gallo \\
  & Dr. Alexei Raspereza \\
  & Jun.-Prof. Dr. Gregor Kasieczka\\
  & Dr. Judith Katzy\\
  
 Vorsitzende/r der Pr{\"u}fungskommission: & Prof. Dr. Sven-Olaf Moch\\
 \\
 
 Datum der Disputation: & 19.01.2023 \\
 \\
 
 Vorsitzender des Fach-Promotionsausschusses PHYSIK: & Prof. Dr. G{\"u}nter H. W. Sigl \\
%Vorsitzende des Promotionsausschusses: & Prof. Dr. Jan Louis \\
 \\

 Leiter des Fachbereichs PHYSIK: & Prof. Dr. Wolfgang J. Parak \\
%Leiter des Fachbereichs Physik: & Prof. Dr. Peter Hauschildt \\
 \\

 Dekan der Fakult{\"a}t MIN: & Prof. Dr.-Ing. Norbert Ritter \\
%Dekan der Fakult\"ar f\"ur Mathematik, & \\
%Informatik und Naturwissenschaften:  & Prof. Dr. Heinrich Graener \\
% 
 \end{tabular*}
 \\ \\ 
 }

\clearpage

% \newpage
% \thispagestyle{empty}
% \mbox{}
% %back of the title
% %\input{include/title_back.tex}
% \clearpage
% % dedication
% %\input{include/dedication.tex}

% % abstract
% \thispagestyle{empty}
\vspace{-3cm}
\section*{\centering Abstract}

\noindent

The measurement of the CP properties of the Yukawa coupling of the Higgs boson to $\tau$ leptons is presented. The data set used for the analysis is collected by the CMS experiment at the LHC during the Run 2 data-taking period in proton-proton collisions at $\sqrt{s}=13$ TeV and corresponds to an integrated luminosity of 137 \ifb. The Yukawa coupling between the Higgs boson and $\tau$ leptons is parametrised in terms of the effective mixing angle \mixa, where the value $\mixa = 0^\circ (90^\circ)$ corresponds to the SM scenario of the pure CP-even (CP-odd) $\text{H}\tau\tau$ coupling. 

The angle between the decay planes of the $\tau$ leptons is used as the observable encoding the CP nature of the Higgs boson. The measurement is performed in the \et channel where one $\tau$ lepton decays into a single electron and the other hadronically. The results are combined with the measurement in the \mt and \tata channels. The observed (expected) value of the effective mixing angle for the combination is measured to be:

\begin{equation}
    \mixa = -1 \pm 19^\circ (0 \pm 21^\circ) ~@68.3\% \text{ CL}.
\end{equation}

The results are compatible with the SM expectation and the pure CP-odd hypothesis is rejected at the observed (expected) significance level of $3.0 (2.6)$ standard deviations.

The improvements to the $\tau$ lepton identification in CMS in the context of the Run 3 preparation are described. Retraining and optimisation of the DeepTau algorithm with addition of the adversarial fine-tuning procedure is performed. The resulting model improves upon the previous DeepTau model in terms of the background rejection by 10-50\% and has a better description of data with simulation in the \htt selection region. 

A new algorithm called Tau Transformer (TaT) is proposed to overcome the limitations of the DeepTau architecture. The TaT core is based on self-attention layers and features the embedding module allowing for the multimodality treatment of the input representation. Comparison of the TaT model with the retrained DeepTau model and a comparable ParticleNet-based architecture shows consistently improved performance by up to 50\% in the misidentification rate across the \pt, $\eta$, and decay mode ranges of interest.   

\newpage
\thispagestyle{empty}
\mbox{}


% %\clearpage
% % abstract - German version
% \thispagestyle{empty}
\vspace{-3cm}
\section*{\centering Zusammenfassung}
\noindent

\newpage
\thispagestyle{empty}
\mbox{}

% \clearpage

% %\input{Articles/articles.tex}
% %\input{02_Epigraph/epigraph.tex}

% %This is commented out to have page numbers on outer side of the page
% %\newpage
% %\thispagestyle{empty}
% %\mbox{}	


% reset numbering for the table of contents
\clearpage
\pagenumbering{roman}

%%%%%%%%%%table of contents
\tableofcontents

%\input{conclusion.tex}
%\newpage
%\thispagestyle{empty}
%\mbox{}

% reset the numbering for the main text
%\clearpage
%\pagenumbering{arabic}

%%%%%%%%%% main text
\mainmatter
\pagestyle{fancy}

% \chapter{Introduction}
% \thispagestyle{empty}






%\end{abstract}
\chapter{CP analysis in $\text{H} \to \et$ decays}\label{sec:cp-etau}
\section{Introduction}\label{sec:cp-intro}

In order to probe the CP nature of the interaction between the Higgs boson and tau leptons one needs to modify the SM Lagrangian in a way to incorporate effects deviating from the Standard Model expectations. This can be done already at the tree level and without assuming any model dependence. For the tau lepton case, one can write the following extension of the Yukawa coupling term in the SM Lagrangian \cite{Gritsan:2016hjl}:
\begin{equation}\label{eq:l_y}
    \mathcal{L}_Y = -\frac{m_\tau}{v}(\kappa_\tau\bar{\tau}\tau + \tilde{\kappa}_\tau\bar{\tau}i\gamma_5\tau) \text{H}.
\end{equation}

Here $m_\tau = 1.776(86)$ GeV is the mass of the tau lepton, $v=\left(\sqrt{2G_\text{F}}\right)^{-1/2} \approx 246$ GeV is the vacuum expectation value of the Higgs field (Sec. \ref{sec:break}), $G_\text{F} = 1.1663787(6) \cross 10^{-5}$ GeV$^{-2}$ is the Fermi constant, $\kappa_\tau$ and $\tilde{\kappa}_\tau$ are the coupling strength modifiers, H is the SM Higgs boson. While this term is written specifically for the SM scenario, it is generally applicable to any neutral spin-zero particle H of arbitrary CP nature with a flavor-diagonal Yukawa interaction with a fermion. 

Few things should be noted regarding Eq. \ref{eq:l_y}. First, while such parametrisation is model-independent \textit{per se}, the couplings $\kappa_\tau$ and $\tilde{\kappa}_\tau$ can depend on the specific model under consideration and can be interpreted within the framework of these theories. For example, in the context of a two-Higgs doublet model, e.g. minimal supersymmetric SM
extension (MSSM) \cite{Accomando:2006ga}, depending on the CP violation (CPV) scenario, the three mass eigenstates $h_i$ can be either two scalars (denoted as $h, H$) and one pseudoscalar (denoted as $A$) or states representing CP mixtures with both $\kappa$ and $\tilde{\kappa}$ couplings having non-zero values. In this work, no model-dependent interpretation of the couplings is made.    

Second, an assumption of real-valued $\kappa$ and $\tilde{\kappa}$ is made. While this makes the Yukawa coupling term Hermitian similarly to the rest of the SM Lagrangian, it is arguably an imposed property which might not hold true in nature \cite{Mannheim:2021kjs}. Therefore, a generalisation of the study presented in this work towards a non-Hermitian Yukawa coupling \cite{Korchin:2016rsf, Korchin:2021xxl} is an interesting direction for future studies.

From the couplings formulation one can rewrite Eq. \ref{eq:l_y} as:
\begin{equation}\label{eq:l_y_angle}
    \mathcal{L}_Y = -g_\tau(\cos\alpha^{H\tau\tau}\bar{\tau}\tau + \sin\alpha^{H\tau\tau}\bar{\tau}i\gamma_5\tau)\text{H},
\end{equation}

where $g_\tau$ is a generalised coupling modifier and an effective mixing angle is introduced:
\begin{equation}\label{eq:mixa}
    \tan(\alpha^{H\tau\tau}) = \frac{\tilde{\kappa}_\tau}{\kappa_\tau}.
\end{equation}

For the SM Higgs boson with the quantum numbers $J^{PC} = 0^{++}$ (pure scalar), $\tilde{\kappa}_\tau = 0$ and $\kappa_\tau = 1$, therefore $\mixa = 0^\circ$. The scenario of $J^{PC} = 0^{++}$ (pure pseudoscalar) corresponds to $\mixa = 90^\circ$. Any intermediate value corresponds to the mixture of the couplings between CP-even and CP-odd components.

Having the interaction defined in terms of the Lagrangian term, one can proceed to the derivation of the partial decay width of the SM Higgs boson into a pair of tau leptons. Using Eq. \ref{eq:l_y_angle} with the approximation $\beta_\tau = \sqrt{1-4m_\tau^2/m_h^2}\approx1$ one can obtain \cite{Berge:2014wta}:
\begin{equation}
    d\Gamma_{\htt} \sim 1 - s^+_zs^-_z + \cos(2\mixa)(\boldsymbol{s}_T^+\cdot \boldsymbol{s}_T^-) + \sin(2\mixa)\left[(\boldsymbol{s}_T^+\cross \boldsymbol{s}_T^-)\cdot\hat{\boldsymbol{k}}^-\right ],
\end{equation}

where $\hat{\boldsymbol{k}}^-$ is a normalised $\tau^-$ momentum in the Higgs rest frame which points towards a positive direction of the $z$ axis, and $\boldsymbol{s}_\text{T}^\pm$ ($s_z^\pm$) is a projection of normalised spin vector of the tau lepton in its rest frame on the $xy$ plane ($z$ axis) in a right handed coordinate system. It can be seen that it is the spin correlation between transverse components of the tau leptons' spin vectors which is sensitive to the CP structure, parametrised by \mixa. Introducing $\phi_{s}$ as an angle pointing from $\boldsymbol{s}_T^+$ to $\boldsymbol{s}_T^-$, one obtains:

\begin{equation}\label{eq:h_width_spin}
    d\Gamma_{\htt} \sim 1 - s^+_zs^-_z + |\boldsymbol{s}_T^+||\boldsymbol{s}_T^-|\cos(\phi_{s}-2\mixa).
\end{equation}

Conceptually, Eq. \ref{eq:h_width_spin} lays out the strategy to experimentally probe the CP structure of the Yukawa coupling between the Higgs boson and tau leptons. First, one needs to reconstruct the $\phi_{s}$ angle between the spin vectors $\boldsymbol{s}_T^\pm$ of the tau leptons. This can be achieved by studying the angular distributions of $\tau$ decay products, as described in Sec. \ref{sec:phicp}. Second, in a simplified picture the differential distribution of the $\phi_{s}$ angle will allow to extract the phase shift $\mixa$ from the fit with $a\cdot\cos(\phi - 2\mixa)+b$ function. This in turn directly points to the CP nature (CP-even, CP-odd, or their mixture) of the SM Higgs boson via Eq. \ref{eq:mixa}. 

In the following sections of this chapter a step-by-step overview towards this goal is described. Starting from the description of the data sets used in the analysis (Sec. \ref{sec:samples}), an overview of physics objects and observable reconstruction is given in Sec. \ref{sec:reco}. In Sec. \ref{sec:selection} a procedure to select \htt candidates is described, followed by techniques to model background processes (Sec. \ref{sec:bkgr}). After the selection of the H candidates is performed, ML methods are used to categorise a given candidate as either originating from a signal or background processes (Sec. \ref{sec:categ}). Taking into account necessary systematic uncertainties (Sec. \ref{sec:syst}), a statistical inference procedure is performed (Sec. \ref{sec:stat}) to extract the effective mixing angle \mixa. Finally, results of the measurement and the corresponding conclusion are given in Sec. \ref{sec:results}. 

\section{Data \& Simulation}\label{sec:samples}

For this work, a data set of $pp$ collisions collected by the CMS detector at $\sqrt{s}=13~\text{TeV}$ in 2016, 2017, and 2018 years is used. The corresponding integrated luminosities are $35.9$, $41.5$, and $59.7$ \fbi.

Several Monte Carlo simulated data sets are produced in order to model signal and background processes. The signal processes consist of a Higgs boson being produced through the gluon-gluon fusion (ggH), vector boson fusion (VBF), or associated production with a W or Z boson (WH, ZH, or VH for their combination). These samples are generated at next-to-leading order (NLO) in QCD with the POWHEG 2.0 event generator \cite{Nason:2004rx,Frixione:2007vw,Alioli:2010xd,Bagnaschi:2011tu,Nason:2009ai,Jezo:2015aia,Granata:2017iod}. The procedure is configured to produce a scalar Higgs boson. However, addition of CP mixing effects in the production mechanism, for example, by modifying the Higgs coupling to top and bottom quarks, can affect the distribution of physical observables (e.g. related to the accompanying jets), as well as the signal acceptance. It is studied that this contribution is negligible comparing to the theoretical uncertainties and therefore does not affect the CP measurement in the \htt decay.

Reweighting is applied to distributions of the Higgs boson transverse momentum and the jet multiplicity to match with those of the samples produced at next-to-NLO with the POWHEG NNLOPS (version 1) generator \cite{Hamilton:2013fea,Hamilton:2015nsa}. The decay of the Higgs boson into a pair of tau leptons is described by the PYTHIA generator version 8.230 \cite{Sjostrand:2014zea} without accounting for the $\tau$ spin correlations. These are included within the TAUSPINNER package \cite{Przedzinski:2018ett}, which reweights the signal samples according to predefined values of the mixing angle $\mixa=\{0^\circ, 45^\circ, 90^\circ\}$ chosen to define the signal templates for the statistical inference (Sec. \ref{sec:temp}). For all 2016 samples NLO parton distribution functions (PDFs) are generated with the NNPDF3.0 \cite{NNPDF:2014otw}. For all 2017 and 2018 samples NNLO PDFs distributions are generated with the NNPDF3.1 \cite{NNPDF:2017mvq}.

Processes with a Z or W boson accompanied by up to four outgoing partons are generated with MADGRAPH5 aMC@NLO (version 2.6.0) \cite{Alwall:2014hca}. W bosons originating from the top quark decay are generated at leading order with the MLM jet matching and merging approach \cite{Alwall:2007fs}, as well as the diboson production at NLO. POWHEG 2.0 (1.0) is used for single top (ST) quark production (associated with a W boson) \cite{Re:2010bp,Frederix:2012dh} and top quark-antiquark pair production \cite{Alioli:2011as}. For modelling of the parton showering, fragmentation, and the decay of the $\tau$ lepton the generators are interfaced with PYTHIA with its parameters set to the CUETP8M1 tune \cite{CMS:2015wcf} , and CP5 tune \cite{CMS:2019csb} in 2017 and 2018. 

The simulation of the CMS detector is based on GEANT 4 \cite{GEANT4:2002zbu}. Additional $pp$ interactions per bunch crossing (also referred to as pileup interactions) are generated with PYTHIA and reweighted to match the pileup distribution in data. 

\section{Event reconstruction}\label{sec:reco}
The particle-flow (PF) algorithm (Sec. \ref{pf}) is at the core of the physics object reconstruction in CMS. It builds upon the idea of combining information from all the subsystems of the detector in order to improve the overall reconstruction efficiency. Using a hierarchical approach which starts from the construction of fundamental building blocks (e.g. tracks or clusters) it further combines them into high-level physics objects such as muons or charged hadrons. Furthermore, it serves as a basis for other algorithms building more complex objects, such as the jet clustering (Sec. \ref{sec:jets}) or the hadron-plus-strips algorithm (Sec. \ref{hps}). 

\subsection{Electrons}\label{sec:reco_e}
Electron object reconstruction \cite{CMS:2020uim} also builds on top of the PF basic elements: GSF tracks and ECAL clusters (Sec. \ref{sec:pf_base}). Conceptually, these elements are further combined, refined and filtered to yield a final electron object in the following procedure: 
\begin{enumerate}

\begin{figure}[ht!]
    \centering
    \includegraphics[width=0.65\textwidth]{Figures/CP_etau/mustache.png}
    \caption{Distribution of PF clusters around the seed cluster for simulated electrons with $1 < E_T^{\text{seed}} < 10~\text{GeV}$ and $1.48 < \eta^{\text{seed}} < 1.75$ \cite{CMS:2020uim}. The $z$ axis shows the number of PF clusters around the seed matched to simulation. The red line illustrates the region where the clusters are selected by the mustache algorithm.}
    \label{fig:mustache}
\end{figure}

    \item ECAL clusters are combined into a supercluster (SC) with a so-called mustache algorithm (Fig. \ref{fig:mustache}). The idea is to aggregate clusters coming from extensive bremsstrahlung and photon conversion within a \enquote{mustache} window in $\eta$-$\phi$ plane which accounts for the magnetic field of the CMS detector.

    \item Association of SCs with GSF tracks is performed based on the output of a boosted decision tree (BDT) using as input SCs observables, track parameters and the SC-GSF matching variables.
    \item Refinement of the mustache SCs is done, which leverages the information from subdetectors outside of ECAL. This step recovers additional bremsstrahlung and conversion clusters. Moreover, a conversion-finding algorithm \cite{CMS:2015myp} with a dedicated BDT are used to identify pairs of tracks compatible with a converted photon.
    
    \item All the input elements (ECAL clusters, mustache SCs, electron associated generic tracks, GSF tracks, conversion-identified tracks) are submitted to the PF algorithm to form electron candidates. After the linking, the final set of ECAL clusters for each candidate is promoted to a refined supercluster.
    
    \item Final electron objects are formed from a refined SC with an associated GSF track based on the loose requirements on the BDT output. The BDT is trained using the shower-shape, isolation and track-related variables as input.
\end{enumerate}

\begin{figure}[t!]
    \centering
    \includegraphics[width=0.65\textwidth]{Figures/CP_etau/e_reco_eff.png}
    \caption{Electron reconstruction efficiency versus $\eta$ for various \pt ranges (upper panel) and ratios of data and simulation efficiencies (lower panel) in 2017 data taking period \cite{CMS:2020uim}.}
    \label{fig:e-reco-eff}
\end{figure}

Overall, the procedure results in a good efficiency of electron reconstruction across \pt and $\eta$ ranges (Fig. \ref{fig:e-reco-eff}). However, it should be noted that a graph neural network (GNN) based algorithm to form supercluster has been recently proposed to recover for inefficiency of mustache energy aggregation and also to provide better robustness to pileup \cite{Valsecchi:2022rla}.

\begin{figure}[ht!]
    \centering
    \includegraphics[width=0.48\textwidth]{Figures/CP_etau/e_reso.png}
    \includegraphics[width=0.48\textwidth]{Figures/CP_etau/m_zee.png}
    \caption{Left: relative energy resolution as a function of electron \pt, measured by the tracker, by ECAL (\enquote{corrected SC}), and after the third step of the energy regression (\enquote{E-p combination}). Right: invariant mass of an electron pair in the barrel region from $Z\to ee$ events in 2017 data before and after applying regression and scale corrections. \cite{CMS:2020uim}.}
    \label{fig:e_corr}
\end{figure}

Since the energy of electrons is not fully reconstructed due to losses in the tracker or shower leakage in ECAL, corresponding corrections should be applied. This is achieved by firstly performing correction of SC energy and resolution via a 3-step BDT regression.  Second, residual discrepancies between data and simulation are taken into account with energy scale and smearing corrections derived from $Z \to ee$ events. These results in a significant improvement both on the side of energy resolution (Fig. \ref{fig:e_corr}, left) and physical observables (Fig. \ref{fig:e_corr}, right).

After the reconstruction of electron objects an identification step follows. Since the reconstruction algorithms are designed to be general-purpose and as inclusive as possible, it results in a sizeable fraction of background objects in the reconstructed electron collection. The identification step aims at the separation of prompt (created in the primary $pp$ interaction) genuine electrons from misidentified objects or non-prompt electrons (usually from heavy flavour jets). For that purpose, two methods are used. The first one is a cut-based discriminator based on the isolation variable:
\begin{equation}\label{eq:iso_e}
    I^e_{\text{rel}} = \dfrac{\sum \pt(\text{h}^\pm) + \max\left(\sum \pt(\text{h}^0) + \pt(\gamma) - \rho \cdot A_\text{eff}, 0\right)}{\pt^e},
\end{equation}
where $A_\text{eff}$ is the $\eta$-dependent isolation area \cite{CMS:2015xaf}, $E_T \equiv \sqrt{m^2 + \pt^2}$, $\rho$ is the average neutral component of the pileup energy density per unit area in the $\eta$-$\phi$ plane, and the sums are computed across PF candidates of a given type within a cone $\Delta R \equiv \sqrt{\Delta\eta^2+\Delta\phi^2}< 0.3$ around the reconstructed electron. The sum over the charged PF candidates runs over the candidates associated with PV, while the sum over the neutral PF candidates does not have this requirement. Thresholds on this discriminator are derived to target specific predefined selection efficiencies. In this work, the requirement $I^e_{\text{rel}} < 0.15$ is applied to the selected electron objects (Sec. \ref{sec:selection}). 

The second method of electron identification is based on a boosted decision tree. It uses information about the track-cluster matching and energy deposits in HCAL/ECAL, as well as cluster-shape, track-quality variables and provides a score for a reconstructed electron object to be a genuine prompt electron. Working points are defined as thresholds on the score to target predefined electron selection efficiency. In this work, a working point corresponding to 90\% efficiency is used.

Lastly, additional requirements are applied on the transverse ($|d_{xy}| < 0.045$ cm) and longitudinal ($|d_z| < 0.2$ cm) impact parameters of the selected electrons.

\subsection{Muons}\label{sec:reco_mu}

Muon reconstruction relies on the standalone algorithm as described in Sec. \ref{sec:pf_base}. An identification step for muons is based on a set of requirements aimed to provide a predefined selection efficiency. In this work the muon object is required to be reconstructed as a tracker or global muon and pass the hit and segment compatibility quality selection. Same requirements on the impact parameters as in case of electrons are applied: $|d_{xy}| < 0.045$ and $|d_z| < 0.2$. The isolation variable is also defined as:
\begin{equation}\label{eq:iso_mu}
    I^\mu_{\text{rel}} = \dfrac{\sum \pt(\text{h}^\pm) + \max\left(\sum \pt(\text{h}^0) + \pt(\gamma) - \frac{1}{2}\sum \pt(\text{h}^\pm_{\text{PU}}), 0\right)}{\pt^\mu},
\end{equation}

where the sums are taken for the PF candidates in the isolation cone $\Delta R < 0.4$ centered around the reconstructed muon direction of flight. The sum $\sum \pt(\text{h}^\pm_{\text{PU}})$ is computed over the charged PF candidates originating from pileup vertices and scaled down by a factor 1/2 to approximate and subtract the pileup contribution from neutral particles. A requirement $I^\mu_{\text{rel}} < 0.15$ is also applied.  

\subsection{Tau leptons}\label{sec:reco_tau}
Tau leptons decaying hadronically ($\tauh$) are reconstructed with a dedicated hadron-plus-strips (HPS) algorithm as described in Sec. \ref{hps}. First, it aims to reconstructs $\pi^0$ coming from the \tauh decays in a form of \enquote{strips}. Second, it combines them with charged hadrons to form potential \tauh candidates according to the expected decay modes (DM) (Sec. \ref{tau-intro}).

For the identification step, a DeepTau model (Sec. \ref{deeptau1}) is used to separate \tauh candidates, reconstructed by the HPS algorithm, from jets, electrons, and muons. The model is built from 1D and 2D convolutional layers operating on a grid in the $\eta$-$\phi$ plane centered around the HPS-reconstructed \tauh candidate. It combines low-level information from PF candidates and RECO electrons/muons to  separate between genuine \tauh and fakes. For this work, the \tauh candidate is required to pass the working points which correspond to the probability of 70\%, 80\%, and 99.95\% (Medium, Tight, Very loose, respectively) for the genuine \tauh to pass DeepTau discriminators against jets, electrons, and muons, respectively. Furthermore, the $z$ component of the impact parameter of the leading charged track with respect to PV is required to be $|d_z| < 0.2$ cm.

\begin{figure}[t!]
    \centering
    \includegraphics[width=0.9\textwidth]{Figures/CP_etau/svfit.png}
    \caption{Distribution of $m_{\tau\tau}$ variable derived with the SVFit algorithm (left) and visible mass $m_\text{vis}$ of the ditau system (right) for the \htt (black line histogram) and \ztt (yellow filled histogram) events in the \mt final state \cite{Bianchini:2014vza}.}
    \label{fig:svfit}
\end{figure}

Since there are undetectable neutrino(s) present in the $\tau$ decays, the full reconstruction of the $\tau\tau$ system is not possible by means of \tauh reconstruction algorithms only. A dedicated SVFit algorithm \cite{Bianchini:2014vza} is used to recover for this loss of information. It combines the missing transverse momentum vector \met and its uncertainty matrix with the reconstructed four-vectors of two tau leptons and uses a simplified matrix-element approach to reconstruct the invariant mass ($m_{\tau\tau}$) of the ditau system. This variable provides better separation between \htt and $Z/\gamma^*\to\tau\tau$ events, compared to the visible mass ($m_\text{vis}$) of the ditau system (Fig. \ref{fig:svfit}).    

The analysis in this work, as it will be shown in Sec. \ref{sec:phicp}, heavily relies on identification of \tauh decay modes. Several changes have been already introduced with the DeepTau algorithm to improve their purity and efficiency at the stage of the HPS algorithm. However, the migration from, for example, $\text{DM}=11$ ($\h\text{h}^\mp\h\text{h}^0$) to $\text{DM}=10$ ($\h\text{h}^\mp\h$) is still sizeable ($\sim25\%$) and leads to the contamination of the latter DM category, which in turn affects the analysis sensitivity. Furthermore, $\text{DM}=2$ ($\h\text{h}^0\text{h}^0$) is merged with $\text{DM}=1$ ($\h\text{h}^0)$ which does not allow for their separate analysis. 

To mitigate these limitations, two BDTs (referred to as MVA DM) are trained and applied on top of the HPS reconstructed \tauh candidates to predict their decay mode \cite{CMS-DP-2020-041}. One BDT is designed to identify decay modes with one charged prong and the number of $\pi^0$ $n(\pi^0) = \{0,1,2\}$ ($\text{DM}=0,1,2$), while the other targets \tauh candidates with three charged prongs and $n(\pi^0) = \{0,1\}$ ($\text{DM}=10,11$). The input variables to the BDT describe the kinematics, invariant mass properties and angular information of the constituents of an HPS reconstructed \tauh candidate. \htt events in the \mt and \tata final states with the H produced via vector-boson or gluon-gluon fusion are used for the training. Overall, the BDTs provide the identification of \tauh candidates with $\text{DM}=2$ and consistently improve the purity of DM selection by up to $25\%$ without significant reduction in efficiency (Fig. \ref{fig:mva_dm}).  

\begin{figure}[t!]
    \centering
    \includegraphics[width=0.45\textwidth]{Figures/CP_etau/mva_purity.png}
    \includegraphics[width=0.45\textwidth]{Figures/CP_etau/mva_eff.png}
    \caption{Comparison of purity and efficiency of the \tauh decay mode identification between the HPS (orange bars) and MVA DM algorithms (blue bars) \cite{CMS-DP-2020-041}.}
    \label{fig:mva_dm}
\end{figure}



\subsection{Jets and missing transverse energy}\label{sec:jets}
An anti-$k_\text{T}$ algorithm \cite{Cacciari:2008gp} with the distance parameter $R=0.4$ as implemented in the FastJet package \cite{Cacciari:2011ma} is used for the reconstruction of jets. Effectively, it is proposed as an extension of the $k_\text{T}$ \cite{Ellis:1993tq} and Cambridge/Aachen \cite{Wobisch:1998wt} algorithms with redefining the distance measure as:

\begin{align}
    &d_{ij} = \min(k_{\text{T},i}^{2p},k_{\text{T},j}^{2p})\dfrac{\Delta^2_{ij}}{R^2}\\
    &d_{iB} = k_{\text{T},i}^{2p},
\end{align}

where $d_{ij}$ is the distance between entities $i$ and $j$ (either particles or \enquote{pseudojets}), $d_{iB}$ is the distance between the entity $i$ and the beam,  $\Delta^2_{ij} = (y_i-y_j)^2 + (\phi_i - \phi_j)^2$ with $k_{\text{T},i}, y_i, \phi_i$ being the transverse momentum, rapidity and azimuthal angle of the entity $i$, respectively. The parameter $p$ balances between the energy and geometrical scales. For $p=0$ one obtains an inclusive Cambridge/Aachen algorithm, while the case $p=1$ corresponds to the $k_\text{T}$ algorithm.

An inclusive anti-$k_\text{T}$ algorithm corresponding to the case $p=-1$ starts by combinatorically computing the distances $d_{ij}$ and $d_{iB}$ between input PF particles. If $d_{ij}$ is the smallest out of two, the particles/entities are merged together into a single entity (so-called \enquote{pseudojet}). Otherwise, the particle/entity $i$ is removed from the list and called a jet. Then the distances are recalculated until there are no entities left.

In order to correct for the impact of pileup interactions on the jet observables, a charge hadron subtraction (CHS) technique is used \cite{CMS-PAS-JME-14-001}. It identifies the PF candidates which originate from pileup vertices and removes them from the collection used to cluster jets. Residual jet energy corrections are applied to correct for differences between data and simulation \cite{CMS:2016lmd}. A large amount of noise in the ECAL endcaps during the 2017 data taking period caused a disagreement between the data and simulation. Therefore, jets with $\pt < 50~\text{GeV}$ and $2.65 < |\eta| < 3.10$ are removed from the analysis of the 2017 data set. 

Jets containing b-quarks are identified with a DeepCSV algorithm \cite{CMS:2017wtu}. The Medium working point is used which corresponds to approximately $70\%$ identification efficiency of b-jets with the misidentification of jets from light quarks/gluons at the level of $1\%$.

The missing transverse energy (MET) \met is reconstructed as the momentum imbalance in the transverse plane \cite{CMS:2019ctu}. \met is calculated as a negative vectorial sum of the reconstructed PF candidates in the event with the jet energy corrections being taken into account. Pileup effects are mitigated with a pileup per particle identification (PUPPI) algorithm \cite{Bertolini:2014bba} which assigns a weight to each PF candidate that indicates the likelihood of the candidate to originate from a pileup interaction. These weights are further used to rescale the four-momentum of the PF candidates, which showed to improve both jet and MET observables comparing to the CHS method.

\subsection{Primary vertex}\label{sec:pv}

\begin{figure}[t!]
    \centering
    \includegraphics[width=0.9\textwidth]{Figures/CP_etau/pv.png}
    \caption{Difference between the generator-level and reconstructed primary vertex position for the $x$ (left), $y$ (middle), and $z$ (right) coordinates. Blue (orange) histogram corresponds to the nominal (refitted beamspot-corrected) PV reconstruction as described in step 1 (2) in Sec. \ref{sec:pv}}
    \label{fig:pv}
\end{figure}
 Vertex corresponding to a primary $pp$ interaction is reconstructed in two steps:
\begin{enumerate}
    \item An initial collection of primary vertex (PV) candidates is obtained by clustering tracks with a deterministic annealing algorithm \cite{726788}. A vertex having the largest value of $\sum \pt^2$ of the physics objects in the event (jets reconstructed from the tracks assigned to a candidate vertex and MET) is selected as a primary vertex.
    \item A refitting procedure with an adaptive vertex fitter \cite{Fruhwirth:2007hz} is performed to improve the PV position resolution. Tracks originating from $\tau$ decay are removed from the fit in order to remove  the bias arising from the displacement of the $\tau$ decay vertex. An additional constraint to the LHC beam spot -- 3-D region where LHC beams collide in the CMS detector -- is added.
\end{enumerate}

The position and covariance matrix of the beam spot are precisely measured as an average over multiple collision events \cite{CMS:2014pgm}. Therefore, using the beam spot information as an initial estimate of the PV position and uncertainty in the fit instead of a default fit configuration improves the fitting convergence and the PV position resolution in the transverse plane by a factor of 3 (Fig. \ref{fig:pv}). 

\subsection{Impact parameter}\label{sec:ip}

In order to perform CP analysis in some of the $\tau\tau$ final states reconstruction of the impact parameter (IP) -- a vector from the PV to the point of the closest approach of a charged particle track to PV -- is needed for the charged prongs originating from $\tau$ decays  (Fig. \ref{fig:ip}). To improve the IP resolution, a dedicated approach is developed. Contrary to another method using a tangent track extrapolation (Fig. \ref{fig:ip}), it parametrises the particle trajectory in the magnetic field as a helix $\vec{x}(t)$ and minimizes the distance between the trajectory and the primary vertex $\vv{d}(t) = |\vec{x}(t) - \vv{PV}|$. The resulting vector obtained after the minimisation $\text{IP} = \vec{x}(t_{\text{min}}) - \vv{PV}$ is used as an impact parameter vector. It should be noted, that the constructed IP vector is used only for the \phicp and IP significance (described below) computation. The selection requirements mentioned in Sec. \ref{sec:reco_e}, \ref{sec:reco_mu}, and \ref{sec:reco_tau} are applied on the impact parameter vectors computed by the minimisation in the transverse plane in contrast to the 3D minimisation described in this section.

\begin{figure}[t!]
    \centering
    \includegraphics[width=0.6\textwidth]{Figures/CP_etau/ip.png}
    \caption{Illustration of the impact parameter vector $\boldsymbol{n}_-$ reconstruction via a tangent method for a decay of a tau lepton with momentum $\text{\textbf{k}}_-$ to a one charged prong with momentum $\text{\textbf{p}}_-$ in a laboratory frame \cite{Berge:2008dr}. The impact parameter vector is obtained by extrapolating $\text{\textbf{p}}_-$ in the direction of the primary vertex (PV) from the first reference point on the track (the tracker hit nearest to PV, on the illustration corresponding to the intersection of the dashed and $\text{\textbf{p}}_-$ lines). A vector pointing from PV to the point of the closest approach on the extrapolated tangent is an impact parameter vector.}
    \label{fig:ip}
\end{figure}

Furthermore, the minimisation procedure allows for a propagation of track parameter uncertainties to the impact parameter vector, therefore enabling the construction of an impact parameter significance variable:
\begin{equation}
S_{\text{IP}} = \dfrac{|\text{IP}|}{\sigma(\text{IP})},
\end{equation}

where $\sigma(\text{IP}) = \dfrac{\vv{\text{IP}}^{\text{T}}}{|\vv{\text{IP}}|}\boldsymbol{\Sigma}_{\vv{\text{IP}}}\dfrac{\vv{\text{IP}}}{|\vv{\text{IP}}|}$ and $\boldsymbol{\Sigma}_{\vv{\text{IP}}}$ is the covariance matrix for the impact parameter vector derived with the error propagation. $S_{\text{IP}}$ variable is further used in the event categorisation step described in Sec. \ref{sec:categ}.

\subsection{\phicp observable}\label{sec:phicp}
\subsubsection{Introduction}
As described in Sec. \ref{sec:cp-intro} and specifically in Eq. \ref{eq:h_width_spin}, the CP nature of the Higgs boson coupling with tau leptons can be accessed through the spin correlations of the tau leptons resulting from its decay. However, it is not straight-forward \textit{a priori} how to analyse this correlations experimentally. Furthermore, the situation is also complicated by the necessity to reconstruct the Higgs rest frame, which is not available in $pp$ collision at LHC. 

The following approach is proposed by Berge et al. \cite{Berge:2011ij, Berge:2014sra, Berge:2014wta}. Firstly, considering the general form of the tau lepton decay via a charged prong $\tau^\pm \to a^\pm + X$ with $a^\pm \in \{e^\pm, \mu^\pm, \pi^\pm, \rho^\pm, a_1^{L,T,\pm}\}$, one obtains the partial decay width of the tau lepton:
\begin{equation}\label{eq:tau_width}
    \Gamma_ad\Gamma(\tau^\pm(\boldsymbol{s}^\pm) \to a^\pm(q^\pm)+X) = n(E_\pm)\cdot[1 \pm b(E_\pm) \boldsymbol{s}^\pm \cdot \boldsymbol{q}^\pm]\cdot dE_\pm\dfrac{d\Omega\pm}{4\pi},
\end{equation}

where $\boldsymbol{s}^\pm$ is a normalised spin vector of the tau lepton in its rest frame, $E_\pm$ and $\boldsymbol{q}^\pm$ are the energy and the direction of flight of $a^\pm$ in the tau rest frame. $n(E_\pm)$ and $b(E_\pm)$ are referred to as spectral functions \cite{Berge:2011ij}. 

\begin{figure}[t!]
    \centering
    \includegraphics[width=0.6\textwidth]{Figures/CP_etau/phicp_vis.png}
    \caption{Illustration of the \htt in its rest frame where each $\tau$ decays into a single charged pion \cite{CMS:2021sdq}. The \phicp angle between the tau lepton decay planes is shown as a red arrow.}
    \label{fig:phicp_vis}
\end{figure}

Using Eq. \ref{eq:l_y_angle} and \ref{eq:tau_width} one obtains for the partial decay width of \htt:
\begin{equation}\label{eq:master}
    \dfrac{d\Gamma}{d\phicp}(\htt) \sim 1 - \dfrac{\pi^2}{16}b(E^+)b(E^-)\cos(\phicp - 2\mixa),
\end{equation}

where a \phicp observable is introduced as the angle between the tau lepton decay planes in the Higgs rest frame (Fig. \ref{fig:phicp_vis}). However, since the latter cannot be reconstructed in $pp$ collisions, a zero-momentum frame (ZMF) using the charged decay products of the tau leptons is used in this work as an approximation of the Higgs rest frame. This might potentially reduce the overall sensitivity of the analysis, therefore hinting towards further studies of the ditau system reconstruction in $pp$ collisions.

\begin{figure}[t!]
    \centering
    \includegraphics[width=0.95\textwidth]{Figures/CP_etau/sfunc.png}
    \caption{Spectral functions $n(E_\pi)$ and $b(E_\pi)$ for the charged pion in $\tau^- \to \rho^- \nu_\tau \to \pi^-\pi^0\nu_\tau$ (left) and $\tau^\pm \to a_1^- \nu_\tau \to \pi^-\pi^0\pi^0\nu_\tau$ (right) decays as a function of the charged pion energy ($E_\pi$) in the tau rest frame \cite{Berge:2011ij}. The values of $n(E_\pi)$ and $b(E_\pi)$ are given in units of GeV$^{-1}$.}
    \label{fig:sfunc}
\end{figure}

One can contrast Eq. \ref{eq:master} with Eq. \ref{eq:h_width_spin} and observe that \phicp angle effectively resembles the angle between the transverse spin vectors of the tau leptons. The $\tau$ decay product topology can therefore be viewed as having a spin analysing power which allows to access the spin information experimentally. However, this power is dependent on the $\tau$ decay mode and on the properties of the charged prong as encoded with the spectral functions. The functions show complex behaviour (Fig. \ref{fig:sfunc}) and for some scenarios can change their sign therefore affecting the separation between pure scalar and pseudoscalar hypotheses. No dedicated optimisation of CP sensitivity is carried out in this work as the analysis is largely limited by the available statistics. This strongly affects the room for optimisation of the event selection with respect to the spectral functions as it will further reduce the amount of signal candidates. However, as more data will be available in the future, such optimisation can be carried out in the context of, for example, the differential measurement of CP \htt properties.

\begin{figure}[t!]
    \centering
    \includegraphics[width=0.6\textwidth]{Figures/CP_etau/phicp.png}
    \caption{The distribution of \phicp angle in the Higgs rest frame at the generator level for the \htt events where both tau leptons decay into a charged pion and a neutrino \cite{CMS:2021sdq}. The hypotheses of a scalar (solid red), pseudoscalar (dashed blue), and a CP mixture with $\mixa=45^\circ$ (dash-do-dot green) Higgs boson as well as a Z vector boson (dash-dot black) are shown.}
    \label{fig:phicp}
\end{figure}

Similarly to Eq. \ref{eq:h_width_spin}, a CP mixing angle \mixa enters in Eq. \ref{eq:master} as a phase shift of the cosine distribution. Therefore, given enough sensitivity one would be able to gauge the CP nature of the $\text{H}\tau\tau$ coupling by the shift of the modulation from the expected SM (CP-even) scenario (Fig. \ref{fig:phicp}). It should also be noted that for the $Z/\gamma^* \to \tau\tau$ process, which constitute one of the major background in this work, the distribution of \phicp observable is uniform at the generator level. 

\begin{figure}[t!]
    \centering
    \includegraphics[width=0.45\textwidth]{Figures/CP_etau/phicp_dy.png}
    \includegraphics[width=0.45\textwidth]{Figures/CP_etau/phicp_ggh.png}
    \caption{Unnormalised \phicp distributions for the simulated Drell-Yan (left) and \htt (right) events on the reconstructed level for four final states analysed in this work. Dashed lines on the left figure show the mean of the corresponding histogram counts. On the right figure, solid (dashed) lines represent the distribution of the scalar (pseudoscalar) \htt hypothesis. Histogram counts are in auxiliary units.}
    \label{fig:phicp_e}
\end{figure}

However, at the detector level the limited track and PV resolution leads to the distortion of the \phicp distribution both for the signal and background processes \cite{Berge:2014sra}. For example, for the \et analysed in this work, for the $e\pi$ final state (one tau decays into electron and neutrinos, the other into a charged pion and neutrinos) it is visible on Fig. \ref{fig:phicp_e} (left) that the distribution of the simulated Drell-Yan events is not uniform and peaks towards $0$ and $2\pi$ values of \phicp. This is due to the the PV misreconstruction effects that on average pull the reconstructed IP vectors towards smaller values, which consequently translates to \phicp values. This effect is pronounced only for the final states where IP vectors are used for the reconstruction of \phicp for both tau leptons, as described below in this section. Despite these effects destruct the uniformity of the Drell-Yan background events, some symmetries can still be used in the construction of the templates for the statistical inference, as described in Sec. \ref{sec:stat}. For the \htt events, the modulations are clearly visible at the reconstruction level for all the final states being considered, and the pure CP even and CP odd hypotheses are separable. 

\subsubsection{Methods}

By definition, \phicp is the angle between the tau lepton decay planes in the Higgs rest frame. Intuitively, this can be reconstructed for the tau leptons both decaying into at least two reconstructable objects by using the momenta vectors of the latter. However, if one of the tau leptons decays into a single charged prong and neutrino, it is no longer possible to define in this way its decay plane because of the neutrino escaping detection. This is particularly the case for the \et final state analysed in this work, where one tau lepton decays into an electron and a neutrino.

Since there are two decay planes involved in the \phicp computation, the problem factorises into the problem of reconstructing separately a decay plane for each of the tau leptons followed by computing the angle between them. Furthermore, there are generally two four-vectors in the laboratory frame needed to construct the plane in the ZMF. To unify the notation across various decay modes, these are further referred to as $\lambda^\pm$ ($\lambda^{ZMF^\pm}$) and $q^\pm$ ($q^{ZMF^\pm}$) in the laboratory (zero-momentum) frame, where $\pm$ refers to the charge of the tau lepton. Vectors in the ZMF are obtained by the Lorentz boost from the laboratory frame. Depending on the decay mode, the four-vectors are constructed using various approaches, as described further.

A transverse component of $\lambda^{ZMF^\pm}$ with respect to $q^{ZMF^\pm}$ is derived and the corresponding normalised unit vector is denoted as $\hat{\lambda}_\perp^{ZMF^\pm}$. Then, variables $\phi^{ZMF}$ and $O^{ZMF}$ are defined as:
\begin{align}
    &\phi^{ZMF} = \arccos(\hat{\lambda}_\perp^{ZMF^+} \cdot \hat{\lambda}_\perp^{ZMF^-}),\\
    &O^{ZMF} = \hat{q}_\perp^{ZMF^-} \cdot (\hat{\lambda}_\perp^{ZMF^+} \cross \hat{\lambda}_\perp^{ZMF^-}).
\end{align}

Finally, one obtains $\phicp \in [0^\circ,360^\circ]$ angle as:
\begin{equation}\label{eq:phicp}
    \phicp = 
    \begin{cases}
    \phi^{ZMF} & \text{if} ~O^{ZMF} \geq 0 \\
    360^\circ - \phi^{ZMF} & \text{if} ~O^{ZMF} < 0
    \end{cases}
\end{equation}

For the decays into one charged prong $\tau^- \to e^- \nu_\tau$ and $\tau^- \to \pi^- \nu_\tau$, an \textbf{impact parameter (IP) method} is used to construct the spanning four-vectors (Fig. \ref{fig:planes}, left). In these decay modes there is only one momentum vector available as $q^\pm$. Therefore, the impact parameter vector (Sec. \ref{sec:ip}) of a charged prong is used as $\lambda^\pm$ to be able to define a decay plane. It should be noted that in this case the resulting decay plane is not a \enquote{genuine} one, but rather a necessary approximation.  

For the decays $\tau^- \to \rho^- \nu_\tau$, $\tau^- \to a_1^-(\text{1pr}) \nu_\tau \to \pi^\mp \pi^0  \pi^0 \nu_\tau$, and $\tau^- \to a_1^-(\text{3pr})\nu_\tau \to \pi^\mp \pi^\pm \pi^\mp \nu_\tau$, a \textbf{neutral-pion (NP) method} is used (Fig. \ref{fig:planes}, right). For the $\tau \to \rho$ case, a four vector of the neutral pion resulting from the $\rho$ meson decays is taken as $\lambda^\pm$. For this four-vector the energy is set to the sum of energies of electrons and photons collected by the HPS algorithm, the momentum direction is taken as the direction of the leading electron/photon, and the mass is set to the $\pi^0$ mass. 

For the $\tau \to a_1(\text{1pr})$ case, all the electromagnetic constituents from the $a_1^-(\text{1pr})$ decay are combined together and treated analogously to the $\tau \to \rho$ case. Additionally, in order to avoid destructive interference between longitudinal and transverse polarised components of the intermediate mesons, in all the decay modes where the NP method is used the components are separated by the following variable:
\begin{equation}
    y^{\tau^\pm} = \dfrac{E_{\pi^\pm} - E_{\pi^0}}{E_{\pi^\pm} + E_{\pi^0}}, ~y^\tau = y^{\tau^+}y^{\tau^-},
\end{equation}

where $E_\pi$ is the energy of the pion in the laboratory frame. If $y^\tau < 0$, \phicp is recomputed with a shift as $\phicp \to \phicp - 360^\circ$.

For the $\tau \to a_1(\text{3pr})$ case, a pair of oppositely charged pions with the mass closest to the $\rho^0$ meson mass is chosen. Out of these two pions, the one with the charge of the tau lepton is used for the definition of the ZMF and the $q^\pm$ vector. The pion with the charge opposite to the one of the tau lepton is treated like a neutral pion. Then, the neutral-pion is applied as described for the $\tau \to \rho$ case. 

\begin{figure}[t!]
    \centering
    \includegraphics[width=0.3\textwidth]{Figures/CP_etau/plane_ip.png}
    \includegraphics[width=0.3\textwidth]{Figures/CP_etau/plane_np.png}
    \caption{Illustration of the \phicp angle construction for two decay topologies \cite{CMS:2021sdq}. Left: decay planes for both tau leptons are defined with the impact parameter method as spanned by the momentum and IP vector of the corresponding charged prong. Right: for one tau lepton the impact parameter method is used and the neutral-pion method for the other with the plane spanned by the momenta of the charged and the neutral pions. Vectors are shown in the zero-momentum frame constructed from the momenta of the charged constituents in the \htt decay.}
    \label{fig:planes}
\end{figure}

To summarise, the final states to be analysed with the corresponding methods to reconstruct decay planes are:
\begin{itemize}
    \item $e\pi$ ($\tau\tau \to e^\pm \pi^\mp + 3\nu$) $\longrightarrow$ IP + IP $\longrightarrow$ Fig. \ref{fig:planes}, left,
    \item $e\rho$ ($\tau\tau \to e^\pm \rho^\mp + 3\nu$) $\longrightarrow$ IP + NP $\longrightarrow$ Fig. \ref{fig:planes}, right,
    \item $e a_1(\text{1pr, 3pr})$ ($\tau\tau \to e^\pm a_1^\mp(\text{1pr, 3pr}) + 3\nu$) $\longrightarrow$ IP + NP $\longrightarrow$ Fig. \ref{fig:planes}, right.
\end{itemize}

\section{Event selection}\label{sec:selection}
The following procedure to select events for the analysis of the \et final state, where one tau lepton decays into an electron plus neutrinos ($\taue$) and the other into hadrons plus neutrino ($\tauh$), is followed:

\begin{enumerate}
    \item Events are selected online by the CMS trigger system (Sec. \ref{}). Either a cross trigger ($e \, \& \, \tauh$) or a single electron trigger ($e$) should be fired depending on the data taking year (online \pt thresholds for the corresponding objects are shown in brackets in GeV):
    \begin{itemize}
        \item 2016: $e$(25),
        \item 2017: $e$(27) OR $e(24) \, \& \, \tauh(30)$,
        \item 2018: $e$(32) OR $e(24) \, \& \, \tauh(30)$.
    \end{itemize}
    
    \item For each event, pairs of oppositely charged electron and hadronically decaying tau lepton reconstructed offline (Sec. \ref{sec:reco_e} and Sec. \ref{sec:reco_tau}) are selected with the requirement of being sufficiently separated (cone distance $\Delta R > 0.5$). These offline objects are required to match the corresponding online objects within the cone distance $\Delta R < 0.5$. Offline electron (\tauh) objects should have \pt at least 1(5) GeV higher when the online \pt thresholds for the corresponding trigger legs. 
    
    \item Events with an additional loosely identified electron or muon as well as a pair of electrons are vetoed.
    
    \item Events containing a single jet (jets) with $\pt > 25$ GeV and $|\eta| < 2.4$ passing the Medium (Loose) working point of the DeepCSV classifier (Sec. \ref{sec:jets}) are vetoed. 
    
    \item The reconstructed electron candidate is selected with $\pt > 25$, $|\eta| < 2.1$ as well as IP, identification, and isolation requirements described in Sec. \ref{sec:reco_e}. 
    
    \item The reconstructed \tauh candidate is selected with $\pt > 20$, $|\eta| < 2.3$ ($|\eta| < 2.1$ for the cross trigger) as well as IP and identification requirements described in Sec. \ref{sec:reco_tau}.
    
    \item The transverse mass of the electron candidate and the missing transverse energy \met is defined as:
    \begin{equation}
        m_\text{T} \equiv \sqrt{2p_\text{T}^e p_\text{T}^\text{miss}[1-\cos(\Delta\phi)]},
    \end{equation}
    with $\Delta\phi$ denoting the azimuthal angle between the vector of electron transverse momentum $\vv{p}_\text{T}^e$ and \met. The requirement $m_\text{T} < 50 ~\text{GeV}$ is applied in order to reject the background from the W+jets process.
    
    \item In case there are multiple \et candidates in the event, the pairs are ranked firstly with the highest priority given to the pairs with the most isolated electron, then with the electron with the highest \pt, then with the \tauh candidate with the highest DeepTau against jet score, then with the \tauh candidate with the highest \pt.
\end{enumerate}


\section{Background estimation}\label{sec:bkgr}
The main background sources in the \et final state can be roughly classified as those involving genuine tau leptons, jets faking \tauh ($\text{jet}\to \tauh$), and prompt/non-prompt leptons faking \tauh ($l\to \tauh$). In terms of physical processes, the expected contributions are from the Drell-Yan, QCD multijet, top
quark-antiquark pair production (bar), single top quark production (ST), W+jets, and diboson production processes. In this work, the backgrounds are largely modelled with data-driven methods: a $\tau$-embedding technique \cite{CMS:2019pkt} is used to model background with two genuine tau leptons, and a \enquote{fake factor} (\ff) method \cite{CMS:2018lkr} is used to model $\text{jet}\to \tauh$ background. These two methods together account for approximately $90\%$ of all background processes. Other minor background processes are modelled from the simulation where events with a pair of genuine tau leptons or with a jet faking \tauh are removed to avoid double-counting. 

\subsection{$\tau$-embedding method}\label{sec:emb}
The most challenging part in the simulation of the $Z/\gamma^* \to ll$ process is to model jets originating from partons emitted in the initial state radiation. This translates, after hadronisation, in the multiplicity of jets in the final state and the corresponding hadronic activity in the detector. Its modelling would require resource-demanding simulation of samples at NLO and higher orders, which in practise still does not guarantee adequate description of event observables in data. Therefore, finding a way to model the $Z$+jets process without relying only on simulation would be beneficial. 

\begin{figure}[t!]
    \centering
    \includegraphics[width=0.75\textwidth]{Figures/CP_etau/embedding.png}
    \caption{Illustration of the $\tau$-embedding steps as described in Sec. \ref{sec:emb} \cite{CMS:2019pkt}.}
    \label{fig:emb}
\end{figure}

A $\tau$-embedding method is designed with that purpose and builds up on an idea of using the lepton universality to extract and transfer the detector activity from data to simulation across leptonic final states in $Z/\gamma^* \to ll$. It proceeds with the following steps (Fig. \ref{fig:emb}):

\begin{enumerate}
    \item A sample of $Z\to \mu^+\mu^-$ events is recorded in data with a dedicated dimuon trigger. This sample of muon pairs is of the highest purity thanks to the excellent muon reconstruction in CMS. However, an excessively tight selection aimed at high purity might introduce a bias to the selected events as having, for example, little hadronic activity in the detector. Therefore, only a loose kinematic selection without any isolation requirement is applied. The final dimuon sample has the $Z\to \mu^+\mu^-$ purity of $99.11\%$ for $m_{\mu\mu} > 70$ GeV. Remaining contributions come from \ttbar ($0.55\%$), diboson and single top ($0.17\%$), QCD ($0.10\%$), \ztt ($0.05\%$), and W+jets ($0.02\%$) processes. 
    
    \item All traces in the detector which are associated with the muons are removed from the event. This includes hits in the tracking system and muon chambers, plus energy deposits in the calorimeter which are compatible with the fitted global-muon track. 
    
    \item A pair of tau leptons with the kinematic properties of the two muons in data is simulated with PYTHIA and passed through the empty detector environment (no other particles, no pileup). The tau leptons are forced to decay into a predefined \et final state with a branching fraction $100\%$. However, electrons and muons can also in principle be simulated and injected for validation purposes. 
    
    \item A hybrid event is created as an overlay of the event with the removed muon traces and the event with the simulated pair of tau leptons. The combination is performed at the reconstruction level of physics objects (tracks, $e/\mu$, calorimeter clusters). The resulting sample of hybrid events can be used in the analysis to model the backgrounds with two genuine $\tau$ leptons.

\end{enumerate}
 
 Overall, the $\tau$-embedding method provides a fully data-driven description of detector activity in $Z/\gamma^* \to ll$ events. This saves both computational resources for simulation of the highly dense pileup environment and provides an excellent description of jet-related physical observables. Furthermore, only systematical uncertainties related to the simulated pair of tau leptons and the selection efficiency of $\mu^+\mu^-$ pairs in data have to be introduced, therefore improving the sensitivity of the analysis. 

\subsection{\ff method}\label{sec:ff}
The $\text{jet} \to \tauh$ background constitutes another important background in the analysis. It is driven by the presence of QCD jets which fake rate to \tauh is hard to model and requires computationally intensive simulation to reach the desired level of statistics. Therefore, a data-driven approach is also desired in this case.

\begin{figure}[t!]
    \centering
    \includegraphics[width=0.7\textwidth]{Figures/CP_etau/ff.png}
    \caption{Illustration of the \ff method steps as described in Sec. \ref{sec:ff} \cite{CMS:2018rmh}.}
    \label{fig:ff}
\end{figure}

The fake factor method is proposed to tackle the modelling of this background. It proceeds with the following steps (Fig. \ref{fig:ff}):
\begin{enumerate}
    \item A signal region (SR) is defined in data as described in Sec. \ref{sec:selection}. This includes the nominal Medium working point of the DeepTau ID discriminator against jets applied to the \tauh candidate (Sec. \ref{sec:selection}).
    
    \item An application region is defined in data with all the selection requirement as in SR, except that the WP selection against jets for the \tauh candidate is inverted: the candidate is required to pass the loosest VVVLoose WP (nominal \tauh efficiency $98\%$) and fail the Medium WP. Events in the AR will serve as an estimate of the \jtt background in SR once assigned a fake factor weight \ff on the event-by-event basis. The latter is derived as follows:
    
    \begin{align}
        &\ff = \sum_{i}w_i \cdot \ff^i,\\ \label{eq:ff}
        &w_i = \dfrac{N^i_\text{AR}}{\sum_j N^j_\text{AR}},\\ \label{eq:ff_fr}
        & \ff^i = \dfrac{\text{N}_{\text{DR}}^i(\text{Medium})}{\text{N}_{\text{DR}}^i(\text{VVVLoose} ~\& ~!\text{Medium})}.
    \end{align}

    Here the final FF weight is obtained as a weighted sum across the \jtt background processes $i \in \{\text{QCD, W+jets, } \ttbar\}$. The weights $w_i$ correspond to the fraction of the background process $i$ in all \jtt events in AR. $\ff^i$ for each process is obtained as the number of events passing the Medium WP against jets divided by the number of events passing the VVVLoose WP and failing the Medium one in a so-called determination region (DR). Both $w_i$ and $\ff^i$ are parametrised functions of several variables as described below. 
    
    \item The determination region (DR) is constructed depending on the background process $i$. For \ttbar, it is not straight-forward to find sufficiently pure region in data, therefore the fake factor is taken from the simulation of the \ttbar process. For W+jets, DR region is defined in data with inverting the transverse mass requirement (Sec. \ref{sec:selection}) $m_\text{T} > 70 ~\text{GeV}$ with all the other selection remaining the same as in SR. For QCD, DR region is defined in data with inverting the opposite-sign requirement, i.e. electron and \tauh candidates are required to have the same charge. Additionally, $I^e_{\text{rel}} > 0.05$ requirement is applied to remove events with genuine tau leptons. All the other selection criteria are the same as in SR.
    
    \item For QCD and W+jets, $\ff^i$ are measured in bins of $N_\text{jets}$, MVA DM (Sec. \ref{sec:reco_tau}), where the MVA DM equal to 0 is further split into two bins on the IP significance (Sec. \ref{sec:ip}) $S_{\text{IP}} < 1.5, S_{\text{IP}} \geq 1.5$. $\ff^i$ are measured separately for events passing the single $e$ and $e~\&~\tauh$ cross triggers. For \ttbar, $\ff^i$ are measured only in bins of MVA DM and $S_{\text{IP}}$. The contribution of other processes to each of DRs is subtracted using simulation. For each bin, $\ff^i$ is parametrised as a function of \tauh \pt as obtained from the polynomial fit to the $\ff^i$ distribution. 
    
    \item Fractions of the processes in AR $w_i$ are parametrised in bins of the dedicated BDT score. This is motivated by the fact that it is difficult to find a small set of variables providing a good differentiation between the \jtt processes. Therefore, a summary statistics is constructed as a BDT output which is trained to distinguish between the three background processes QCD, W+jets, and \ttbar using kinematic information about the ditau system, $p_\text{T}^\text{miss}$, and $N_\text{jets}$  as input variables. Fractions are then computed according to Eq. \ref{eq:ff_fr} in bins of W+jets and QCD BDT scores for W+jets, \ttbar (both taken from simulation) and QCD (taken from data with all other processes subtracted with simulation) processes.
    
    \item Fake factor weights are computed according to Eq. \ref{eq:ff} in AR, where the contribution of processes with genuine tau leptons and $l\to\tauh$ fakes is subtracted using simulated events.
    
    \item Corrections are applied to account for discrepancies in the closure tests when $\ff^i$ are applied to events in the corresponding DR. Differences between DR and AR resulting in a different \jtt rates between there two regions are also accounted for in the corresponding corrections. 
\end{enumerate}

\subsection{Corrections}\label{sec:corr}
While most of the background is estimated from data, it still relies to a certain extent on simulation which requires dedicated corrections to be applied in order to refine its modelling of data. 

The corrections applied to the embedded samples are:
\begin{itemize}
    \item Electron tracking/ID/isolation/trigger scale factors (SFs),
    \item Electron energy scale (ES) and resolution smearing corrections,
    \item Hadronic tau ID/trigger SFs,
    \item Hadronic tau ES corrections.
\end{itemize}

The scale factors are derived with a tag-and-probe method \cite{CMS:2010svw} generally as a function of e/\tauh \pt, $\eta$, and MVA DM using $Z/\gamma^*\to ll$ events. These are aimed to account for the mismatch in the corresponding ID/trigger/isolation selection efficiencies between data and simulation. The energy scale corrections are derived by varying the lepton energy scale in simulation and performing the maximum likelihood fit to physical observables in data (e.g. \tauh and $\mu\tauh$ invariant mass ($m_\text{vis}$) for \tauh case) in order to find the most optimal value. The value of the energy scale shift corresponding to the minimum of the negative log-likelihood is taken as the correction factor.

It should be noted that while tau leptons in the embedded samples are still simulated, the corresponding corrections are derived specifically for the embedded samples and differ from the same corrections applied to the simulated samples.

The corrections applied to the simulated samples include those applied to the embedded samples, plus the following ones:
\begin{itemize}
    \item Pileup reweighting. The distribution of PU interactions in simulated samples is reweighted to match the one observed in data.
    \item $e\to\tauh$ fake rate and ES corrections. The corrections are obtained for the DeepTau discriminator against electrons similarly to e.g. \tauh ID SFs using the tag-and-probe method with $Z\to ee$ events. 
    \item MET recoil corrections. The corrections are applied to Drell-Yan, W+jets, and Higgs simulated samples and aim to correct for the mismodelling of \met. The corrections are derived with $Z\to\mu\mu$ events on the variable defined as the vectorial difference between measured \met and the sum of the transverse momenta of neutrinos from a boson decay. 
    \item b-tagging efficiency corrections. Since veto is applied on events where the jets pass certain working points of the DeepCSV classifier (Sec. \ref{sec:selection}), one needs to ensure that the mismodelling of the DeepCSV score is not propagated to the analysis. A so-called \enquote{promote-demote} technique is used which randomly assign/remove a given jet to/from b-tagged category in order to match WP selection efficiency in simulation to the one measured in data.
    \item $Z$ mass and \pt reweighting. The corresponding spectra are corrected in simulation to better match those obtained in data for $Z\to \mu\mu$ events. 
    \item Top quark \pt reweighting. The distribution of the top quark transverse momentum is reweighted in the NLO simulated samples to match the distribution obtained from NNLO.
    \item Prefiring. During the 2016 and 2017 data taking periods it was observed that the L1 trigger system would sometimes \enquote{prefire}, i.e. record an event corresponding to the previous bunch crossing. The issue was related to the shift in ECAL pulses and therefore a corresponding weight is introduced to recover for this effect \cite{CMS:2020cmk}. 
\end{itemize}

Lastly, corrections to the impact parameter significance variable are applied both to embedded and simulated samples. As it is discussed in Sec. \ref{sec:categ}, a selection is applied based on this variable to further improve the sensitivity to the CP mixing angle. Therefore, it is important to ensure good modelling of data in this observable. A quantile mapping method is used to correct IP vector coordinates and covariance matrix based on their cumulative distributions in data and simulation using $Z\to ee$ ($Z\to \mu\mu$) events for electron (pion) legs.  

\begin{figure}[t!]
    \centering
    \includegraphics[width=0.6\textwidth]{Figures/CP_etau/mvis.png}
    \caption{Comparison of data with simulation for 2018 data-taking period for the visible mass of the \et system variable.}
    \label{fig:mvis}
\end{figure}

After estimating the backgrounds as described above and applying all the necessary corrections good agreement between data and simulation is observed for the key physical observables for all years (Fig. \ref{fig:mvis}, Fig. \ref{fig:nn_vars}, and Appendix \ref{app:control_plots}). 

\section{Event categorisation}\label{sec:categ}
After the selection of \et candidate pairs (Sec. \ref{sec:selection}), one has to perform their categorisation. Fundamentally, that means that for each candidate pair one wants to understand which physical process it originates from. Since the selection criteria do not yield the sample with only \htt events but the one which is contaminated by several background processes, one is interested in purifying this sample. This is hardly possible to perform manually due to an extremely large number of events to be categorised. Therefore, some automated refining procedure is needed.

One possible solution would be to follow a rule-based approach and define a set of criteria on custom variables constructed using expert knowledge. This is however a difficult task due to the multidimensionality of the problem, which moreover might not give an optimal result. Therefore, a Machine Learning (ML) approach is taken to classify events into predefined categories. These are defined as follows, together with the corresponding samples used for the composition of a training data set:

\begin{itemize}
    \item \textbf{Signal:} to target signal H events originating from the ggH and VBF production processes. Events for the training are taken from the corresponding simulated samples.
    \item \textbf{Genuine} $\mathbf{\tau}$: to target background events with two genuine tau leptons. These include $Z\to\tau\tau$, \ttbar, and diboson processes. Events for the training are taken from the corresponding simulated samples.
    \item \textbf{Fakes:} to target background events with jets or leptons faking \tauh. These include $Z\to ll$, $Z\to \text{jets}$, \ttbar, diboson, W+jets, and QCD processes. Events for the training are taken from the corresponding simulated samples except for QCD, which is estimated from data by inverting the opposite-sign requirement for the electron-\tauh pair. 
\end{itemize}

A neural network (NN) is trained to leverage the multidimensionality of the problem and to construct an optimal classifier in an automated way from the following high-level input variables: 
\begin{itemize}
    \item $p_\text{T}(e)$, electron \pt,
    \item $p_\text{T}(\tau)$, \tauh \pt, 
    \item $p_{\text{T}, \tau\tau} \equiv |\vec{p}_{\text{T}}(e) + \vec{p}_{\text{T}}(\tau) + \met|$, vectorially combined \pt of electron, tau lepton and missing transverse energy, 
    \item $m_\text{vis}$, visible invariant mass of the electron and \tauh decay products, 
    \item $m_{\tau\tau}$, invariant ditau mass obtained with the SVFit algorithm (Sec. \ref{sec:reco_tau}),
    \item $\text{E}_\text{T}^\text{miss}$, missing transverse energy obtained with the PUPPI algorithm (Sec. \ref{sec:jets}),
    \item $m_\text{T}$, transverse mass of the electron and MET (Sec. \ref{sec:selection}),
    \item $\text{N}_\text{jets}$, number of jets in the event,
    \item $p_\text{T}(\text{jet1})$, \pt of the jet with the highest \pt, referred to as the \enquote{leading} jet (in events with at least one jet),
    \item $p_\text{T}(\text{jet2})$, \pt of the jet with the second highest \pt, referred to as the \enquote{trailing} jet (in events wit hat least two jets),
    \item $p_{\text{T}, jj}$, combined \pt of the two leading jets, 
    \item $m_{jj}$, invariant mass of the two leading jets, 
    \item $\Delta\eta_{jj}$, difference in pseudorapidity between the two leading jets.
\end{itemize}

\begin{figure}[H]
    \centering
    \includegraphics[width=0.31\textwidth]{Figures/CP_etau/pt_e.png}
    \includegraphics[width=0.31\textwidth]{Figures/CP_etau/pt_tau.png}
    \includegraphics[width=0.31\textwidth]{Figures/CP_etau/pt_tautau.png}
    % \includegraphics[width=0.31\textwidth]{Figures/CP_etau/mvis.png}
    \includegraphics[width=0.31\textwidth]{Figures/CP_etau/mtautau.png}
    \includegraphics[width=0.31\textwidth]{Figures/CP_etau/puppimet.png}
    \includegraphics[width=0.31\textwidth]{Figures/CP_etau/mt.png}
    \includegraphics[width=0.31\textwidth]{Figures/CP_etau/njets.png}
    \includegraphics[width=0.31\textwidth]{Figures/CP_etau/pt_jet1.png}
    \includegraphics[width=0.31\textwidth]{Figures/CP_etau/pt_jet2.png}
    \includegraphics[width=0.31\textwidth]{Figures/CP_etau/pt_jj.png}
    \includegraphics[width=0.31\textwidth]{Figures/CP_etau/mjj.png}
    \includegraphics[width=0.31\textwidth]{Figures/CP_etau/deta_jj.png}
    \caption{Comparison of data with simulation for 2018 data-taking period for the variables used in the neural network training, as described in Sec. \ref{sec:categ}.}
    \label{fig:nn_vars}
\end{figure}

This set of variables is chosen as the ones which are known \textit{a priori} to provide discriminating power between the classes and to be well-modelled in data (Fig. \ref{fig:mvis}, Fig. \ref{fig:nn_vars}, and Appendix \ref{app:control_plots}). Each event is therefore represented as a vector of length 13, which defines the input to the model.

The model architecture consists of three consecutive blocks. Each block has the same structure and is constructed from a feed-forward layer with 100 nodes, followed by a batch normalisation layer, a ReLU activation function, and a dropout layer with probability $p=0.5$. The output layer is a feed-forward layer with three nodes normalised to sum up to 1 with a softmax function. It therefore defines the probability of an event to belong to either of the three classes, as defined above.

The batch size equals to 1000 and the training is performed using the TensorFlow library \cite{tensorflow2015-whitepaper} until convergence with an early stopping in case of no validation loss improvement for 20 consecutive iterations. Three separate trainings are done with the same architecture used for each of the data-taking years (2016, 2017, 2018). Each training in fact corresponds to the training of two models in a \enquote{two-fold} manner. The training data set, consisting of the mixture of various data samples as defined above, is split into two parts based on an ID number which is unique for each event. Then, one network is trained on a half of the data set with even event IDs, while the other on the odd ones. At the prediction step, the networks change halves and the even network is applied to the odd half of the data set, and vice versa. This procedure allows to use all available simulated samples to produce templates for the statistical inference. Furthermore, no bias is introduced due to the usage of the same events for both training and template composition with the same model.  

The loss function for each of the models is a categorical cross-entropy which is minimised with an Adam optimiser \cite{kingma2014adam} with the learning rate $10^{-4}$. The training weights are added to the loss function to balance the difference in expected number of events for across physical processes in the corresponding data-taking period. $90\%$ of the even/odd halves which are provided to each of the models for the training is used for the actual training, while the other $10\%$ is used for validation.

After the training, each event for both data and simulated samples is classified to one of the three categories where the corresponding NN score is the highest. This score is also used further at the statistical inference step (Sec. \ref{sec:stat}). 

Lastly, a requirement on the impact parameter significance $S_\text{IP} > 1.5$ is applied for the electron in events which are classified into a signal category. The same requirement $S_\text{IP} > 1.5$ is applied for the single charged pion from the \tauh candidate with MVA DM equal to 0 (Sec. \ref{sec:reco_tau}) for events both in the signal and background categories. This selection requirement removes events with poorly reconstructed IP vectors. Furthermore, it showed to have slightly better separation between CP-even and CP-odd hypotheses at the reconstructed level without introducing significant deviations to the \phicp distribution in the embedded samples (Fig. \ref{fig:ip_cut}).  

\begin{figure}[!t]
    \centering
    \includegraphics[width=0.32\textwidth]{Figures/CP_etau/ip_cut_emb.png}
    \includegraphics[width=0.32\textwidth]{Figures/CP_etau/ip_cut_even.png}
    \includegraphics[width=0.32\textwidth]{Figures/CP_etau/ip_cut_odd.png}
    \caption{Normalised distributions of the \phicp observable for events passing the selection requirements in Sec. \ref{sec:selection} with the \tauh candidate identified with MVA DM equal to 1 ($e\rho$ final state) from the embedded samples (left), ggH sample under the CP-even hypothesis (middle), and ggH sample under the CP-odd hypothesis (right). The histogram in gray (red) corresponds to no ($S_\text{IP} > 1.5$) selection applied to the IP significance of the electron.}
    \label{fig:ip_cut}
\end{figure}

\begin{figure}[!t]
    \centering
    \includegraphics[width=0.49\textwidth]{Figures/CP_etau/NN_score_embed_et_2018_tau_prefit.pdf}
    \includegraphics[width=0.49\textwidth]{Figures/CP_etau/NN_score_fakes_et_2018_fakes_prefit.pdf}
    \caption{Pre-fit distribution of the NN score in the genuine $\tau$ (left) and fakes (right) background categories for 2018 data-taking period.}
    \label{fig:bkgr_cat_prefit}
\end{figure}

\begin{figure}[H]
    \centering
    \includegraphics[width=0.9\textwidth]{Figures/CP_etau/Bin_number_e-rho_et_2018_higgs_prefit.pdf}
    \caption{Pre-fit blinded distribution of unrolled in bins of the NN score \phicp observable in $e\rho$ signal category for 2018 data-taking period.}
    \label{fig:sig_cat_prefit}
\end{figure}

Overall, good agreement between data and simulation is observed in background categories before performing the statistical fit (Fig. \ref{fig:bkgr_cat_prefit}). For the blinded signal categories, the unrolled distribution of the \phicp observable shows increasing signal over background ratio from the first to the last bins of NN score, with a clear separation between the pure CP-even and CP-odd hypotheses in the most sensitive $e\rho$ category (Fig. \ref{fig:sig_cat_prefit} and Appendix \ref{}).

\section{Statistical inference}\label{sec:stat}
\subsection{Framework}
The strategy to extract the value of the main parameter of interest (POI) \mixa follows the likelihood formalism \cite{Conway:2011in, ATLAS:2011tau, CMS:2014fzn}. The likelihood function is parametrised by several POIs and nuisance parameters as follows: 
\begin{equation}\label{eq:like}
    L(\mixa, \vec{\mu}, \vec{\theta}) = \prod_j^{\text{N}_\text{categories}}\prod_i^{\text{N}_\text{bins}} P\left(n_{ij}|\mathcal{L} \cdot \vec{\mu} \cdot \vec{A}_{ij}(\vec{\theta}, \mixa) + B_{ij}(\vec{\theta})\right) \prod_m^{\text{N}_\text{nuisance}}C_m(\theta_m).
\end{equation}

It is parametrised by:
\begin{itemize}
    \item \mixa: the mixing angle between SM and anomalous couplings as defined in Eq. \ref{eq:mixa}.
    
    \item $\vec{\mu} \equiv (\mu_\text{ggH}, \mu_\text{qqH})$: a vector of the Higgs boson signal strength modifiers which are defined as the ratio of the corresponding cross section times the \htt branching ratio with respect to the SM value. The ggH and qqH processes are considered in the statistical inference, where the latter scales the combined VBF and VH production modes.
    
    \item $\vec{\theta}$: a vector of nuisance parameters corresponding to the systematic uncertainties (Sec. \ref{sec:syst}).
    
\end{itemize}

The likelihood function is computed as a product over categories $j$ and bins $i$. In the analysis of the \et final state the former product is taken over the categories for each of the three (2016, 2017, 2018) data-taking periods as defined by the corresponding neural networks (Sec. \ref{sec:categ}). In the combination with the other channels (Sec. \ref{sec:comb}), the corresponding categories as defined by the analysis of these channels are additionally included in the product. The signal (Higgs) category is further split into four categories based on the MVA DM predictions (Sec. \ref{sec:reco_tau}) for the \tauh candidate with the decay modes MVA DM = \{0, 1, 2, 10\} being considered. Therefore, the final set of categories in the \et analysis is:
\begin{itemize}
    \item $e\pi$ (signal),
    \item $e\rho$ (signal),
    \item $e a_1^\text{1pr}$ (signal),
    \item $e a_1^\text{3pr}$ (signal),
    \item Genuine $\tau$ (background),
    \item Fakes (background).
\end{itemize}

The bins in Eq. \ref{eq:like} corresponds to the bins in the unrolled 2D histogram (\phicp, NN score) for the signal categories and the bins in 1D histogram of NN score for the background categories. The unrolled histogram is constructed by firstly binning the NN score distribution and then plotting for the events in each of the bins the histograms of the \phicp distribution. The following NN score bin edges are used for all the data-taking periods:
\begin{itemize}
    \item Signal categories: [0, 0.45], [0.45, 0.6], [0.6, 0.7], [0.7, 0.8], [0.8, 0.9], [0.9, 1.0].
    \item Genuine $\tau$ category: [0, 0.5], [0.5, 0.6], [0.6, 0.7], [0.7, 1.0].
    \item Fakes category: [0, 0.6], [0.6, 0.7], [0.7, 0.8], [0.8, 0.9], [0.9, 1.0].
\end{itemize}

In the signal categories, for each of the NN bins as defined above, the \phicp distribution has 10, 8, and 4 equally sized bins in the range $[0^\circ, 360^\circ]$ for $e\rho$, $e\pi$, and $e a_1^\text{1pr}/e a_1^\text{3pr}$ categories, respectively.

Counts in each of the bins are modelled in Eq. \ref{eq:like} with a Poisson distribution $P(n_{ij}|n_{ij}^\text{exp})$ where $n_{ij}$ is the observed number of events in data and expected counts $n_{ij}^\text{exp}$ are modelled as a sum of the signal $\mathcal{L} \cdot \vec{\mu} \cdot \vec{A}_{ij}(\vec{\theta}, \mixa)$ and background $B_{ij}(\vec{\theta})$ contributions. Here $\mathcal{L}$ is the integrated luminosity, $\vec{A}_{ij}(\vec{\theta}, \mixa)$) is a vector of signal acceptances for each of the H production modes (ggH and qqH). $\vec{A}_{ij}$ and $B_{ij}(\vec{\theta})$ are produced in a form of templates as described in Sec. \ref{sec:temp}. Constraints on the systematic uncertainties are incorporated as prior probability density functions $C_m(\vec{\theta})$. For uncertainties altering only the normalisation of the counts with the same rate across all the bins (referred to as normalisation uncertainties) these are taken as log-normal distribution. For the uncertainties producing asymmetric count variation across the bins and therefore altering the shape of the templates (referred to as shape uncertainties) these are implemented in the likelihood minimisation as the continuous morphing with a Gaussian prior probability density
function. Parameters of the probability density functions are further described in Sec. \ref{sec:syst}.

The statistical inference is performed using a Combine statistical toolkit \cite{combine}. The main parameter of interest (POI) is the CP mixing angle \mixa. In order to extract its value from data, a test statistic is constructed as a log-likelihood ratio:
\begin{equation}
    q(\mixa) \equiv -2\ln\left(\dfrac{L(\mixa, \hat{\vec{\mu}}, \hat{\vec{\theta}})}{L(\hat{\alpha}^{\text{H}\tau\tau}, \hat{\vec{\mu}}, \hat{\vec{\theta}})}\right),
\end{equation}
where the denominator is the best fit value of the likelihood function with respect to all the POIs (\mixa, $\vec{\mu}$, and $\vec{\theta}$), and the numerator corresponds the likelihood function where all the POIs but the main one are profiled. The value $\hat{\alpha}^{\text{H}\tau\tau}$ which corresponds to the minimum of $q(\mixa)$ is quoted as the best-fit value with the 68.3, 95.5, and 99.7\% confidence intervals obtained using asymptotic approximation as the values of the mixing angle \mixa where $q(\mixa)$ equals to 1.00, 4.02, and 8.81 \cite{Cowan:2010js}.

\subsection{Template composition}\label{sec:temp}
As it was previously mentioned in Sec. \ref{sec:phicp}, one expects certain symmetries to be preserved in the \phicp distribution for the background processes: for example, the fact that genuine tau backgrounds are uniformly distributed at the generator level. It was also mentioned that smearing and resolution effects come into play when one moves from the generator to the reconstructed level. The distributions are therefore distorted and the original symmetries are no longer applicable. However, some symmetries still remain, and they can be exploited as described below.

The motivation to impose symmetries comes from the observation that the statistical fluctuations for the simulated samples in the last bins of the unrolled \phicp distribution are sizeable (Fig. \ref{fig:sig_cat_postfit}). Therefore, in order to constrain the associated statistical bin-by-bin uncertainties it is beneficial to correct the signal and background templates and associated statistical uncertainties to have the expected symmetry properties. This also removes potential bias on the statistical inference which might appear due to the statistical fluctuation in the template bins.

For the background templates the following modifications are applied depending on the background process and the method used to reconstruct the \phicp observable (Fig. \ref{fig:symm_flat}):
\begin{itemize}
    \item Genuine \tauh (IP+NP): flattening.
    \item Genuine \tauh (IP+IP): symmetrisation around $\phicp=180^\circ$.
    \item $\text{jet} \to \tauh$ fakes (IP+NP, IP+IP): symmetrisation around $\phicp=180^\circ$.
    \item $l \to \tauh$ fakes (IP+NP): flattening.
    \item $l \to \tauh$ fakes (IP+IP): symmetrisation around $\phicp=180^\circ$.
\end{itemize}

\begin{figure}[!t]
    \centering
    \includegraphics[width=0.9\textwidth]{Figures/CP_etau/symm_flat.png}
    \caption{Illustration of the symmetrisation (top) and flattening (bottom) procedures applied to the signal and background templates \cite{Cardini:2021hbb}. Red circles represent the bin content in the histogram of the \phicp observable, which originally is distributed randomly around the expected values (dashed lines). After the symmetrisation/flattening, the bin content is adjusted to the value estimated from the averaging across the symmetric bins as described in the text.}\label{fig:symm_flat}
\end{figure}

In general, the flattenning is performed by setting the value of all the \phicp bins in a single NN score bin to their average and introducing a single statistical uncertainty as a fully-correlated quadratic sum of uncertainties of the original bins. The symmetrisation is performed by setting the value of the symmetric with respect to $180^\circ$ pair of bins to their average, also with the single common nuisance parameter as in the flattening case. The effective number of associated nuisance parameters is thus reduced by 1/2 (1/$\text{N}_\text{bins}$) for the symmetrisation (flattening) procedures.

As it can be seen, the templates are kept uniform only for the background processes involving genuine $\tau$ leptons and in the signal categories where the neutral-pion method is used for the \tauh side ($e\rho, e a_1^\text{1pr}, e a_1^\text{3pr}$). The usage of the impact parameter vector for both of the prongs is sensitive to the smearing effects affecting the primary vertex reconstruction. These effects introduce a correlated behaviour in the \phicp reconstruction where the \phicp values of $0^\circ$ and $360^\circ$ are more favoured. However, the symmetry around the \phicp value of $180^\circ$ still holds and can be used. The same applies also to the $\text{jet} \to \tauh$ fake events due to the kinematic properties of these events.

As it was mentioned in Sec. \ref{sec:samples}, the Higgs signal samples are generated according to the three discrete values of $\mixa = \{0^\circ, 45^\circ, 90^\circ\}$, which corresponds to the CP-even, CP-odd, and CP-mix scenarios, respectively. Since in the statistical inference procedure a continuous range of \mixa values is assumed, the general template $T(\mixa)$ for any auxiliary value of the mixing angle is constructed separately for ggH and qqH production from the basis three templates ($T_\text{even}, T_\text{odd}, T_\text{mix}$) as follows:
\begin{align*}
    T(\mixa) = (\cos^2\mixa - \cos\mixa\sin\mixa) \cdot T_\text{even} &+ (\sin^2\mixa - \cos\mixa\sin\mixa) \cdot T_\text{odd} + \\
    &+ 2\cos\mixa\sin\mixa \cdot T_\text{mix}.
\end{align*}

The signal templates for the \phicp observable can also be affected by statistical fluctuations in the highest NN score bins. Therefore, similarly to the background templates, $T_\text{even}$ and $T_\text{odd}$ are symmetrised around $\phicp=180^\circ$. The symmetrisation of the $T_\text{mix}$ template is performed by the generation of an additional signal sample (separately for ggH and qqH templates) corresponding to $\mixa = -45^\circ$ and its averaging with the $T_\text{mix}$ template shifted by $180^\circ$. 

\subsection{Systematic uncertainties}\label{sec:syst}
As it was mentioned earlier, there are two distinct types of systematic uncertainties: normalisation and shape uncertainties. 

Normalisation uncertainties shift the normalisation of the templates without affecting the shape. Their parameters are constrained by adding a log-normal prior with the mean parameter corresponding to the nominal case of no systematic variation. The sigma of the distribution depends on the source of uncertainty and is provided below as a percentage of the nominal value. 

Shape uncertainties are modelled with a continuous morphing procedure with a Gaussian prior on a morphing parameter. The parameter interpolates between two discrete up/down variations of the template corresponding to $\pm 1\sigma$ variations of the Gaussian prior. The mean of the prior distribution is set to 0, corresponding to the nominal template shape. Magnitudes of shape variations are provided below as a percentage of the systematic source variation resulting in $1\sigma$ up/down template variations. 

Since in the statistical inference procedure categories for the three data-taking periods are analysed jointly, some sources of systematic uncertainties are (partially) correlated across the years, resulting in shared nuisance parameters in the fit. These cases are marked in Table \ref{tab:nuis}, which also summarises all the sources of uncertainties incorporated into the statistical inference procedure together with the samples they are applied to. Lastly, uncertainties related to electron/\tauh identification and energy scale are treated as 50\% correlated between the simulated and embedded samples. All the other common uncertainties are taken to be uncorrelated.

\subsubsection{Normalisation}
The following sources of normalisation uncertainties are considered in the analysis:
\begin{itemize}
    \item Electron reconstruction (tracking/ID/isolation) efficiency: 2\%.
    \item Electron trigger efficiency: 2\%.
    \item \tauh ID (against $e,\mu$): 3\%.
    \item b-tagging scale factors: 1-9\%.
    \item Integrated luminosity: 2.5, 2.3, and 2.5\% for 2016, 2017, and 2018 respectively \cite{CMS:2017sdi, CMS:2018elu,CMS:2019jhq}.
    \item Embedded yield: 4\%.
    \item Cross section uncertainties:
    \begin{itemize}
        \item W+jets: 4\%,
        \item Drell–Yan: 2\%,
        \item Diboson: 5\% \cite{CMS:2016jdy},
        \item Single top: 5\% \cite{CMS:2016lel},
        \item \ttbar: 4.2\%,
        \item H: 2–5\% \cite{LHCHiggsCrossSectionWorkingGroup:2016ypw},
        \item \htt branching fraction: 2\% \cite{LHCHiggsCrossSectionWorkingGroup:2016ypw}.
    \end{itemize}
    \item $e\to\tauh$ misidentification rate: up to 10\%, decay mode dependent. 
    \item Impact parameter significance: the $S_\text{IP}$ correction is varied by $\pm25\%$ ($\pm40\%$) for a single pion (electron) and the variation is converted into a normalisation uncertainty in the range of 1-5\%.
\end{itemize}

\subsubsection{Shape}

\begin{table}[ht!]
	\caption{Summary of systematic uncertainties included into the statistical inference as described in Sec. \ref{sec:syst}. The first column describes the source of uncertainty. The second column describes the magnitude of the systematic variations and its dependency on observables. The third column describes the samples to which the uncertainty is applied (where \enquote{MC} corresponds to simulated samples). The fourth column describes if the uncertainty is correlated across the data-taking periods. The fifth column describes the type of uncertainty, where $ln\text{N}$ corresponds to normalisation uncertainty.}
    \centering
	\begin{tabular}{ccccc}
	    \hline
		Uncertainty & Magnitude & Samples & Correlation & Type \\
		\hline
        Electron reconstruction & 2\% & MC & Yes & $\ln\text{N}$\\
        Electron trigger & 2\% & MC & No & $\ln\text{N}$\\
        \tauh ID (against $e,\mu$) & 3\% & MC, embedded & No & $\ln\text{N}$\\
        b-tagging scale factors & 1-9\% & \ttbar, single top & No & $\ln\text{N}$\\
        Integrated luminosity & 2.3-2.5\% & MC & Partial & $\ln\text{N}$\\   
        Embedded yield & 4\% & Emb. & No & $\ln\text{N}$\\  
        W+jets cross section & 4\% & W+jets MC & Yes & $\ln\text{N}$\\
        DY cross section & 2\% & DY MC & Yes & $\ln\text{N}$\\
        Diboson cross section & 5\% & Diboson MC & Yes & $\ln\text{N}$\\
        Single top cross section & 5\% & Single top MC & Yes & $\ln\text{N}$\\
        \ttbar cross section & 4.2\% & \ttbar MC & Yes & $\ln\text{N}$\\
        H cross sections & 2-5\% & Signal MC & Yes & $\ln\text{N}$\\
        \htt branching fraction & 2\% & Signal MC & Yes & $\ln\text{N}$\\
        $e\to\tauh$ rate & 10\% & MC with $e\to\tauh$ & No & $\ln\text{N}$\\
        $S_\text{IP} (e,\pi)$ & 1-5\% & MC & No & $\ln\text{N}$\\
        \tauh reconstruction & \pt/DM dep. & MC, embedded & Partial & Shape\\
        \tauh trigger & \pt/DM dep. & MC & No & Shape\\
        \tauh energy scale & \pt/DM dep. & MC, embedded & No & Shape\\
        Electron energy scale & \pt/$\eta$ dep. & MC, embedded & No & Shape\\
        $e\to \tauh$ energy scale & 0.5-6.5\% & MC with $e\to \tauh$ & No & Shape\\
        Jet energy scale & Event-dep. & MC & Partial & Shape\\
        Jet energy resolution & Event-dep. & MC & No & Shape\\
        \met unclustered scale & Event-dep. & ST, \ttbar, diboson MC & No & Shape\\
        \met recoil corrections & Event-dep. & Z/W+jets, signal MC & No & Shape\\
        \ttbar/diboson in embedded & 10\% & embedded & Yes & Shape\\
        Top quark \pt reweighting & top \pt dep. & ST, \ttbar & Yes & Shape\\
        Z mass and \pt reweighting & Z \pt/mass dep. & DY MC & Partial & Shape\\
        \ff & Described in text & $\text{jet}\to\tauh$ fakes & Partial & Shape\\
        Prefiring & Event-dep. & MC & Yes & Shape\\
        Theoretical uncertainties & Event-dep. & Signal MC & Yes & Shape \\
	\end{tabular} \label{tab:nuis}
\end{table}


The following sources of shape uncertainties are considered in the analysis:
\begin{itemize}
    \item \tauh reconstruction \& ID: up to 3\%, \pt and decay mode dependent.
    \item \tauh trigger: \pt/decay-mode dependent.
    \item \tauh energy scale: 0.8–1.1 (0.2–0.5)\% for simulated (embedded) samples, \pt/decay-mode dependent.
    \item Electron energy scale: $<1\%$, \pt and $\eta$ dependent.
    \item $e\to \tauh$ energy scale: 0.5–6.5\%.
    \item Jet energy scale: event-by-event depending on the jet topology and kinematics. The uncertainties are also propagated to \met and observables which are dependent on \met for the simulated samples where no recoil corrections is applied (single top quark, \ttbar, and diboson production). 
    \item Jet energy resolution: event-by-event depending on the jet topology and kinematics. The uncertainties are also propagated to \met and observables which are dependent on \met for the simulated samples where no recoil corrections is applied (single top quark, \ttbar, and diboson production).
    \item \met unclustered scale: event-dependent, used for the samples where recoil corrections are not applied (single top quark, \ttbar, and diboson production). The uncertainties are also propagated to \met and observables which are dependent on \met.
    \item \met recoil corrections: event-dependent, applied for the Z+jets, W+jets and signal samples. The uncertainties are also propagated to \met and observables which are dependent on \met.
    \item \ttbar/diboson in the embedded samples: 10\% of \ttbar and diboson contribution as estimated from the simulation is added/subtracted in the embedded templates. This is aimed to reduce a potential bias introduced by the embedding procedure to the selection of genuine tau lepton pairs originating from these processes.
    \item Top quark \pt reweighting: $\pt(t)$ dependent, defined with up (down) variation corresponding to twice (no) correction size.
    \item Z mass and \pt reweighting: $\pt(Z), m(Z)$ dependent, defined with up/down variation corresponding to $\pm10\%$ of the correction size.
    \item $F_\text{F}$: uncertainties associated with \met/electron \pt non-closure corrections and extrapolation to same-sign/high-$m_\text{T}$ regions corrections are applied for each of the fake factors (W+jets, QCD, \ttbar). For \ttbar \ff (derived using simulated samples) an uncertainty is added to account for the differences between data and simulation. Additional uncertainty is assigned due to the subtraction of background processes without $\text{jet}\to\tauh$ fakes.
    \item Prefiring: 0-4\%, dependent on the process and category.
    \item Theoretical uncertainties: event-dependent, applied to the signal samples these include renormalisation and factorisation scales and parton showering uncertainties.
\end{itemize}

\section{Results}\label{sec:results}

After the likelihood minimisation is performed (Sec. \ref{sec:stat}), one can firstly investigate if there is no significant discrepancies between data and simulation in the post-fit distributions in the background categories. Overall, good description of data with the fitted templates is observed for all the categories and data-taking periods (Fig. \ref{fig:bkgr_cat_postfit} and Appendix \ref{}).

In a more formalised way, one can perform a goodness-of-fit (GoF) test to estimate if there is a statistically significant difference between data and fitted templates. Results of the saturated model GoF test \cite{Cousins2013GeneralizationOC} performed for the combination of all the categories (signal and background) and all the data-taking years (2016, 2017, 2018) show the $p$-value of 0.22, which also indicates a good quality of the fit.

\begin{figure}[H]
    \centering
    \includegraphics[width=0.4\textwidth]{Figures/CP_etau/NN_score_embed_et_2018_tau_postfit.pdf}
    \includegraphics[width=0.4\textwidth]{Figures/CP_etau/NN_score_fakes_et_2018_fakes_postfit.pdf}
    \caption{Post-fit distribution of the NN score in the genuine $\tau$ (left) and fakes (right) background categories for 2018 data-taking period.}
    \label{fig:bkgr_cat_postfit}
\end{figure}

\begin{figure}[H]
    \centering
    \includegraphics[width=0.55\textwidth]{Figures/CP_etau/pulls.png}
    \caption{Summary of the post-fit analysis of the 30 leading nuisance parameters (left panel). In the middle panel a value of $(\hat{\theta} - \theta_0)/\Delta\theta$ is shown for each nuisance parameter, where $\hat{\theta}$ is a post-fit value, $\theta_0$ is a pre-fit nominal value, and $\Delta\theta$ is a nominal variance. The error bars correspond to the $68.3\%$ ($1\sigma$) confidence level as obtained from the profiled likelihood scan. In the right panel, an impact distribution is shown, where each nuisance parameter is varied by $\pm1\sigma$ and the corresponding variation of the main POI (\mixa) from its best fit value is shown as a red/blue bar.}
    \label{fig:pulls}
\end{figure}

\begin{figure}[H]
    \centering
    \includegraphics[width=0.99\textwidth]{Figures/CP_etau/Bin_number_e-rho_et_2018_higgs_postfit.png}
    \includegraphics[width=0.99\textwidth]{Figures/CP_etau/Bin_number_e-pi_et_2018_higgs_postfit.png}
    \caption{Post-fit distribution of unrolled in bins of the NN score \phicp observable in the two most sensitive $e\rho$ (top) and $e\pi$ (bottom) signal categories for 2018 data-taking period.}
    \label{fig:sig_cat_postfit}
\end{figure}

In order to gauge the behaviour of the systematic uncertainties in the fit, a scan of the likelihood function Eq. \ref{eq:like} is performed for each of the nuisance parameter with all the POIs except for the nuisance parameter being profiled. Results are shown on Fig. \ref{fig:pulls} for the first 30 leading nuisance parameters as well as the impact of the each nuisance parameter variation on the main POI \mixa. Overall, no anomalous behaviour is observed in the nuisance parameter diagnostics.  

The signal categories are then unblinded and the resulting unrolled distributions of \phicp observable are shown in Fig. \ref{fig:sig_cat_postfit} and Appendix \ref{}. A slight presence of the \htt signal is visible in the bins of the NN score, albeit diluted by statistical uncertainties. 

Before proceeding to the extraction of the observed \mixa value from the fit to data it is good to have an estimate of what one would expect under the null hypothesis, which is the Standard Model. The expected values of \mixa are obtained with the same procedure as described in Sec. \ref{sec:stat} but with the template fit being performed to the Azimov data set. The latter is obtained with fixing the cross sections of all the physical processes to their SM values and the \phicp distribution to the prediction of the pure CP-even hypothesis. 

The obtained expected value of the CP mixing angle is $\mixa_\text{exp} = 0^\circ \, \pm \, 90^\circ$ (Fig. \ref{fig:mixa}, left), which corresponds to the expected CP sensitivity of $0.99 \sigma$. The latter describes the statistical significance to exclude the pure CP-odd hypothesis if taken as a null hypothesis. Expected sensitivities split by the final states are:
\begin{itemize}
    \item $e\tauh$ (total): $0.99\sigma$, 
    \item $e\rho$: $0.57\sigma$,
    \item $e\pi$: $0.54\sigma$, 
    \item $ea_1^\text{3pr}$: $0.38\sigma$, 
    \item $ea_1^\text{1pr}$: $0.17\sigma$. 
\end{itemize}

% \begin{figure}[h!]
%     \centering
%     \includegraphics[width=0.55\textwidth]{Figures/CP_etau/mutautau_exp.png}
%     \caption{.}
%     \label{fig:mixa_exp}
% \end{figure}

One can further introduce $\mu$ as an inclusive signal strength modifier which scales the cross section times the \htt branching fraction of all the three production modes altogether (opposite to the two separate $\mu_\text{ggH}$ and $\mu_\text{qqH}$ used to obtained the final result). The likelihood scan on the Azimov data set gives its expected value $\mu_\text{exp} = 1.00^{+0.26}_{-0.24}$. Overall, the expected values of both CP sensitivity and the signal strength show that the \et channel is not sufficient on its own to provide significant information about the CP structure of the $\text{H}\tau\tau$ interaction. However, as it is shown in Sec. \ref{sec:comb} a combination with the other \mt and \tata leads to conclusive results.

\begin{figure}[H]
    \centering
    \includegraphics[width=0.49\textwidth]{Figures/CP_etau/alpha_exp.png}
    \includegraphics[width=0.49\textwidth]{Figures/CP_etau/alpha.png}
    \caption{Left: the profiled likelihood scan for the \mixa parameter of interest on the Azimov data set. Right: the profiled likelihood scan for the \mixa parameter of interest on the observed data set.}
    \label{fig:mixa}
\end{figure}

Substituting the Azimov data set with the observed data, one can obtain with the same statistical inference procedure the observed values of the CP mixing angle $\mixa_\text{obs} = -48^{+51^\circ}_{-42^\circ}$ (Fig. \ref{fig:mixa}, right). The observed value of the inclusive signal strength modifier is $\mu_\text{obs} = 1.14^{+0.27}_{-0.25}$, which is in agreement with the expectation. 

To summarise, the \et final state is \textit{a priori} expected to provide subleading contribution the analysis in terms of the CP sensitivity. This is due to the challenges in the electron reconstruction where the bremsstrahlung significantly impacts the resolution of the impact parameter vector, which in turn decreases the separation between CP-even and CP-odd hypothesis with the \phicp observable. Moreover, a larger number of jets misidentified as electrons leads to higher \pt thresholds at the trigger level, which further reduces the electron selection efficiency. Quantitatively, the observed (expected) value of the CP mixing angle is obtained to be $\mixa = -48^{+51^\circ}_{-42^\circ} (0^\circ \, \pm \, 90^\circ)$ which does not hint to any preferable CP hypothesis with the main limiting factor being the lack of statistics in this final state. Nevertheless, the expected CP sensitivity in the \et channel is comparable with the expected contribution of the most sensitive $\mu\rho, \rho\rho, \rho\pi$ final states ($1.16\sigma$, $1.10\sigma$, and $1.08\sigma$) respectively. Therefore, it plays an important role in the final combination of all the final states considered in the CP analysis of the $\text{H}\tau\tau$ coupling, as described in Sec. \ref{sec:comb}.
% \chapter{Tau lepton reconstruction \& identification}\label{sec:tau}
% \pagestyle{plain}

The tau lepton, being the heaviest of the three discovered charged leptons in the Standard Model, plays a crucial role in understanding of matter at the most fundamental level. For example, in the context of the minimal supersymmetric extension of the SM (MSSM) \cite{Fayet:1974pd, Fayet:1977yc}, there is a special interest in searches for neutral and charged Higgs bosons decaying into a pair of tau leptons \cite{CMS:2022goy}. Furthermore, in the light of testing the Lepton Flavour Universality (LFU) several observed tensions with the SM predictions are yet to be understood \cite{HFLAV:2022pwe, Cheaib:2022ral, LHCb:2017vlu, LHCb:2021trn}. 

The precision of such analyses heavily relies on the ability in a given experiment to accurately reconstruct and separate tau leptons from background processes. However, the tau lepton stands out from the other leptons in its properties, which poses several challenges in this endeavour. In particular, it is the only lepton known to decay into hadrons, which makes it difficult to distinguish such decays (hereafter labelled as \tauh) from jets originating from QCD processes. Therefore, this Chapter will take an experimentalist's perspective and will describe, with a particular emphasis on the CMS experiment, challenges and achievements accomplished so far in the \tauh reconstruction and identification (hereafter also referred to as \enquote{tau lepton reconstruction/identification}).

The Chapter is organised as follows. In Section \ref{tau-intro} a brief overview of the tau lepton's history and its properties measured to this date is given. Section \ref{hps} introduces methods to reconstruct tau leptons in the CMS experiment, in particular a hadron-plus-strip (HPS) algorithm built on top of the Particle Flow (PF) algorithm, described in Section \ref{pf}. After the reconstruction of tau lepton candidates, an identification step has to be performed to categorise whether a given candidate originates from a genuine tau or a jet/lepton faking tau. An algorithm named DeepTau was developed for that purpose and its details -- including the recent improvement in the context of the Run 3 data taking -- will be described in Sections \ref{deeptau1} and \ref{deeptau5}. However, the algorithm has several intrinsic limitations in its design, and ongoing efforts to overcome them with new Machine Learning (ML) models will be detailed in Section \ref{tat}.

\section{Discovery \& Properties} \label{tau-intro}

The tau lepton was observed for the first time in the Mark I experiment at the SPEAR $e^+e^-$ storage ring at the Stanford Linear Accelerator Center (SLAC) in 1974 by Martin L. Perl et al. \cite{Perl:1975bf}. Fundamentally, the motivation \cite{Perl:1992ad} behind the analysis was to solve  an electron-muon problem, which manifests itself in two questions \cite{Perl:1996dk}:
\begin{itemize}
    \item Why is the muon 206.8 times heavier than electron?
    \item Why doesn't the muon decay through the process $\mu \to e + \gamma$?
\end{itemize}

One of the ideas to understand this difference was to change the perspective and search for additional heavy leptons, which, in case of their existence, could help to gain insights into the initial problem. The theoretical framework to search for such leptons was a sequential heavy lepton model due to its elegance, symmetry and simplicity \cite{Perl:past_future}. The minimalistic and main assumption it makes is the existence of pairs ($L_\alpha, \nu_\alpha$) of charged leptons and associated neutrinos with the lepton masses larger than those of the electron and the muon. 

Additionally, the sequential heavy lepton model builds upon the concept of the lepton number conservation. In the original formulation, it postulates that electron and muon each possess a unique property not possessed by other particle, a lepton family number, meaning that electron $e^-$ and its associated neutrino $\nu_e$ are assigned a lepton number $n_e = +1$, $\mu^-$ and $\nu_\mu$ receive a number $n_\mu = +1$ and antiparticles have the corresponding number negative. This lepton number should be preserved in reactions separately for each of the lepton family.

Assuming the lepton family conservation, the sequential heavy lepton model expands this principle to other lepton families with higher masses.  From these principles it follows that given a high enough mass of a heavy charged lepton $L^-$, there should exist the following decays:
\begin{enumerate}[label=D\arabic*]
    \item \label{Ltoe} $L^- \to e^- \bar{\nu_e} \nu_L $, 
    \item \label{Ltomu} $L^- \to \mu^- \bar{\nu_\mu} \nu_L $,
    \item $L^- \to \pi^- \nu_L $,
    \item $L^- \to \pi^-\pi^+\pi^- \nu_L $,
\end{enumerate}
where the former two are exactly analogous to the corresponding decay of the muon into electron and two neutrinos via the weak interaction.

\begin{figure}[ht!]
    \centering
    \includegraphics[width=0.6\textwidth]{Figures/Tau/tau_discovery.png}
    \caption{The observed background-subtracted cross section versus center-of-mass energy in the Mark I experiment within the detector acceptance for events with the $e^\pm\mu^\mp$ signature. \cite{Perl:1975bf}}
    \label{fig:tau-discovery}
\end{figure}

After making an additional ansatz that heavy charged leptons can be produced similarly to electrons and muons in reactions $e^+e^- \to L^+L^-$ ,  Martin L. Perl et al. proposed an elegant idea to search for them in this production mode by looking into a process where one $L$ would decay via \ref{Ltoe} and the other via \ref{Ltomu}. Such an unusual final state consisting of $e^\pm$ and $\mu^\mp$ of opposite charge and a missing energy due to neutrinos escaping detection would hint to anomalous processes appearing in the detector. An excess of such events over the small background expectations (Fig. \ref{fig:tau-discovery}) for which the analysts had \enquote{no conventional explanation} was exactly what was observed at SPEAR. After the follow-up studies \cite{Feldman:1976fm, PLUTO:1977ctk, Barbaro-Galtieri:1977kfn, Bartel:1978ii, Bacino:1978gb, Bacino:1978wj} it was finally concluded, that the observed excess indeed can be attributed to a lepton with the mass 3600 times the electron mass and 17 times the muon mass, later called \textit{tau} (from greek $\tau\rho\iota\tau o\nu$, \enquote{third}). These results together with an observation of the tau neutrino by the DONUT collaboration \cite{DONUT:2000fbd} therefore established the existence of the third generation of leptons.

Since the era of its discovery, the properties of the tau lepton has been extensively studied in several experiments including Belle, BaBar, BESIII, CLEO, KEDR, LEP experiments, PLUTO, and others. They can be summarised as follows \cite{ParticleDataGroup:2020ssz}:
\begin{itemize}
    \item Mass $m_\tau = 1776.86 \pm 0.12$ MeV.
    \item Mean lifetime $\tau = (290.3 \pm 0.5) \cdot 10^{-15}$ s, with the lifetime difference between $\tau^+$ and $\tau^-$: $(\tau_{\tau^+} - \tau_{\tau^-})/\tau_\text{average} < 7.0 \times 10^{-3}$ at 90\% C.L.  
    \item Decay modes (DM): notable feature is the existence of hadronic decays, not present for the other leptons. A brief summary of those decay modes relevant to this study is presented in Table \ref{tab:tau_decays}, inspired by \cite{CMS:2022prd}.
\end{itemize}

\begin{table}[ht!]
	\caption{Decay modes of the tau lepton with the corresponding branching fractions $\mathcal{B}$ \cite{ParticleDataGroup:2020ssz}. If applicable, intermediate known resonances contributing to decay modes are mentioned. $\text{h}^\pm$ denotes a charged hadron and the same numbers apply for the charge-conjugated decays.}
    \centering
	\begin{tabular}{c|c|c}
		Decay mode & Resonance & $\mathcal{B}$ (\%)\\
		\hline
		Leptonic decays & & \multicolumn{1}{l}{35.2}\\
        $\tau^- \to e^- \bar{\nu_e} \nu_\tau$ &  & 17.8\\
        $\tau^- \to \mu^- \bar{\nu_\mu} \nu_\tau$ &  & 17.4 \\
        \hline
        Hadronic decays  & & \multicolumn{1}{l}{64.8} \\
        $\tau^- \to \text{h}^- \nu_\tau$ & & 11.5 \\
        $\tau^- \to \text{h}^- \pi^0 \nu_\tau$ & $\rho(770)$ & 25.9 \\ 
        $\tau^- \to \text{h}^- \pi^0 \pi^0 \nu_\tau$ & $\text{a}_1(1260)$ & 9.5 \\
        $\tau^- \to \text{h}^- \text{h}^+ \text{h}^- \nu_\tau$ & $\text{a}_1(1260)$ & 9.8 \\
        $\tau^- \to \text{h}^- \text{h}^+ \text{h}^- \pi^0 \nu_\tau$ & & 4.8 \\
        Other & & 3.3 \\
	\end{tabular} \label{tab:tau_decays}
\end{table}

% \begin{table}[h]
% 	\caption{...}
%     \centering
% 	\begin{tabular}{c|c|c}
% 		Decay mode & Resonance & \mathcal{B} (\%)\\
% 		\hline
% 		Leptonic decays & & 35.2\\
%         $\tau^- \to e^- \bar{\nu_e} \nu_\tau$ &  & $17.82 \pm 0.04$\\
%         $\tau^- \to \mu^- \bar{\nu_\mu} \nu_\tau$ &  & $17.39 \pm 0.04$ \\
%         \hline
%         Hadronic decays  & & 64.8 \\
%         $\tau^- \to \text{h}^- \nu_\tau$ & & $11.51 \pm 0.05$ \\
%         $\tau^- \to \text{h}^- \pi^0 \nu_\tau$ & $\rho(770)$ & $25.93 \pm 0.09$ \\ 
%         $\tau^- \to \text{h}^- \pi^0 \pi^0 \nu_\tau$ & $\text{a}_1(1260)$ & $9.48 \pm 0.10$ \\
%         $\tau^- \to \text{h}^- \text{h}^+ \text{h}^- \nu_\tau$ & $\text{a}_1(1260)$ & $9.80 \pm 0.05$ \\
%         $\tau^- \to \text{h}^- \text{h}^+ \text{h}^- \pi^0 \nu_\tau$ & & $4.76 \pm 0.05$ \\
%         Other & & 3.3 \\
% 	\end{tabular} \label{tab:tau_decays}
% \end{table}

\section{Reconstruction in CMS}
\subsection{Particle Flow algorithm} \label{pf}
% \subsubsection{Motivation}
In order to perform physical measurements in the particle physics context, one usually operates with an abstract notion of a \textit{physics object}, which is an entity reconstructed from the signals observed in the detector and representing a particle candidate of a particular kind. The goal of the reconstruction process is to build physics objects which are as close and as representative as possible of the genuine particles appearing in the detector. Since the precision of the object reconstruction has a direct impact on the precision of the physical measurement, it therefore plays a crucial role in every particle physics analysis.

Historically, the reconstruction of particles of a given type was primarily based on the information of detector's subsystems which were specifically built to identify them. For example, the reconstruction of electrons and photons was primarily based on the ECAL response and was aimed to capture rather isolated particles. This approach can be referred to as \textit{local}, because it does not make a full use of the signals across all detector subsystems due to a technical granularity limitation. 

With detectors becoming more fine-grained one could turn from a local to a \textit{global} approach of the physics objects' reconstruction. With that it became possible to build a \textit{holistic} image of an event in the detector by linking information from various detector subsystems. This is exactly the core idea behind a \textit{particle-flow (PF) algorithm} \cite{CMS:2017yfk}, developed in the CMS experiment, which aims at tracing the entire \enquote{flow} of particles as they are traversing the detector.

\subsubsection{Basic elements}\label{sec:pf_base}
The PF algorithm follows a hierarchical approach in the reconstruction of physics objects. The first step in the algorithm is to construct basic PF elements which will later serve as a basis for building more complex high-level objects. The main PF elements being constructed at this point are:

\begin{itemize}
    \item Charged-particle tracks,
    \item Electron and muon tracks,
    \item Preshower, ECAL and HCAL energy clusters.
\end{itemize}

\begin{figure}[ht!]
    \centering
    \includegraphics[width=.95\textwidth]{Figures/Tau/track_eff.png}
    \caption{Efficiency (left) and misreconstruction rate (right) as a function of the reconstructed track \pt for the charged hadrons in a sample of simulated QCD multijet events for the track reconstruction algorithm in CMS \cite{CMS:2017yfk}. Black squares correspond to the global combinatorial track finder \cite{CMS:2006myw}, green triangles to the prompt iterations of the iterative procedure \cite{CMS:2014pgm} seeded by at least one hit in the pixel detector, red dots to all the iterations of the procedure.}
    \label{fig:track_eff}
\end{figure}

For the reconstruction of \textbf{charged-particle tracks}, a pattern recognition approach using a combinatorial Kalman filter has been an indispensable tool among experimentalists for decades, also within the CMS experiment \cite{Adam:2005cg}. To improve the overall track reconstruction efficiency while keeping the misreconstructed rate at the same level, an iterative approach is taken \cite{CMS:2014pgm}. With each fitting iteration, it targets to recover inefficiencies for a specific track type by e.g. tailoring the seed construction to a given track type, which allows the loosening of the requirement on the number of hits. The track types include prompt or displaced high/low \pt tracks, tracks inside high \pt jets, and muon tracks. Overall, the iterative procedure brings a significant recovery in efficiency across the \pt range while also performing twice faster compared to a single iteration approach due to the optimised seed construction (Fig. \ref{fig:track_eff}).

% Conceptually, it is a recursive algorithm predicting the evolution of some system in time, which at each iteration consists of two steps:
% Firstly, given an estimate of a system's state \hat{x_{t-1|t-1}} and its covariance P_{t-1|t-1} from the previous time step, a prediction of the state \hat{x_{t|t-1}} and its covariance P_{t|t-1} at the next time step is derived from a physical model describing the dynamics of the system.
% Secondly, a measurement is obtained and the new state estimate \hat{x_{t|t}} and its covariance P_{t|t} are derived as a weighted average of the already predicted and just observed information, where more certain configurations of the system receive higher weights. 

% Projecting this approach to a tracking domain, the time steps correspond to tracker layers starting from the innermost and going outwards with the dedicated generation of seeds for the first step [ref]; observations are the hits in these layers; and the physical model corresponds to the equations of motion of a charged particle in a constant magnetic field, accounting for multiple scattering and energy loss. Additionally, after the convergence of trajectory building procedure with the Kalman filter the final track fitting and smoothing is performed to determine the particle properties.

% However, the traditional pattern recognition approach had the limitation in the efficiency of charged hadron reconstruction. For example, due to the stringent requirements in the number of tracker hits which may not be satisfied for low pt particles because they undergo nuclear interactions with the detector material.

\begin{figure}[ht!]
    \centering
    \includegraphics[width=.6\textwidth]{Figures/Tau/ele_eff.png}
    \caption{Efficiency for electrons (triangles) from simulated b quark jets and charged hadrons (circles) to give rise to an electron seed as a function of \pt. Efficiencies for both ECAL-only seeding (hollow symbols) and ECAL-only with added tracker-based seeding (solid symbols) are displayed.}
    \label{fig:ele_eff}
\end{figure}

Reconstruction of \textbf{electron tracks} largely profits from the iterative tracking procedure, since the latter allows to efficiently reconstruct electrons radiating bremsstrahlung photons. It therefore forms the basis of a tracker-based seeding method, in contrast to a conventional ECAL-based, which suffers from misreconstruction of non-isolated electrons as well as radiating electrons. Generally, for electrons with a small fraction of radiated energy, the tracks are usually well-reconstructed and therefore can be extrapolated to the ECAL surface and matched with the closest ECAL cluster to form an electron candidate seed. However, when energetic photons are radiated, the pattern recognition may result in tracks fitted with large $\chi^2$ values. For such tracks, an additional fitting is performed with a Gaussian-sum filter (GSF) \cite{adam2005reconstruction}, which essentially is an adaptation of the Kalman filter fitting accounting for possible sudden and substantial energy losses along the electron's trajectory. Finally, both tracker-based and ECAL-based electron seed are passed to an extended GSF algorithm for a full track reconstruction. Overall, this procedure significantly improves electron reconstruction efficiency by up to a factor of two compared to an ECAL-based only approach, while also extending the allowed \pt range down to 2 GeV (Fig. \ref{fig:ele_eff}).

While electron track reconstruction was largely improved in the context of the PF algorithm development, \textbf{muon tracking} is not its specific part but a standalone step \cite{CMS:2018rym}. The output of this procedure is a collection of muon track candidates of three types:

\begin{itemize}
    \item \textit{Standalone-muon tracks} built by running the pattern recognition on the hits exclusively in the muon spectrometer subsystems (DT, CSC, RPC).
    \item \textit{Tracker muon tracks} reconstructed with an "inside-out" approach, where firstly the tracker tracks are extrapolated to the muon system. If at least one muon segment matches to extrapolated track, the inner tracks is declared as a tracker muon.
    \item \textit{Global muon tracks} reconstructed with an "outside-in" approach, where standalone muon tracks are being matched with tracker tracks. The matching is performed by propagating both tracks to a common surface and is followed by a combined fit with the Kalman filter. 
\end{itemize}
The resulting approach yields an excellent reconstruction of muon with about 99\% of the muons produced in the acceptance region of the detector being classified as either global or tracker muon tracks.

\textbf{Calorimeter clusters} reconstruction is an essential part of the PF algorithm. On the one hand, it plays a crucial role in the identification of neutral particles for which no track information is available, and their separation from the charged particles. On the other hand, it brings additional information to the reconstruction of electrons radiating bremsstrahlung  photons and also charged hadrons. Once combined with the track information, it helps to sizeably increase the purity of the final collection of physics objects.

The building of calorimeter clusters, performed separately in each subdetector (ECAL barrel/endcap, HCAL barrel/endcap, preshower layers), consists in two steps: 

\begin{enumerate}
    \item \textit{Clustering.} Firstly, seeds are formed from the calorimeter cells with the deposited energy larger of the neighbouring cells and above a given seed threshold. The neighbouring cells are either the four closest cells sharing a side with the seed candidate, or the eight closest cells sharing either a side or a corner with the seed candidate. Then, topological clusters are grown on the principle of recursively adding neighbouring cells, passing a certain set of energy requirements, to the current cell, starting from the seeds candidates.
    \item \textit{Energy attribution.} Due to a potential overlap of topological clusters, a dedicated approach was developed to allow for sharing of the cells energy across several clusters. Performed with an expectation-maximisation algorithm based on a Gaussian-mixture model, it assumes that the energy deposited in a topological cluster is a composition of as many spatially Gaussian-distributed energy clusters, as there are seeds in the topological clusters. The model parameters (location and total energy of each cluster in the Gaussian mixture) obtained after convergence are used as cluster parameters in the downstream reconstruction. 
\end{enumerate}

After the clusters are formed, it is important to calibrate the response of the calorimeters to have a correct energy scale and identification of neutral and charged particles. Performed for ECAL \cite{CMS:2013lxn} and HCAL \cite{CMS:2019hpr} by fitting a parametric function which maps true values of energy and pseudorapidity of the cluster to the calibrated ones, the calibration procedure was successfully validated on data and showed overall improved energy response compared to a raw one.

\subsubsection{Linking \& Object construction}
Having formed the fundamental PF elements, the PF algorithm proceeds to linking them between each other in order to form \textit{PF blocks}, essentially representing chains of PF elements. In order to reduce the computing time, the linking is performed as a search for the nearest neighbours in the $\eta-\phi$ plane via a k-dimensional tree \cite{10.1145/361002.361007}.

With the PF blocks being built, the PF algorithm proceeds to the identification of physics objects on the per block basis. The objects are formed in the following order: muons,  electrons together with isolated photons (converted or unconverted), hadrons (charged or neutral) and nonisolated photons (e.g. from $\pi^0$ decays). To account for possible nuclear interactions in the tracker material, secondary charged-particle tracks which are linked via a common nuclear-interaction vertex are merged into a single primary particle. Lastly, events are post-processed to resolve rare cases of anomalously large $\pt^{\text{miss}}$ coming from e.g. the misreconstruction of muons. A detailed description of requirements applied to PF elements within a block at each of the steps can be found in the original paper \cite{CMS:2017yfk}.

Once all PF blocks are emptied and physics objects of the aforementioned types are formed, one can proceed to grouping them into more complex objects. One of them is the tau lepton object, which reconstruction with a \textit{hadron-plus-strips (HPS) algorithm} is described in the next section. 

\subsection{HPS algorithm} \label{hps}

Since the tau lepton in about 65\% of all cases decays into the final state with hadrons (Section \ref{tau-intro}), it is important to efficiently identify such topologies in the detector. While the leptonic decays of the tau lepton in the CMS experiment are handled by the usual techniques for muon \cite{CMS:2018rym} and electron \cite{CMS:2020uim} reconstruction and identification, the hadronic decays pose a challenge of separating them from an overwhelming background of QCD jets. To tackle this, a hadron-plus-strips (HPS) algorithm was designed, originally for the LHC operation at $\sqrt{s} = $ 7 TeV \cite{cms2012performance} and 8 TeV \cite{CMS:2015pac}, followed by improvements for the data taking at $\sqrt{s} = $ 13 TeV \cite{CMS:2018jrd} and for the tau lepton identification with a DeepTau algorithm \cite{CMS:2022prd}. Below, the most recent overview of the HPS algorithm is provided with relevant references to the original implementation.

As it was previously mentioned, the main challenge in reconstructing hadronic tau decays is that of efficiently distinguishing them from a large amount of jets originating from quarks or gluons. However, hadronic decay products of the tau are usually more collimated compared to those of the QCD jets. In addition, \piz in the final state coming from intermediate $\rho(770)$ or $a_1(1260)$ resonances provide a unique handle to identify genuine \tauh as well as its corresponding decay modes (DM). In this work, DMs (Table \ref{tab:tau_decays}) are enumerated according to the formula: $\text{DM} = 5 \cross (N(\text{h}^\pm) - 1) + N(\pi^0)$, where $N(\text{h}^\pm)$ and $N(\pi^0)$ are the numbers of the reconstructed (or identified, depending in the context) charged prongs and strips/$\pi^0$, respectively. 

Motivated by these observations, the HPS algorithm starts from constructing so-called \textit{strips}, which serve as a proxy for \piz particles. In the detector, a \piz promptly decays into a pair of photons, which consequently, due to a sizeable amount of the tracker material, are very probable to convert to a pair of electrons, which can furthermore radiate bremsstrahlung photons, etc. In the presence of the magnetic field, the electrons trajectories are bent and therefore, on the ECAL surface in the $\eta-\phi$ plane, the clusters associated to \piz decay products have an extended \enquote{strip} shape in the $\phi$ direction.

In order to construct a strip, an iterative clustering procedure with the following steps is performed:

\begin{enumerate}
    \item In an event, hadronic jets are reconstructed by clustering PF particles using the anti-$k_T$ algorithm \cite{Cacciari:2008gp} (Sec. \ref{sec:jets}) with the distance parameter $\Delta R = 0.4$. For each jet, all PF particles in the cone of radius $\Delta R = \sqrt{\Delta \eta^2 + \Delta  \phi^2} = 0.5$ around the axis are passed as an input to the next step. 
    %%%
    \item Within a jet, a strip is seeded by a highest \pt photon or electron that is not yet included in any strip. The algorithm proceeds with a one-by-one aggregation of electrons/photons with $\pt > 0.5$ GeV within a $(\Delta \eta, \Delta  \phi)$ window in the $\eta$-$\phi$ plane of the dynamically adjusted size (originally, of the fixed size \cite{CMS:2015pac}). The size of the strip window is a parametrised function of \pt of the strip at the current iteration and the $e/\gamma$ to be included in the strip \cite{CMS:2018jrd}. The functional form is derived from simulated single tau events with a uniform \pt spectrum with the goal of capturing 95\% of possible $e/\gamma$ in \tauh decay products. In case of adding an $e/\gamma$ candidate to the strip, its position is recomputed as a \pt-weighted average of the coordinates in the $\eta-\phi$ plane of all the strip's constituents, and the strip momentum is set to a sum of the strip's constituents momenta. The procedure is terminated if there is no other $e/\gamma$ within a $(\Delta \eta, \Delta  \phi)$ window and the clustering of a new strip continues with selecting a new seed.
    %%%
    \item For each jet seed, \tauh hypotheses are formed by combining reconstructed strips with the charged PF candidates. Combinations are formed on the basis of decay modes to be targeted: \h, $\h \pi^0$, $\h \pi^0 \pi^0$, $\h \text{h}^\mp \h$ , $\h \text{h}^\mp \h \pi^0$, $\text{h}^\pm\text{h}^{\pm/\mp} (\pi^0)$, where \piz represents a reconstructed strip, \h a charged PF candidate and the last category targets $\tau^- \to \text{h}^- \text{h}^+ \text{h}^- \pi^0$ with one of the charged hadrons ($\pi^0$) escaping detection. The latter two were included into the reconstruction workflow together with the DeepTau algorithm \cite{CMS:2022prd}. In order to be assigned to a DM category, each combination is required to pass a mass window constraint to be compatible with the corresponding intermediate resonance (Table \ref{tab:tau_decays}). Originally, the mass window was statically defined but later it was updated to be dynamically dependant on the strip \pt. In the following, $\h \pi^0$, $\h \pi^0 \pi^0$ DMs are analysed together and referred to as $\h \pi^0$. 
    %%%
    \item Among the \tauh hypotheses formed at the previous step, a set of further requirements is applied. \tauh candidates should  have a charge $\pm 1$, except for the DMs with the missing charged hadron, where the \tauh charge is set to the charge of the charged hadron with the highest \pt. All reconstructed \h and strips in the combination should be located within the tau signal cone defined by the radius $R_\text{sig} = 3.0 / \pt \, \text{(GeV)}$, limited to the range 0.05-0.10, with respect to the \tauh momentum. Finally, for each seeding jet a single \tauh candidate with the highest \pt is selected.
    \end{enumerate}

\begin{figure}[t!]
    \centering
    \includegraphics[width=.6\textwidth]{Figures/Tau/hps_confusion.png}
    \caption{A fraction of \tauh candidates with a given generated decay mode to be reconstructed by the HPS algorithm in different decay modes \cite{CMS:2022prd}.}
    \label{fig:hps_confusion_matrix}
\end{figure}

Overall, more than a half of each of the most significant \tauh decay modes (\h, $\h \pi^0$, $\h \text{h}^\mp \h$) is reconstructed in the targeted DMs (Fig. \ref{fig:hps_confusion_matrix}). Although $\text{h}^\pm\text{h}^{\pm/\mp} (\pi^0)$ category helps to recover 19\% (13\%) of $\h \text{h}^\mp \h$ ($\h \text{h}^\mp \h \pi^0$) DMs, it is not considered in the main \tauh reconstruction routine due to its large charge mis-assignment probability. Despite the fact that DM reconstructed efficiencies are naturally bounded by the 90\% efficiency of the charged track reconstruction and even lower efficiency for photons coming from \piz decays, one can observe that there is still room for improvement in the reconstruction of all DMs. This is particularly true for the DMs with one charged prong \h and at least one \piz, where the HPS algorithm fails to reconstruct 25\% of these DMs, which amounts to $\approx$8\% of all possible tau decays. Therefore, this motivates future studies in the direction of improvement of the HPS algorithm.

\section{Identification in CMS}
Conceptually, the reconstruction step, starting from the PF algorithm (Section \ref{pf}) and going hierarchically to more complex algorithms, e.g. jet clustering or the HPS algorithm (Section \ref{hps}), aims at providing physics objects as inclusively as possible, i.e. maximising the efficiency of capturing original genuine particles. This approach inherently creates a collection of physics objects which is not pure in the objects of interest and is contaminated by background objects. Therefore, an additional step is needed to refine the purity of the collection. 

This step is usually referred to as \textit{identification} (ID), and its goal is to identify the types of objects appearing in the collection of reconstructed objects among the categories which are expected. This two-staged \enquote{RECO-ID} paradigm of building physics objects has been a standard in high-energy physics for years. However, with the emergence of powerful ML techniques, novel end-to-end approaches unifying two steps into a single one proved to be a promising and efficient solution to the problem of reconstructing physics objects \cite{CMS:2022wjj,Pata:2021oez}. 

In the RECO-ID paradigm, ML-based algorithms have also proved to bring significant improvement to the ID step. The historical evolution pattern of ID methods is moving from a so-called \textit{cut-based} (or rule-based) set of criteria to algorithms based on \textit{linear classifiers} or an ensemble of \textit{decision trees} and then finally to algorithms based on \textit{Deep Learning} (DL). The hadronic decays of the tau lepton also fit into this historical pattern, where the RECO step with the HPS algorithm was initially followed by a set of isolation criteria targeting predefined misidentification probabilities of \tauh against quark/gluon jets \cite{cms2012performance}. Later on, algorithms based on boosted decision trees (BDT) were introduced \cite{CMS:2015pac, CMS:2018jrd} each trained to distinguish \tauh from either jets, or electrons, or muons. Lastly, a DeepTau algorithm \cite{CMS:2022prd} combined previously separate classifiers into a single neural network, providing an excellent discrimination power between \tauh, jets, electrons and muons altogether. 

The DeepTau architecture in its original implementation (Section \ref{deeptau1}), referred to as DeepTau v2.1, showed a significant improvement in \tauh identification against jets and leptons, compared to the previous approaches. Building upon this milestone, several improvements have been made in the context of the Run 3 preparation as outlined in Section \ref{deeptau5}, with the corresponding model being referred to as DeepTau v2.5. 

\subsection{DeepTau v2.1} \label{deeptau1}
There is one important aspect, in addition to the already mentioned unification of jet, electron and muon discriminants, which motivated a switch towards more advanced techniques for \tauh identification -- the usage of low-level information. While hand-crafted high-level variables (also called \textit{features}), provided as an input to a BDT, generally encapsulate the object to be identified, they are still limited in the representation power by the domain knowledge of the one who designed them. Since jets, being an input to a given model for \tauh identification, inherently exhibit complex hadronisation patterns, it is expected that their behavior cannot be fully described in terms of only several variables.

In the field of Computer Vision (CV) it has been shown that Convolutional Neural Networks (CNN) trained on images learn notions of growing complexity, starting from simple patters at the first layers and capturing more complex abstract concepts at the deeper layers \cite{olah2017feature}. Since pixels in the image carry only low-level intensity information, one can therefore view the process of training a CNN model as an \textit{automated feature engineering}: more complex features are automatically learnt based on the low-level inputs. Moreover, the performance of ML models has been shown to increase as the model size grows \cite{726791,NIPS2012_c399862d,simonyan2014very,szegedy2015going,he2016deep,huang2017densely,tan2019efficientnet}. That hints towards a large scope of high-level features which models can learn without explicit guidance. Furthermore, it is yet to be understood if (and how) it is possible to design such automatically learnt features manually.

One of the perspectives on a particle detector is when it is viewed as a \enquote{camera} imaging collisions. That makes it natural to use an \textit{image representation} to describe the physics objects and the activity in the detector \cite{Cogan:2014oua,deOliveira:2015xxd,Kagan:2020yrm}. Despite the fact that this representation comes with certain limitations (Section \ref{tat}), it has proved to be very performative in tasks like jet tagging \cite{Kasieczka:2019dbj,ATLAS:2017dfg,CMS:2020poo}, particle reconstruction and identification \cite{KM3NeT:2020zod,Collado:2020ehf,Collado:2020fwm,Abbasi:2021ryj}, and particle shower generation \cite{Paganini:2017dwg,Khattak:2021ndw,Buhmann:2021caf}.

\begin{figure}[!t]
    \centering
    \includegraphics[width=.75\textwidth]{Figures/Tau/deeptau_grids.pdf}
    \caption{The grid representation in $\eta$-$\phi$ plane passed as the input to the DeepTau model \cite{CMS:2022prd}. The signal cone ($R = 0.1$) and the isolation cone ($R = 0.5$) are also shown to motivate the choice of the inner and outer grids. The former enters the isolation requirement on the PF candidates within the HPS algorithm and has a higher cell granularity to better capture the hadronisation activity within the core of the \tauh candidate, especially in boosted scenarios. The latter enters in the computation of the high-level isolation variables used for the training and aims to capture the hadronisation activity at a larger scale to specifically identify quark and gluon jets.}
    \label{fig:deeptau_grid}
\end{figure}

All these aspects motivate the usage of convolution layers as the main building blocks of the DeepTau architecture and an image-like structure of the \tauh \textbf{input representation} (Fig. \ref{fig:deeptau_grid}). The latter is constructed by defining in $\eta$-$\phi$ space an inner grid with $11\times11$ cells of size $0.02\times0.02$, and an outer grid with $21\times21$ cells of size $0.05 \times 0.05$ (in the $\eta/\phi$ units of measurement). The grids overlap and are centered around the HPS-reconstructed direction of flight of the \tauh candidate (Section \ref{hps}). Seven types of particles in the vicinity of the \tauh-axis are taken as an input:

\begin{itemize}
    \item PF-reconstructed: muons, electrons, photons, charged hadrons, and neutral hadrons (Section \ref{pf}). 
    \item Standalone-reconstructed (RECO): electrons, muons.
\end{itemize}

The latter category uses dedicated standalone reconstruction algorithms which provide additional information about electrons and muons compared to those available from the PF algorithm. Each of the particles is attributed to a cell on both inner and outer grids according to its position in $\eta$-$\phi$ space, and the corresponding cell is filled with the features specific for a given particle type. Generally, the features describe the track quality, the quality of the associated PV or SV, the particle kinematics, the calorimeter and PU information. If several particles which are attributed to a grid of the given type \{inner, outer\} $\cross$ \{$e^\pm/\gamma$, $\mu^\pm$, $\text{h}^\pm/\text{h}^0$\} (described below) enter the same cell, the features of the one with the highest \pt are filled into the cell. 

In addition, high-level features are also provided as an input, as described below, to improve the discriminating power of the model. These handcrafted variables, describing the \tauh isolation, kinematic properties, associated vertex information, information about the associated strips, have been successfully used previously for \tauh identification with the MVA classifiers. Although in theory, the model should be able to learn these variables, in practise this is often not the case due to a limited number of training data. Therefore, these variables are added explicitly to augment the model with expert knowledge. 

The overall \textbf{architecture} is illustrated on Fig. \ref{fig:deeptau_v2p1_arch}. The model hyperparameters are described in detail in the original paper and below a conceptual overview of the model structure is provided. It starts from three streams, each processing its inputs independently: 
\begin{itemize}
    \item \textit{Global:} a set of fully connected layers which processes high-level features.
    \item \textit{Inner:} a set of one-dimensional (1D) followed by two-dimensional (2D) convolutional layers which processes inputs from the inner cone around the reconstructed tau direction of flight.
    \item \textit{Outer:} a set of 1D followed by 2D convolutional layers which processes inputs from the outer cone around the reconstructed tau direction of flight.
\end{itemize}

\begin{figure}[ht!]
    \centering
    \includegraphics[width=1.\textwidth]{Figures/Tau/deeptau_v2p1.pdf}
    \caption{The DeepTau v2.1 architecture. Three processing streams (inner, outer, global) are illustrated as arrows, three particle subsections ($e^\pm/\gamma$, $\mu^\pm$, $\text{h}^\pm/\text{h}^0$) per inner/outer streams are illustrated as rectangular blocks in white. Additionally, the detailed information about the number of input features per particle block/stream, the hyperparameter values, the evolution of the input tensor shape at various stages within the model as well as the number of trainable parameters (TP) for the different subsections of the model is provided.}
    \label{fig:deeptau_v2p1_arch}
\end{figure}
1D section of both inner and outer streams are further split into three subsections individually processing three particle blocks: 
\begin{itemize}
    \item $e^\pm/\gamma$: to process inputs from combined PF electrons, PF photons, RECO electrons.
    \item $\mu^\pm$: to process inputs from combined PF muons and RECO muons.
    \item $\text{h}^\pm/\text{h}^0$: to process inputs from combined PF charged hadrons and PF neutral hadrons.
\end{itemize}

After being processed individually, the three particle blocks are concatenated and passed to another set of 1D convolutional layers before being passed to a 2D section. In general, the idea of using 1D convolutions is to encode input features on the per-particle level into a more compact representation compared to the dimensionality of the input space. Otherwise, the  usage of 2D convolutional layers directly on the input feature space would make the training task computationally hard to perform. Lastly, for each of the inner and outer streams, the 1D section is followed by the 2D section, where $3\cross3$ filters extract spacial correlations between cells across the grid while also downsampling the spatial dimensions from $11x11$ ($21\cross21$) cells for the inner (outer) grid to a single cell (Fig. \ref{fig:deeptau_v2p1_arch}). 

Overall, the three-stream part of the DeepTau architecture can be viewed as an \textit{encoder}, which extracts high-level features from the low-level ones while operating on a physically-motivated representation of $\eta$-$\phi$ space. From this perspective, the following processing part of the architecture can be viewed as a \textit{decoder}, which maps the learned features for a given \tauh candidate to a class probability. First, it concatenates extracted features from the inner and outer streams with the handcrafted high-level features after being processed in the global stream. Second, it processes them by a set of fully-connected layers, finally followed by a fully-connected layer with four output nodes with a softmax activation function. The latter outputs the probability $y_\alpha$ of the given \tauh candidate to belong to one of the four classes: electron, muon, genuine \tauh, quark or gluon jet. 

In order to measure the model performance and also derive working points (WPs), the final discriminators against electrons, muons and jets are defined as:
\begin{equation} \label{eq:d_alpha}
    D_\alpha(\bm{y}) = \dfrac{y_\tau}{y_\tau + y_\alpha},
\end{equation}
where $\bm{y} = (y_e, y_\mu, y_\tau, y_\text{jet})$ is the output of the softmax layer of the model.

To perform the training, a \textbf{loss function} is constructed and minimized with Nesterov-accelerated adaptive momentum estimation (NAdam) \cite{dozat2016incorporating}. The loss function consists of three terms (Appendix \ref{app:loss}):
\begin{enumerate}
    \item Binary cross-entropy term for \tauh class against all the other $(e, \mu, \text{jet})$ classes combined.
    \item Focal-loss \cite{lin2017focal} term for \tauh class against all other classes combined.
    \item Focal-loss terms separately for each of the $(e, \mu, \text{jet})$ classes, smoothed by a step function to target only \tauh candidates which are likely to be classified as \tauh.
\end{enumerate}
The composition of the loss function is designed to guide the training to have better performance in the regions which are important for most of the analyses. Namely, it aims to provide better performance in the range 50-80\% of \tauh efficiency, while on the other hand to not focus on the identification of background classes in the high-purity regime.   

The \textbf{data set} used for the training consists of events from the following simulated processes: Z+jets (NLO), W+jets, \ttbar, $Z^{'} \to \tau \tau$, $Z^{'} \to ee$, $Z^{'} \to \mu \mu$, (with $m(Z^{'})$ ranging from 1 to 5 TeV),and QCD multijet production. For testing, additional event samples from $\text{H} \to \tau\tau$ and Z+jets (LO) are used. To ensure that no additional biases are introduced, the \tauh candidates are sampled from the input samples such that the contribution of each class ($e$, $\mu$, \tauh, jet) in different (\pt, $\eta$) bins is the same. Furthermore, during the training additional weights are applied to make the distribution of classes uniform within each (\pt, $\eta$) bin. In total, around 140 million \tauh candidates are used for the training, while around 10 million are used for the validation. The model implementation and the training are done using the TensorFlow library \cite{tensorflow2015-whitepaper}. 

Overall, large gains in the \textbf{performance} are reported across various regions of the phase space with respect to the previous cut-based/tree-based \tauh identification approaches (Fig. \ref{fig:deeptau_v2p1_performance}). In summary, at a given \tauh efficiency, the DeepTau discriminator consistently reduces the misidentification probability against jets by more then a factor of 2. Against electrons, the improvement in misidentification probability ranges from by a factor of 2 for a \tauh efficiency of 70\% and goes up to a factor of 10 for \tauh efficiencies larger than 88\%. Against muons, the misidentification probability is reduced by almost a factor of 10 in the region of \tauh efficiency around 99\%. This expected performance improvement, obtained from the simulated events, was successfully validated on collision data, therefore establishing a new milestone for the tau identification task. 

\begin{figure}[t!]
    \centering
    \includegraphics[width=0.32\textwidth]{Figures/Tau/deeptau_v2p1_vs_jet.pdf}
    \includegraphics[width=0.32\textwidth]{Figures/Tau/deeptau_v2p1_vs_e.pdf}
    \includegraphics[width=0.32\textwidth]{Figures/Tau/deeptau_v2p1_vs_mu.pdf}
    \caption{Efficiency for jets (left), electrons (middle), and muons (right) versus efficiency for genuine \tauh to pass various tau identification discriminators as well as the corresponding $D_\alpha$ discriminators (Eq. \ref{eq:d_alpha}). \tauh are selected with the requirements $20 < \pt < 100$ GeV, $|\eta| < 2.3$, and should additionally pass the loosest WPs against other tau types, with the thresholds as defined in the original paper. For evaluation, genuine \tauh are taken from the $H \to \tau\tau$ event sample, electrons and muons are taken from the $Z \to ll$ event sample, and jets are taken from the \ttbar event sample.}
    \label{fig:deeptau_v2p1_performance}
\end{figure}

\subsection{DeepTau v2.5} \label{deeptau5}
As the LHC and the CMS detector in particular are continuously being upgraded and are moving on with collecting new data, every particle identification model deployed in production largely benefits from retraining on the data corresponding to the new detector conditions. It ensures that the quality of the model does not degrade due to the lack of robustness to the gradual changes in the underlying data (so-called \textit{data drift}) and the model performance stays at the nominal level.

This applies also to DeepTau v2.1 (Sec. \ref{deeptau1}), which is originally trained on the data samples simulated and reconstructed with the 2017 data-taking conditions. With the start of the new Run 3 period, it is expected that the model performance will become suboptimal on the newly collected data as well as on the Run2 Ultra Legacy (UL) reprocessed data. Therefore, a dedicated DeepTau retraining has been performed with the following set of updates, further described in more details: 

\begin{itemize}
    \item Data preparation:
    \begin{itemize}
        \item Updated data samples,
        \item Shuffle \& Merge procedure,
        \item Feature preprocessing.
    \end{itemize}
    \item Training:
    \begin{itemize}
        \item Parallel data loading,
        \item Hyperparameter optimisation,
        \item Adversarial training.
    \end{itemize}
\end{itemize}

\subsubsection{Data preparation}
The global tau identification task remains unchanged w.r.t. DeepTau v2.1 in terms of the input and output spaces: for a given HPS-reconstructed \tauh candidate the goal is to predict its probability to be either of (jet, $e$, $\mu$, \tauh) classes, with the class assignment criteria described below. The \tauh candidate is represented with the surrounding PF and RECO electron/muon candidates placed onto inner and outer grids in $\eta$-$\phi$ space centered around \tauh axis, with the grid parameters being the same as defined in Sec. \ref{deeptau1}. The set of input features describing particle collections is the same as it was used for the DeepTau v2.1 training.

In order to compose the training data set, \tauh candidates of the given classes are sourced from the following data samples, simulated under the detector conditions corresponding to the 2018 year:
\begin{itemize} \label{v2.5:datagroups}
    \item DY: inclusive, jet/$H_\text{T}$-binned $\Rightarrow$ jet, $e$, $\mu$, \tauh.
    \item QCD: \pt/$H_\text{T}$-binned $\Rightarrow$ jet.
    \item \ttbar: leptonic, semileptonic $\Rightarrow$ jet, $e$, $\mu$, \tauh.
    \item \ttbar: fully-hadronic $\Rightarrow$ jet, $e$, $\mu$.
    \item W+jets: jet/$H_\text{T}$-binned $\Rightarrow$ jet, $e$, $\mu$, \tauh.
    \item Higgs: $Z \text{H}\to \tau\tau$, $W^\pm \text{H}\to \tau\tau$, $\text{H}\to \tau\tau$ (vector-boson fusion), $\text{H}\text{H}\to bb\tau\tau$ (gluon-gluon fusion) $\Rightarrow$ \tauh.
    \item $Z' \to e^+e^-$: $m(Z') \in [1000, 4000]$ GeV $\Rightarrow$ $e$.  
    \item TauGun: $\pt(\tau) \in [15, 3000]$ GeV $\Rightarrow$ \tauh.
\end{itemize}

A \tauh candidate is selected and assigned to one of the classes based on the following criteria:
\begin{itemize}
    \item $e (\mu)$: matching to a prompt $e (\mu)$ or $\taue (\taum)$ at the generator level within a cone of radius $R=0.2$. The associated generated lepton should pass the requirement $\pt^{\text{vis}}(l)>8$ GeV.
    \item \tauh: matching to a hadronically decaying tau lepton at the generator level within a cone of radius $R=0.2$. The associated generated lepton should pass the requirement $\pt^{\text{vis}}(l)>15$ GeV.
    \item jet: absence of an associated generator-level lepton and matching to a generator-level jet within a cone of radius $R=0.4$. Additionally, in the selection procedure jets with the reconstructed $\pt < 80$ GeV are randomly rejected with the probability $p = 1 - \exp{-0.05\cdot(80-\pt)}$ in order to balance the contributions from low- and high-\pt ranges in the training data set.
\end{itemize}

After the initial selection of \tauh candidates from the data samples, a training data set has to be formed as a set of \tauh candidates grouped into batches, later fed into the model. In order to have a stable training procedure, it is beneficial for batches in the data set to have the following properties:
\begin{itemize}
    \item Homogeneous: every two randomly selected batches are statistically similar. 
    \item Unbiased: every batch should be sampled in an unbiased way from the predefined target distribution $p_\text{target}(x_s)$ over a set of \textit{spectrum variables} $x_s$. In the case of DeepTau v2.5, these are selected to be: $x_s = \{\pt(\tauh), \eta(\tauh), \text{class}(e, \mu, \tauh, \text{jet})\}$. The target distribution is chosen to be a uniform histogram (referred to as a spectral histogram) in a predefined binning for the spectrum variables. 
\end{itemize}

The following procedure, referred to as \textbf{Shuffle \& Merge} (S\&M), is used to form the training data set satisfying these properties. Performed in a memory-efficient manner, it allows for a better control over the spectrum variables compared to the balancing approach used for the DeepTau v2.1 training. It follows a stochastic approach, where a data group is firstly sampled from the categorical distribution, where the categories are the groups of the data samples used for the training as listed above, and the probability of sampling from a given data group is proportional to the number of entries in the group. Secondly, a random \tauh candidate is sampled from the data group, and it is kept for the training with the probability:
\begin{equation}
    p(\tauh) \sim \dfrac{p_\text{target}(x_s(\tau_h))}{N(\text{bin},\text{group})}.
\end{equation}

Here, $x_s(\tau_h)$ are the values of the spectrum variables for the given \tauh candidate, $N(\text{bin},\text{group})$ is the total number of events in a bin of the spectral histogram corresponding to the given $x_s(\tau_h)$ value and the data group. An additional correction to the probabilities in the spectrum histograms is made to keep the ratio between the number of \tauh candidates in the last \pt bin and the other \pt bins per each class more than 0.001. 

The procedure is distributed in a parallel manner and for a given thread a random subset of \tauh candidates of each of the data group is provided. The condition to terminate the procedure for a thread is when there is no more \tauh to select from one of the data groups. Upon the completion of the S\&M procedure, a Kolmogorov-Smirnov test is performed in order to validate the compatibility of randomly selected subsets of the formed data set between each other. Overall, an acceptable level of homogeneity is observed across the spectrum variables for each of the data group. The final data set comprised of around 100M \tauh candidates, with 70\% of them used for the training and 30\% for the validation. This data set is further referred to as a \textit{S\&M data set}.  

The last step before the training is the \textbf{feature standardisation}. Its goal is to bring the values of the input features to a common domain, which is achieved by subtracting the mean and dividing by the standard deviation, followed by a clamping to a range $[-5,5]$. Categorical features and features derived from $\pt, \eta, \phi$ are normalised by clamping to a predefined range, followed by a mapping to a range $[-1,1]$. The clamping procedure also removes outlying values, which together with the feature standardisation provides stable gradient updates during the training. The mean and the standard deviation for each of the features are derived in a cumulative manner by aggregating the sums and the counts of the feature values over the input data set. The validation is performed to check that the clamping procedure does not distort the original distributions of the input features.

\subsubsection{Training}

After the training data set is formed, for each \tauh candidate an image representation is constructed as described in Sec. \ref{deeptau1}. The corresponding tensors are combined into batches which are subsequently fed into the model during the training. Since the batch shaping procedure is performed on the fly, there is a challenge of how to make the data loading procedure time-efficient during the gradient updates. This is solved by introducing a multiprocessing queue and a set of workers filling/taking batches to/from the queue, thus allowing for a concurrent loading of batches into the model as the training is on-going. A PyTorch \cite{NEURIPS2019_9015} implementation of the multiprocessing queue is used as providing a factor of 3 speed-up compared to a default implementation in Python libraries. Moreover, a better scaling of computational performance with the number of workers and the queue size is observed for the PyTorch implementation. 

The \textbf{hyperparameter optimisation} is performed in several stages in order to select the model with the most optimal performance. For the first stage (stage 0), the training of each of the trials is performed on 20\% of the S\&M data set for one epoch, which takes approximately one day on NVIDIA$^\text{TM}$ Tesla V100. For validation, another 20\% of the S\&M data set is used. The same loss function is used as for the DeepTau v2.1 training (Appendix \ref{app:loss}) and it is minimised with a NAdam optimiser with the following parameters: learning rate = $10^{-3}, \, \beta_1 = 0.9, \, \beta_2 = 0.999, \,  \epsilon=10^{-7}$. Training weights, derived from the spectral histograms of the S\&M data set, are added to the loss function in order to make the contributions from different $(p_\text{T}, \eta, \text{class})$ bins uniform. The hyperparamater values varied during the first stage of the optimisation as well as the resulting performance are summarised in Table \ref{tab:v2p5_stage0}. The trial names correspond to the following configurations of the hyperparameters (changes are with respect to the baseline):

\begin{itemize}
    \item Baseline: DeepTau v2.1 architecture (Sec. \ref{deeptau1}).
    \item (1): number of 2D filters is reduced by 1.8 from one layer to another (*) + number of nodes in the decoder's dense layers form a progression from n/2 to 32 with reduction factor 2, where n is the input dimensionality to the decoder (**). 
    \item (2): (*) + (**) + number of filters in 1D convolutions operating on the merged $e^\pm/\gamma$, $\mu^\pm$, and $\mathrm{h}^\pm/\mathrm{h}^0$ streams forms a progression: $227 \to 141 \to 128$.
    \item (3): (*) + (**) + number of filters in 1D convolutions operating on the merged $e^\pm/\gamma$, $\mu^\pm$, and $\mathrm{h}^\pm/\mathrm{h}^0$ streams forms a progression: $227 \to 141 \to 88 \to 64 \to 40 \to 32$.
    \item (4): (**) + number of 2D filters in each layer is reduced by 2 from one layer to another.
    \item (5): (**) + number of 2D filters in each layer is reduced by 1.6 from one layer to another.
    \item (6): (*) + number of nodes in the decoder's dense layers forms a progression from n/3 to 32 with the reduction factor 3. 
    \item (7): (*) + number of nodes in the decoder's dense layers forms a progression from n/1.5 to 32 with the reduction factor 1.5. 
\end{itemize} 
\begin{table}[h]
	\caption{Performance comparison for the models from the stage 0 of the DeepTau v2.5 hyperparameter optimisation. The changes to the hyperparameters in the trials are described in the text and the number of trainable parameters (TPs) is shown in the middle column. Each trial is run 5 times and the mean and standard deviation values of the loss function on the validation set are reported in the right column.}
    \centering
	\begin{tabular}{c|c|c}
		Trial & TPs $\cross 10^{3}$  & $L_\text{val}$\\
		\hline
		Baseline & 1,151 & $\mathbf{0.273 \pm 0.002}$\\
        (1) & 1,197  & $0.277 \pm 0.003$\\
        (2)  & 3,211 & $0.274 \pm 0.002$ \\
        (3)  & 653 & $0.287	\pm 0.008$ \\
        (4)  & 768 & $0.280 \pm 0.002$ \\
        (5)  & 2,570 & $0.287 \pm 0.019$ \\ 
        (6)  & 1,097 & $0.276 \pm 0.004$ \\
        (7)  & 1,383 & $0.324 \pm 0.080$ \\
	\end{tabular} \label{tab:v2p5_stage0}
\end{table}

Overall, the architecture with the hyperparameter values of the DeepTau v2.1 model gives the best performance if measured by the value of the loss function on the validation data set averaged across 5 independent trials.

While the first stage involves the variations of the network structure, the second stage (stage 1) targets the choice of the optimiser and the learning rate. Furthermore, since at this point 70\% (30\%) of the S\&M data set is used for the training (validation), the goal is to reach better convergence compared to the short training of the first stage. The best performing model from the first stage is therefore taken and the training is continued for additional two epochs with the following optimisers and learning rates values being probed:
\begin{itemize}
    \item NAdam: $10^{-3}$ \textcolor{gray}{(acbad)}, $10^{-4}$ \textcolor{gray}{(5371f)}, $10^{-5}$ \textcolor{gray}{(08f84)}, $10^{-3}$ w/ cross-entropy \textcolor{gray}{(6945b)}.
    \item Adam \cite{kingma2014adam}: $10^{-3}$ \textcolor{gray}{(38c05)}, $10^{-4}$ \textcolor{gray}{(59162)}.
\end{itemize}

where a unique hash value for the corresponding entry in the legend of Fig. \ref{fig:v2p5_HO_stage1_performance} is specified in brackets. For all of the trials, the loss function from the first stage is used in the minimisation, except for the trial NAdam ($10^{-3}$), where the loss function consists only of the weighted sum of binary cross-entropy terms for \tauh versus the other classes.

To evaluate the performance at the second stage, receiver operating characteristic (ROC) curves are derived for each of the discriminators (Eq. \ref{eq:d_alpha}) in bins of \pt, $|\eta|$ and HPS-reconstructed decay mode of the \tauh candidate. For evaluation, genuine \tauh candidates are sourced from the $\text{H} \to \tau\tau$ (gluon-gluon fusion production mode) data sample. Jets are sourced from the fraction of the semileptonic \ttbar data sample not used in the training. Electrons and muons are sourced from the fraction of the DY data sample not used in the training. The results of the evaluation are shown on Fig. \ref{fig:v2p5_HO_stage1_performance} for the region $p_\text{T}(\tauh) \in [20,100), |\eta(\tauh)| < 2.3, \text{DM}(\tauh) \in \{0,1,10,11\}$. While Adam ($10^{-3}$) performs the best in the classification against muons, it shows inferior performance compared to the other models in the classification against electrons (low tau efficiency region) and against jets. Likewise, the model used as the starting one for the second stage tuning (referred to as \enquote{default} on Fig. \ref{fig:v2p5_HO_stage1_performance}), shows the best performance against jets but performs 10-20\% worse against electrons and muons. As a trade-off, the model which corresponds to the trial NAdam ($10^{-4}$) is chosen at this stage as the one compromising the performance against all of the three classes.

\begin{figure}[!t]
    \centering
    \includegraphics[width=0.32\textwidth]{Figures/Tau/v2p5_HO_st1_jet.png}
    \includegraphics[width=0.32\textwidth]{Figures/Tau/v2p5_HO_st1_e.png}
    \includegraphics[width=0.32\textwidth]{Figures/Tau/v2p5_HO_st1_mu.png}
    \caption{Efficiency for jets (left), electrons (middle), and muons (right) versus efficiency for genuine \tauh to pass the corresponding $D_\alpha$ discriminators for each of the trials at the second stage (stage 1) of the hyperparameter optimisation. \enquote{Default} is the model used as a starting model for all of the trials. The panel at the bottom of each of the figures shows the ratio of the ROC curves evaluated for each of the trial to the ROC curve evaluated for the starting model.}
    \label{fig:v2p5_HO_stage1_performance}
\end{figure}

The last stage (stage 2) in the training pipeline introduces an \textbf{adversarial approach} to fine-tuning the model. The motivation for that is to eliminate the discrepancies between data and simulation observed in the high-score region of $D_{\text{jet}}$ for the model after the second stage of the hyperparameter optimisation (Fig. \ref{fig:v2p5_data_mc}, left). This can be viewed as a \textit{domain shift} problem, meaning that the model performance does not transfer from the source domain on which it was trained (simulated data) to the target domain on which it is eventually being applied (collision data).

\begin{figure}[!t]
    \centering
    \includegraphics[width=0.48\textwidth]{Figures/Tau/v2p5_data_mc_stage2.png}
    \includegraphics[width=0.48\textwidth]{Figures/Tau/v2p5_data_mc_stage3.png}
    \caption{Comparison of data and MC simulation agreement for $D_{\text{jet}}$ variable in the adversarial region as defined in the text for DeepTau v2.5 at stage 1 (left) and DeepTau v2.5 (stage 2)}
    \label{fig:v2p5_data_mc}
\end{figure}

Approaching this problem from the perspective of domain adaption \cite{wang2018deep}, one of the techniques to align the model's performance between the source and target domains is to augment the loss function with an additional adversarial term. Initially proposed in \cite{ganin2015unsupervised, Louppe:2016ylz}, the technique showed an improved modelling of displaced jets in a search for new long-lived particles in CMS \cite{CMS:2019dqq} without significant decrease in performance.  

To achieve this goal, the DeepTau v2.5 (stage 1) model is extended to have an additional stream in parallel to the decoder (Fig. \ref{fig:deeptau_v2p5_arch}). This branch is responsible for producing predictions of whether a given \tauh candidate originates from a collider data event or from a simulated data event. Conceptually, the idea is to continue training an augmented version of the model in a way that the decoder stream still tries to classify between the four classes as good as possible, while the adversarial stream tries to predict the underlying \tauh domain as bad as possible. 

\begin{figure}[!t]
    \centering
    \includegraphics[width=\textwidth]{Figures/Tau/deeptau_v2p5.png}
    \caption{The DeepTau v2.5 architecture. The core encoder and decoder parts along with their hyperparameters are the same as in DeepTau v2.1. The key difference is an addition of another adversarial stream in parallel to the decoder which computes the probability of a \tauh candidate to originate from either collider or simulated data.}
    \label{fig:deeptau_v2p5_arch}
\end{figure}

This intuition is implemented by adding a binary cross-entropy over two classes (collider data vs. simulated data) $L_\text{adv}$ as an additional negative term to the classification loss function $L_\text{class}$ to penalise correct identification of the data domain:
\begin{equation}
    L_\text{tot} = k_1 \cdot L_\text{class} - k_2 \cdot L_\text{adv},
\end{equation}\label{eq:adv_loss}

where $k_1, k_2 > 0$ are the hyperparameters to trade off between adversarial regularisation and classification performance. For the final model, $k_1=1$ and $k_2=10$ are chosen and each of the loss components receives its own Adam optimiser (for the classification term, the optimiser inherits its state from the original model). The initial learning rates are $0.001$ for the classification component and $0.01$ for the adversarial component and are decayed exponentially throughout the training. Lastly, the relative class importance constants in the classification loss (Appendix \ref{app:loss}) are modified: $[\kappa_e, \kappa_\mu, \kappa_\tau, \kappa_j] = \frac{4}{10}[1, 2.5, 5, 1.5] \to \frac{4}{14}[2, 5, 6, 1.]$.  

In order to perform fine-tuning with the adversarial component, an adversarial data set is formed, consisting of an equal number of collider data and simulated events. The former are taken from 2018 collider data, while the latter are sourced from Drell-Yan, \ttbar, QCD, W+jets simulated samples with the same detector conditions. The events are required to pass the selection requirements of \mt channel used in this work (Sec. \ref{sec:mt}) with an additional requirement $D_\text{jet}(\text{v2.1}) > 0.9$, where the discriminator is taken from the DeepTau v2.1 model. The region corresponding to this selection is referred to as an adversarial region. The resulting data set comprises 1.9k batches of 100 \tauh candidates (50 from collider data, 50 from simulation).

The training step firstly proceeds with passing a batch from the S\&M data set, used at the previous stages of training, and computing the gradients for the classification loss $L_\text{class}$ with respect to the encoder and decoder weights. Then, a batch from the adversarial data set is passed and the gradients for the adversarial loss $L_\text{adv}$ with respect to the encoder and the adversarial stream weights are computed. Next, the weights are updated with the computed gradients for each of the model parts according to Eq. \ref{eq:adv_loss} in order: decoder $\to$ adversarial stream $\to$ encoder. 

The training proceeds until the convergence of the loss function (\ref{eq:adv_loss}) on the validation data set, with the final validation accuracy of data vs. MC classification task equal to 0.51. This indicates that the model reaches sufficient level of not being able to distinguish between the two domains. Moreover, the agreement of data with simulation improves significantly in the adversarial region for the stage 2 model, compared to the stage 1 model before adversarial fine-tuning (Fig. \ref{fig:v2p5_data_mc}, right). 

\begin{figure}[t!]
    \centering
    \includegraphics[width=0.32\textwidth]{Figures/Tau/v2p5_HO_st2_jet.png}
    \includegraphics[width=0.32\textwidth]{Figures/Tau/v2p5_HO_st2_e.png}
    \includegraphics[width=0.32\textwidth]{Figures/Tau/v2p5_HO_st2_mu.png}
    \caption{Efficiency for jets (left), electrons (middle), and muons (right) versus efficiency for genuine \tauh to pass the corresponding $D_\alpha$ discriminators at the third stage (stage 2) of the hyperparameter optimisation. Three models are compared: DeepTau v2.1 (red), DeepTau v2.5 at stage 1 (green), DeepTau v2.5 at stage 2 (blue). Solid lines correspond to the performance as measured on the samples with Run 2 detector conditions which are used in the training. Dashed lines correspond to the samples with Run 3 detector conditions. The panel at the bottom of each of the figures shows the ratio of the ROC curves evaluated for each of the trial to the ROC curve evaluated for the DeepTau v2.5 at stage 1 on Run 2 samples.}
    \label{fig:v2p5_HO_stage2_performance}
\end{figure}

The model performance at stage 2 is evaluated analogously to stage 1. In addition to the simulated samples with the detector conditions of the 2018 year (referred to as Run 2 performance), the model is evaluated on Run 3 samples with the detector conditions corresponding to the early Run 3 data taking period. Electrons, muons and genuine \tauh candidates are sourced from DY sample, while jets are sourced from a \ttbar (semileptonic) sample. The resulting ROC curves evaluated in the region $p_\text{T}(\tauh) \in [20,100), |\eta(\tauh)| < 2.3, \text{DM}(\tauh) \in \{0,1,10,11\}$ are shown on Fig. \ref{fig:v2p5_HO_stage2_performance}. The performance of the model at stage 1 as well as those of DeepTau v2.1 is also shown. No significant degradation in performance is observed between the model performance at stage 1 and stage 2. The agreement between data and simulation is improved by 5-15\% as can be seen from the data/simulation ratio in the two highest score bins (Fig. \ref{fig:v2p5_data_mc}). This indicates the effectiveness of the procedure and motivates its generalisation to the discriminators against electrons and muons. However, it can be seen that the performance generally does degrade if the model is applied on the early Run 3 samples on which it was not trained. This motivates a dedicated training or fine-tuning of the model on those samples to keep the performance at the nominal level. Overall, the final DeepTau v2.5 model delivers a reduced fake rate at a given \tauh efficiency by 10-50\% across the regions of interest and sets a new improved baseline for the tau identification task.

\subsection{Tau Transformer} \label{tat}
As mentioned in Sec. \ref{deeptau1}, an image representation and a traditional convolutional approach to process it come with certain limitations. Despite yielding good results across various research domains, for the specific task of the \tauh identification there is a set of design issues:
\begin{itemize}
    \item \textbf{Information is not represented compactly.} On average, the DeepTau grid is filled with zeros in 90\% (99\%) cases for the outer (inner) grids which makes the data loading procedure not memory-efficient. The only information stored in empty cell is an implicit positional one which can be passed to the model in a more efficient way.
    \item \textbf{Convolutional layers might not encode information optimally.} Since there is no explicit communication of positional/relational information to the model, one relies on the model as capable to optimally learn the relationships between particles in the spatial 2D frame. This might happen in the limit of infinite data, but in practise, other approaches to encode information can yield better results on the limited data sets.
    \item \textbf{Translational equivariance is not applicable.} The key feature of 2D convolutional layers is that they produce representations which are equivariant with respect to translational shifts. This is not applicable to the tau identification domain, where a symmetry breaking is induced with the choice of the image centering axis (HPS-reconstructed direction of flight) and furthermore with the topology of the CMS detector. Potentially, it may result in undesired behaviour with respect to small spatial perturbations of inputs. 
    \item \textbf{Scaling with the number of PU interactions is limited.} Because of the finite cell size, in case of several particles entering the same cell only the one with the highest value of \pt is kept. On the one hand, it might provide a natural regularisation and improved robustness to the increased number of PU interactions. On the other hand, it assumes that the particles with higher \pt are more important for the \tauh identification, which might not be the case and therefore a significant loss of information can take place.  
\end{itemize}

To put the image representation into the context, there are ongoing studies of various representations in the particle physics area. Historically, representing activity in the detector as an image was one the first ideas together with a classical approach of using handcrafted features dating back to \cite{DENBY1988429}. Then, partially due to the reasons described above, sparse representations were becoming more prominent. These include sets \cite{Komiske:2018cqr}, sequences \cite{deLima:2021fwm}, graphs \cite{Thais:2022iok}, and polynomials \cite{Munoz:2022gjq} with each of these representations coming up with its own way to extract information. Furthermore, physics-motivated representations in particular aiming to preserve the underlying symmetries were also proposed and studied \cite{Dreyer:2020brq, Baldi:2022okj, Bogatskiy:2022hub, Shimmin:2021pkm}.

\subsubsection{Architecture}
In this work, a \tauh candidate is represented as a set of input particles. Taking inspiration from a Natural Language Processing (NLP) domain, this representation is tightly linked with a sentence-based perspective, where a \tauh candidate, being a set of particles (referred to as constituents or tokens), is viewed as a sentence consisting of multiple words with underlying grammar rules (for the \tauh case, decay history) which are not observed directly. In order to extract information from this representation, a self-attention mechanism is used as proposed in the original paper \cite{vaswani2017attention} introducing a Transformer model. The proposed concept of attention kick-started a revolution in the ML field due its dramatic improvement in the performance across multiple domains and due to its excellent scalability with the size of the input data set \cite{phuong2022formal}. 

In HEP domain, models built around various implementations of attention also showed noticeable improvement \cite{Mikuni:2020wpr, Mikuni:2021pou}. The most recent Particle Transformer (ParT) \cite{Qu:2022mxj} model builds upon the original Transformer model and augments it with an interaction mechanism. Notably, the importance of using larger data sets for training in the jet tagging domain is additionally emphasized.

However, the unique feature of the \tauh identification task is the heterogeneity of the input space. Transformer models in the particle physics domain so far assumed that the inputs constitute only particles of one specific kind. Furthermore, no specific treatment of global variables is proposed. This is to be contrasted with the four collections used in the DeepTau v2.1 and v2.5 training to describe a \tauh candidate: global variables, PF-reconstructed particles, RECO electrons and muons. This can be viewed from a \textit{multimodality} perspective, where an input object is described by several various modalities which cannot be \textit{a priori} combined into a single one. The notion of multimodality is strictly speaking not fully applicable to the \tauh case, since RECO electrons/muons and PF constituents are both particles in their essence. However, they have various input features, which makes their treatment as of the same kind not straight-away possible. Furthermore, global variables certainly stand out from the \enquote{particle} modality. A multimodality perspective therefore provides a convenient language to describe the input \tauh representation, also for future studies where additional collections (for example, secondary vertices or tracker hits) can be included into the input representation to make it more informative.  

A Particle Embedding module is therefore introduced (Fig. \ref{fig:tat_embedding}) in order to unify the four collections together into a single representation. The idea is to bring the dimensionality of tokens of each of the modality to a common one and combine the modalities before propagating them to attention layers. It is achieved by firstly using a categorical embedding of an additionally introduced modality variable (one-dimensional, categorical variable) with $N_\text{in}=10$ input values (7 PF types: $e$, $\mu$, $\gamma$, $\text{h}^\pm$, $\text{h}^0$, HF tower identified as an hadron, HF tower identified as an EM particle; 1 RECO muon; 1 RECO electron; 1 global variables) to an output dimensionality $d_\text{cat}=2$ (two-dimensional, real valued). Performed on a per token basis (embedding matrix is shared across the modalities), the result is concatenated to the other features and is processed by embedding blocks. Each embedding block is defined separately for each of the modalities and consists of two feed-forward layers with dimensionalities (number of output nodes) $d_\text{ff}=256$ and $d_\text{model}=64$ (Fig. \ref{fig:tat_embedding}), also operating on the per token basis. Afterwards, all the embedded tokens of all the modalities are concatenated together per \tauh candidate and passed through a dropout layer \cite{JMLR:v15:srivastava14a} with $p=0.1$ to self-attention layers.

It is worth mentioning that in this approach global variables are treated as a \enquote{context} token: it is allowed to interact with the other constituents (both PF and RECO particles) on the equal basis as constituents interact with each other. This interaction is guided via attention mechanism and is learned by the model during the optimisation procedure. However, while the proposed approach of combining modalities is straight-forward in its intuition, it might not be optimally representing the underlying relationships between them \cite{xu2022multimodal}. This question of optimally encoding multiple modalities in the context of jet tagging is left for future research.   

\begin{figure}[t!]
    \centering
    \includegraphics[width=\textwidth]{Figures/Tau/tat_embedding.pdf}
    \caption{A Particle Embedding block. Processing streams for RECO electrons and RECO muons are illustrated together for visualisation purposes. For a single \tauh candidate each modality is represented as a tensor of the shape $[\text{N}(*), \text{N}_\text{f}(*)]$ and the change of the shape throughout the block is shown next to the arrows. Three tokens illustrating a vector of features are shown with the grey, green, brown colors corresponding to a PF constituent, a RECO electron/muon candidate and global variables, respectively. A dotted cell represents an additionally introduced modality variable. The values of the hyperparameters ($d_\text{cat}$, $d_\text{ff}$, $d_\text{model}$) are described in the text.}
    \label{fig:tat_embedding}
\end{figure}

After the particle embedding, each \tauh candidate is represented as a tensor of the shape $[\text{N}(\text{PF})+\text{N}(e)+\text{N}(\mu)+1, \text{d}_\text{model}]$. This structure can now be processed with the encoder, which consists of the $\text{N}_l=6$ self-attention layers having the same structure of two sub-layers as in the original Transformer paper. The first sub-layer consists of a multi-head attention block with $\text{N}_\text{h}=8$ heads of dimensionality $\text{d}_\text{head}=8$ and a layer normalisation. The second sub-layer consists of two feed-forward layers with the dimensionalities $\text{d}_\text{ff, 1}=256$ and $\text{d}_\text{ff, 2}=\text{d}_\text{model}=64$. Residual connections are employed after each of the sublayers, followed by the dropout layer with $p=0.1$ and the layer normalisation. After the encoding layers, the learned embeddings are globally pooled by summing the embedding values across the token axis. Lastly, the decoder part proceeeds with $\text{N}_\text{ff}=3$ feed-forward layers with the decreasing dimensionality $256 \to 128 \to 4$ followed by the Softmax layer.   

The resulting model is referred to as Tau Transformer (TaT) and its architecture is illustrated on Fig. \ref{fig:models} (top) along with the ParticleNet \cite{Qu:2019gqs} architecture (bottom). The ParticleNet is chosen for a benchmark on the \tauh identification task as being one of the most prominent models at the moment, showing an improvement on the jet tagging tasks and successfully used in several physical analyses \cite{CMS:2022psv, CMS:2022nmn}. Its implementation follows the original paper with the exception of being adapted to a multimodality nature of the task. The modalities are embedded with the same Particle Embedding block as for TaT, with the only exception of removing the global token. The reason for that is an inherent limitation of the ParticleNet model: it assumes that the inputs to the first layer are placed in some coordinate system where the distance can be computed to define a k-nearest neighbor (kNN) graph. Since this is not applicable to the global token, it is concatenated with the learned embeddings in the encoder and further processed within the decoder. The encoder in ParticleNet consists of $\text{N}_l=3$ EdgeConv layers as proposed in the original paper with a decreasing number of the nearest neighbours $k = 16 \to 12 \to 8$, the number of channels for each of the layers $C = (160, 128, 96)$ and feature aggregation via averaging. The kNN graph in the first EdgeConv layer is constructed in the $\eta-\phi$ plane around the \tauh direction of flight. After the encoder, the global pooling of embeddings is performed via summing across the constituents axis. It is followed by the decoder consisting $\text{N}_\text{ff}=6$ of Dense layers with the decreasing dimensionality $192 \to 160 \to 128 \to 96 \to 64 \to 4$ and interleaved with the dropout layers ($p=0.1$). Overall, TaT has 416k (embedding + encoder) + 50k (decoder) = 466k , while ParticleNet has 298k (embedding + encoder) + 98k (decoder) of trainable parameters. This is to be compared with the 1314k of trainable paramerers for DeepTau v2.5. 

\begin{figure}[ht!]
    \centering
    \includegraphics[width=\textwidth]{Figures/Tau/models.pdf}
    \caption{Tau Transformer (top) and ParticleNet (bottom) architectures as used for the \tauh identification task in this work. The hyperparameters and the structure of the layers are described in the text.}
    \label{fig:models}
\end{figure}

\subsubsection{Training setup} 

For training, 10\% of the S\&M data set (Sec. \ref{deeptau5}) is used, while another 10\% of the S\&M data set is used for validation. An extended set of features compared to DeepTau v2.5 (Sec.\ref{deeptau5}) is used to describe the input collections. For the PF candidates, additional information about the number of hits and layers in the tracker system and HCAL energy deposits is included. For the RECO electrons, a set of variables is extended with those describing a shower shape in ECAL. For the global variables, information about a relative displacement of secondary vertex from a primary vertex is included if available.  Furthermore, a collection of PF types is extended by adding the towers in the forward calorimeter reconstructed as either an hadron or an electromagnetic particle. Lastly, positional information about each of the particle is encoded as two added features ($r, \theta$), representing a radial and angular position in polar coordinates on $\eta-\phi$ plane centered around the HPS reconstructed \tauh direction of flight. No selection is applied on $r$ compared to the DeepTau case where the projection onto the grid naturally introduces a corresponding requirement. It is worth mentioning that this way of encoding relative positional information between constituents is implicit and might be suboptimal in performance. Studies of encoding it explicitly, for example, by injecting it into the attention matrices \cite{chen2021demystifying}  -- which also resembles the interaction terms from the Particle Transformer paper -- could bring more insights into how Transformer-like models can profit from inductive biases of the particle physics domain.

The training is performed by minimising with an Adam optimiser (learning rate = $1e^{-4}$, $\beta_1=0.9$, $\beta_2=0.999$, $\epsilon=1e^{-7}$) a categorical cross-entropy loss function without any training weights and with the early stopping after 3 epochs. \tauh candidates are grouped into batches of 128 via a so-called uniform dynamic batching scheme. Since \tauh candidates have different number of PF and RECO constituents, a batch is padded with 0 (separately for PF and RECO modalities) for the tensors to have a regular shape. This effectively corresponds to adding for each \tauh candidate in the batch artificial \enquote{0} tokens until the maximum PF/RECO sequence length in the batch. Traditionally, \tauh candidates for batching are sampled randomly from the training data set, which translates into a random sampling of particle sequences from the underlying distribution in the training data set (Fig. \ref{fig:smart_batching}, left). Since the distribution is skewed towards \tauh candidates with a larger number of constituents, it results in a high probability for a given batch to have such a \tauh candidate, while the other \tauh candidates will have on average significantly lower number of constituents. After padding, it results in a large number of artificial tokens in a batch and on average larger batches if measured over the constituent dimension (Fig. \ref{fig:smart_batching}, right, blue distribution). It in turn translates into inefficient computation of attention weights as it scales quadratically with a sequence length. 

To mitigate this inefficiency and produce more compact batches, the training data set is divided into 30 equal bins from 0 to 300 over the number of PF constituents per \tauh candidate. Then, a single batch is allowed to be formed only from \tauh candidates sampled from a single bin. The batches are further shuffled in order to avoid bias in the training procedure. This procedure significantly reduces the number of padded tokens after batching and brings the distribution of the number of constituents per batch close to the original distribution in the training data set before padding (Fig. \ref{fig:smart_batching}, right, orange distribution). Overall, a speed-up by a factor 2-3 in the training duration compared to the traditional batching is achieved without any difference in the performance. 
\begin{figure}[!t]
    \centering
    \includegraphics[width=\textwidth]{Figures/Tau/smart_batching.png}
    \caption{The distribution of the number of PF constituents per \tauh candidate in the training data set consisting of genuine \tauh and jet objects (left), after traditional batching (right, blue) and after uniform dynamic batching, also referred to as smart batching (right, orange). For the latter two, the distribution corresponds to the number of constituents per batch after padding.}
    \label{fig:smart_batching}
\end{figure}

\subsubsection{Experiments}

First, an impact of various modalities on the model performance is studied. Starting from the basic representation of a \tauh candidate as only a set of PF candidates, RECO electrons/muons and then global variables are added to the input representation with the model being trained for each of the three scenarios. To separate the impact of the multiclass setting, the training is performed separately for each of the three binary classification problems \tauh vs. $e/\mu/\text{jet}$. The model, with the parameters as described above, remains fixed in all of the experiments, as well as the training data set. In this particular study, the same set of input features for each of the modalities as for the DeepTau v2.5 training is used. An experiment corresponding to the model trained in a multiclass setup is additionally performed. The performance is evaluated with a ROC curve as described in Sec. \ref{deeptau5}, using the 2018 samples as in the DeepTau v2.5 case with genuine \tauh being sourced from the ggH sample, electrons and muons from the DY sample, and jets from the \ttbar (semileptonic) sample. 

\begin{figure}[!t]
    \centering
    \includegraphics[width=0.32\textwidth]{Figures/Tau/scaling_up_png/vs_jet_ggH_TT_pt_20_100_eta_0_2.3_dm_0_1_2_10_11.png}
    \includegraphics[width=0.31\textwidth]{Figures/Tau/scaling_up_png/vs_e_ggH_DY_pt_20_100_eta_0_2.3_dm_0_1_2_10_11.png}
    \includegraphics[width=0.32\textwidth]{Figures/Tau/scaling_up_png/vs_mu_ggH_DY_pt_20_100_eta_0_2.3_dm_0_1_2_10_11.png}
    \caption{Efficiency for jets (left), electrons (middle), and muons (right) versus efficiency for genuine \tauh to pass the corresponding $D_\alpha$ discriminators for DeepTau v2.1, DeepTau v2.5 and TaT models. For the latter, several configurations are trained with various modalities used as an input: PF candidates only on a binary classification task (dashed red, pale), PF candidates with RECO electrons/muons (PAT e/mu in the legend) on a binary classification task (dashed red, dark), PF candidates with RECO electrons/muons and global features on a binary classification task (solid red, pale), PF candidates with RECO electrons/muons and global features on a multiclass classification task (solid red, dark). Working points (grey dots) for DeepTau v2.1 are also shown, as derived in the original paper. The panel at the bottom of each figure shows the ratio of each of the ROC curves with respect to the one of the DeepTau v2.5 model.}
    \label{fig:tat_modalities}
\end{figure}

Overall, significant gain in performance is observed from addition of both RECO electrons/muons and global features (Fig. \ref{fig:tat_modalities}, Appendix \ref{app:tat-add}). While addition of RECO electron/muons does not affect the performance against the jet scenario, it does improve it for the \tauh against the electron (in a high \tauh efficiency region, up to 10 times reduced fake rate) and against the muon (throughout the \tauh efficiency region of interest, up to 100 times reduced fake rate) scenarios. Addition of global variables closes the gap between TaT and DeepTau v2.5 performance for the scenario against electron and pushes the performance further by up to 30\% for the scenarios against jet and muon. Lastly, switching from a binary classification to a multiclass model also improves the performance. This indicates that the model with a given set of hyperparameters profits from extending the effective size of the training data set and learning a joint representation to simultaneously discriminate between the four classes.

Second, a ParticleNet model with the parameters as described above (referred to as ParticleNet v0.1) is benchmarked against the TaT model corresponding to the multiclass scenario with all the modalities (referred to as TaT v0.2). It should be mentioned that both TaT and ParticleNet models do not have an adversarial fine-tuning step, which makes the comparison with DeepTau v2.5 more optimistic. However, the performance degradation due to this is not expeted to be large, as it was shown previosly during the DeepTau v2.5 (stage 2) training. An additional requirement on the z component of an impact parameter vector of \tauh w.r.t. a primary vertex $|d_z| < 0.2$ cm is applied during evaluation to be aligned with the recommended \tauh candidate selection (Sec. \ref{sec:reco_tau}). 

The corresponding ROC curves, evaluated with the same conditions as in the modality study, are shown on Fig. \ref{fig:benchmark} and in Appendix \ref{app:tat-bench}. Overall, the TaT architecture improves upon the current baseline of DeepTau v2.5. In the most populated region $p_\text{T}(\tauh) \in [20,100), |\eta(\tauh)| < 2.3, \text{DM}(\tauh) \in \{0,1,10,11\}$, TaT consistently reduces the misidentification rate against jet by up to 30\% across the \tauh efficiency range. For the electrons and muons, the TaT performance is slightly better compared to DeepTau v2.5 by up to 10\% in the misidentification rate at the fixed \tauh efficiency. However, the performance gain of TaT in the scenarios against electron/muon is more pronounced for the other regions of the phase space, in particular for \tauh candidates reconstructed in HPS decay modes 10 and 11, where it reaches up to 50\% and 70\% reduced misidentification rate against electrons and muons, respectively (Appendix \ref{app:tat-bench}). Against jets, the ParticleNet model reaches the similar performance as TaT (DeepTau v2.5) in the low (high) \tauh efficiency region. While overall there is little difference in the performance against muons for all of the benchmarked models in the low-\pt barrel region, for the scenario against electron ParticleNet shows a performance lower than both DeepTau v2.1 and DeepTau v2.5 by a significant margin.

In general, both models can profit from further hyperparameter tuning and an increase of the training data set size. On the Transformer side, as it is pointed out in \cite{hoffmann2022training} and also hinted in the Particle Transformer paper, attention-based models scale extremely well with the increase of training data set size. Therefore, future studies on the extended data set are needed to gauge the scalability of such models on the \tauh identification task.

\begin{figure}[!t]
    \centering
    \includegraphics[width=0.32\textwidth]{Figures/Tau/benchmark_0_png/vs_jet_ggH_TT_dz_pt_20_100_eta_0_2.3_dm_0_1_2_10_11.png}
    \includegraphics[width=0.32\textwidth]{Figures/Tau/benchmark_0_png/vs_e_ggH_DY_dz_pt_20_100_eta_0_2.3_dm_0_1_2_10_11.png}
    \includegraphics[width=0.335\textwidth]{Figures/Tau/benchmark_0_png/vs_mu_ggH_DY_dz_pt_20_100_eta_0_2.3_dm_0_1_2_10_11.png}
    \caption{Efficiency for jets (left), electrons (middle), and muons (right) versus efficiency for genuine \tauh to pass the corresponding $D_\alpha$ discriminators for DeepTau v2.1 (grey), DeepTau v2.5 (black), ParticleNet v0.1 (dark cyan) and TaT v0.2 (red) models. Working points (grey dots) for DeepTau v2.1 are also shown, as derived in the original paper. The panel at the bottom of each figure shows the ratio of each of the ROC curves with respect to the one of the DeepTau v2.5 model.}
    \label{fig:benchmark}
\end{figure}

\begin{figure}[!t]
    \centering
    \includegraphics[width=0.32\textwidth]{Figures/Tau/R_studies_png/vs_jet_ggH_TT_dz_pt_20_100_eta_0_2.3_dm_0_1_2_10_11.png}
    \includegraphics[width=0.32\textwidth]{Figures/Tau/R_studies_png/vs_e_ggH_DY_dz_pt_20_100_eta_0_2.3_dm_0_1_2_10_11.png}
    \includegraphics[width=0.327\textwidth]{Figures/Tau/R_studies_png/vs_mu_ggH_DY_dz_pt_20_100_eta_0_2.3_dm_0_1_2_10_11.png}
    \caption{Efficiency for jets (left), electrons (middle), and muons (right) versus efficiency for genuine \tauh to pass the corresponding $D_\alpha$ discriminators for a TaT architecture with the various requirements on the cone distance between the directions of flight of constituents and \tauh candidate (R). The panel at the bottom of each figure shows the ratio of each of the ROC curves with respect to the model without any requirement on the cone distance (inclusive).}
    \label{fig:r_benchmark}
\end{figure}
Last, an impact of the cone requirement on the constituents is studied for the TaT v0.2 architecture. In the original data set, used for the studies above, no selection is applied on $r$, corresponding to the radial (cone) distance in $\eta-\phi$ plane between the constituent (PF candidate or RECO electron/muon) and the HPS-reconstructed \tauh directions of flight. Additional trainings of the same architecture with the requirements on the constituents $r<0.5$, $r<0.3$ and $r<0.1$ are performed. Since training instabilities were observed during the studies with the TaT v0.2 setup, the optimiser for the training was tuned. RAdam \cite{liu2019variance} optimiser with $\beta_1=0.9$, $\beta_2=0.999$, $\epsilon=1e^{-7}$ is used. Initial learning rate is set to $1e^{-4}$ and is reduced during the training by 10 after every 10 epochs.  Also, the GeLU activation function \cite{hendrycks2016gaussian} is introduced in all TaT layers instead of the previously used ReLU function.

The comparison of evaluated ROC curves for each of the trials is presented on Fig. \ref{fig:r_benchmark} and in Appendix \ref{app:tat-cone}. As expected, the performance degrades for the tightest $r < 0.1$ case, which approximately corresponds to using only the candidates from the inner grid of DeepTau (Sec. \ref{deeptau1}). The performance as measured by a misidentification rate against electrons and muons at a given \tauh efficiency improves by about 10-20\% similarly for $r < 0.3$ and $r < 0.5$ cases. However, the $r < 0.3$ scenario is not favoured in the against jet task as being too narrow to capture the hadronisation patterns of QCD jets. Overall, the study indicates that the cone distance $r < 0.5$, also corresponding to the outer grid size of DeepTau, is the optimal option.  

In general, the studies described in this work illustrate that the field of jet tagging and representation learning in HEP can largely profit from adaptation of attention-based architectures. On the side of performance, scalability, flexibility and multimodality treatment they offer a powerful alternative to graph-based architectures for analysts to model various physics objects in the detector. This motivates future studies to understand the scope of these models' performance and the broadness of their applicability.
%\nocite{*}
 %bibliography
% \bibliographystyle{JHEPLibov}
\bibliographystyle{JHEP}
\bibliography{bibliography}

\newpage
\thispagestyle{empty}`
\mbox{}

% \appendix
% \chapter{Appendix}

\section{ML glossary}\label{app:glossary}
\begin{itemize}
	
	\item[] \textbf{Activation function:} a non-linear function which is usually applied to the output of the linear transformation within a feed-forward layer to introduce non-linearity.   

	\item[] \textbf{Backpropagation/backward pass:} a process of computing the gradient of the loss function with respect to each model weight by the chain rule, computing the gradient one layer at a time, iterating backward from the last model layer to the first one. 

	\item[] \textbf{Batch:} a set of input objects from the training data set used for a single step of the model training.

	\item[] \textbf{Categorical variable:} a variable that can take one of a limited and fixed number of possible values.

	\item[] \textbf{Decoder:} a part of the model which maps the representation learned by the encoder into a target prediction. 

	\item[] \textbf{Embedding:} a procedure mapping a given representation of the input into another representation. 
	
	\item[] \textbf{Encoder:} a part of the model which maps the input representation (usually high-dimensional) of the event/object into a representation useful for the given optimisation task (usually low-dimensional).

    \item[] \textbf{Epoch:} a period of time corresponding to the number of steps when the model completes the iteration over the entire training dataset.  

	\item[] \textbf{Forwardpropagation/forward pass:}  a process of propagating a batch of input data through the model layer by layer to the final layer which outputs a prediction and the value of the loss function.

	\item[] \textbf{Hyperparameter:} a parameter whose value is used to control the learning process or the model configuration. Therefore, it usually remains constant during the training rather than being optimised jointly with the model weights.

    \item[] \textbf{Label/Target:} information about the event/object which the model learns to predict given an input representation of the event/object during the optimisation step. 
    
    \item[] \textbf{Loss function:} a function that maps the event or values of one or more variables onto a real number representing a \enquote{cost} associated with the event. Usually parametrised per event as a function of the true event information (target) and the corresponding model predictions, the loss function is minimised as a part of an optimisation problem in order to infer the model's parameters. The term also refers to a cost associated with a group of events, usually by summing/averaging the loss function values for each of the events.

    \item[] \textbf{Model:} a mathematical representation of objects and their relationships to each other. Usually represents (a set) of functions or rules parametrised by some parameters, also called weights.

	\item[] \textbf{Optimiser:} the method to update the model weights during the training. Usually it is done using some modified form of the gradient descent. In its original definition, at the given step of the training each of the weights is updated by the negative gradient of the loss function w.r.t. to the corresponding weight as computed for a given batch during the backpropagation step. The key parameter in this update is a so-called learning rate, which is a multiplier to the gradient vector. Many optimisers do not fix it as a constant throughout the training but rather try to dynamically adjust it as the training is ongoing.

	\item[] \textbf{Receiver Operator Characteristic (ROC) curve:} a plot of the true positive rate (positive class efficiency) against the false positive rate (negative class efficiency) at various model threshold settings used to measure the model performance.
 
	\item[] \textbf{Regularisation:} a reduction of model sensitivity as measured by the change in performance to a certain type of input data variation.
    
    \item[] \textbf{Representation:} generally, the way a given event/object is described. In the context of ML, the representation is the way the object/event is numerically encoded (manually or by the model itself) in order to be used for the downstream task.   

    \item[] \textbf{Training:} the process of inferring the best parameters of the model as obtained by minimising the loss function. In most of the classification problem it is performed in steps where each step corresponds to a forward pass of a single batch to compute the value of the loss function, followed by a backward pass where the weight gradients are computed and the model parameters are updated by the optimiser. 

    \item[] \textbf{Up/Downsampling (image preprocessing):} the increase/decrease of the spatial resolution of the image while keeping the same representation. 
\end{itemize}

\section{DeepTau loss function}\label{app:loss}
The loss function minimised during the training of the DeepTau v2.1 and v2.5 models takes the form:
\begin{align*}
    L(\yt, \yp; \boldsymbol{\kappa}, \boldsymbol{w}, \boldsymbol{\gamma}) = \dfrac{1}{N_\tau}\sum_{i=1}^{N_\tau}w_i L_\text{base}(\yt_i, \yp_i; \boldsymbol{\kappa}, \boldsymbol{\gamma})\\
    L_\text{base}(\yt, \yp; \boldsymbol{\kappa}, \boldsymbol{\gamma}) = \underbrace{\kappa_\tau H_\tau(y_\tau^\text{true}, y_\tau^\text{pred})}_{\text{Categorical CE for }\tau \text{ vs. all background classes}}\\
    + \underbrace{(\kappa_e + \kappa_\mu + \kappa_\text{jet})\bar{F}_\tau(1-y_\tau^\text{true}, 1-y_\tau^\text{pred}; \gamma_\tau)}_{\text{Focal loss for all background classes vs. } \tau} \\ %
    + \underbrace{\kappa_F\sum_{i \in \{e, \mu, \text{jet}\}}\kappa_i \Theta(y_\tau^\text{pred} - 0.1)\bar{F}_\text{i}(y_i^\text{true}, y_i^\text{pred}; \gamma_i)}_{\text{Focal loss for separate background classes with } y_\tau^\text{pred} > 0.1 \text{ vs. all the other classes}}\\
\end{align*}
\begin{align*}
    &H_\tau(y_\tau^\text{true}, y_\tau^\text{pred})) = -y_\tau^\text{true} \log y_\tau^\text{pred}\\
    &\bar{F}(y^\text{true}, y^\text{pred}; \gamma)) = N \cdot F(y^\text{true}, y^\text{pred}; \gamma). \quad F(y^\text{true}, y^\text{pred}; \gamma)) = - y^\text{true} (1-y^\text{pred})^\gamma \log(y^\text{pred})\\
\end{align*}

Here, $\Theta(.)$ is the Heaviside step function, $\bar{F}_i$ are the normalised focal loss terms, $N$ is the factor normalising the focal loss to unity in the range $[0,1]$, $w_i$ are the individual training weights defined per \tauh candidate. The following constants are used for the training of the DeepTau v2.5 model (unless it is specified differently in the text):
\begin{itemize}
    \item $\kappa_e = 0.4$, $\kappa_\mu = 1.$, $\kappa_\tau = 2.$, $\kappa_\text{jet} = 0.6$, $\kappa_F = 5.$.
    \item $N_e = N_\mu = N_\text{jet} = 1.63636$, $N_\tau = 1.17153$.
    \item $\gamma_e = \gamma_\mu = \gamma_\text{jet} = 2$, $\gamma_\tau = 0.5$.
\end{itemize}

\newpage
\section{TaT: input features}\label{app:feats}
Below, the input features are listed in the form of aliases. More detailed description and formal definition of each feature can be found in \cite{code:tauml, code:tatoo}.

\begin{itemize}
	\item Global features (HPS/\tauh candidate related):
	\begin{itemize}
		\item Particle properties: \textit{particle\_type (token identifier), tau\_pt, tau\_eta, tau\_mass, tau\_E\_over\_pt, tau\_charge.}
			
		\item Isolation variables: \textit{tau\_chargedIsoPtSum, tau\_chargedIsoPtSumdR03\_over\_dR05},\\ \textit{tau\_footprintCorrection, tau\_neutralIsoPtSum, tau\_neutralIsoPtSumWeight\_over\_neutralIsoPtSum},\\ \textit{tau\_neutralIsoPtSumWeightdR03\_over\_neutralIsoPtSum, tau\_neutralIsoPtSumdR03\_over\_dR05},\\ \textit{tau\_photonPtSumOutsideSignalCone, tau\_puCorrPtSum.}
		
		\item Secondary vertex features: \textit{tau\_hasSecondaryVertex, tau\_sv\_minus\_pv\_x, tau\_sv\_minus\_pv\_y, tau\_sv\_minus\_pv\_z, tau\_flightLength\_x, tau\_flightLength\_y, tau\_flightLength\_z},\\ \textit{tau\_flightLength\_sig.}
		
		\item IP features: \textit{tau\_dxy\_valid, tau\_dxy, tau\_dxy\_sig, tau\_ip3d\_valid, tau\_ip3d, tau\_ip3d\_sig, tau\_dz\_sig\_valid, tau\_dz, tau\_dz\_sig.}
	
		\item Misc.: \textit{rho, tau\_n\_charged\_prongs, tau\_n\_neutral\_prongs, 
		 tau\_pt\_weighted\_deta\_strip, tau\_pt\_weighted\_dphi\_strip, tau\_pt\_weighted\_dr\_signal, tau\_pt\_weighted\_dr\_iso, tau\_e\_ratio\_valid, tau\_e\_ratio, tau\_gj\_angle\_diff\_valid, tau\_gj\_angle\_diff, tau\_n\_photons, tau\_emFraction, tau\_inside\_ecal\_crack, tau\_leadingTrackNormChi2},\\ \textit{tau\_leadChargedCand\_etaAtEcalEntrance\_minus\_tau\_eta}
	\end{itemize}
	\item PF candidates:
	\begin{itemize}
		\item Positional information (in $\eta-\phi$ plane): \textit{r, theta}.
		\item Particle properties: \textit{rel\_pt, particle\_type (token identifier), charge}.
		\item PV and impact parameter information: \textit{pvAssociationQuality, fromPV, vertex\_dx, vertex\_dy, vertex\_dz, vertex\_dx\_tauFL, vertex\_dy\_tauFL, vertex\_dz\_tauFL,  dxy, dxy\_sig, dz, dz\_sig}. 
		\item Tracker information:  \textit{lostInnerHits, nPixelHits, hasTrackDetails, nHits, nPixelLayers, nStripLayers, track\_ndof, chi2\_ndof}.
		\item  Calorimeter information: \textit{hcalFraction, rawCaloFraction, rawHcalFraction}. 
		\item Misc.: \textit{tauLeadChargedHadrCand, puppiWeight}.
	\end{itemize}
	\item RECO electrons:
	\begin{itemize}
		\item Positional information (in $\eta-\phi$ plane): \textit{r, theta}.
		\item Particle properties: \textit{rel\_pt, particle\_type (token identifier).}
		
		\item MVA variables: \textit{mvaInput\_earlyBrem, mvaInput\_lateBrem, mvaInput\_sigmaEtaEta, mvaInput\_hadEnergy, mvaInput\_deltaEta.}
		
		\item Track properties: \textit{rel\_trackMomentumAtVtx, rel\_trackMomentumAtCalo},\\\textit{rel\_trackMomentumOut, rel\_trackMomentumAtEleClus},\\ \textit{rel\_trackMomentumAtVtxWithConstraint, gsfTrack\_normalizedChi2},\\ \textit{gsfTrack\_numberOfValidHits, rel\_gsfTrack\_pt},\\ \textit{gsfTrack\_pt\_sig, has\_closestCtfTrack, closestCtfTrack\_normalizedChi2},\\ \textit{closestCtfTrack\_numberOfValidHits.}
		
		\item Cluster variables: \textit{cc\_valid, cc\_ele\_rel\_energy, cc\_gamma\_rel\_energy, cc\_n\_gamma, rel\_ecalEnergy, ecalEnergy\_sig, eSuperClusterOverP, eSeedClusterOverP, eSeedClusterOverPout, eEleClusterOverPout,
		deltaEtaSuperClusterTrackAtVtx, deltaEtaSeedClusterTrackAtCalo, deltaEtaEleClusterTrackAtCalo,
		deltaPhiEleClusterTrackAtCalo, deltaPhiSuperClusterTrackAtVtx, deltaPhiSeedClusterTrackAtCalo.}
		
		\item Shower shape variable: \textit{sigmaEtaEta, sigmaIetaIeta, sigmaIphiIphi, sigmaIetaIphi, e1x5, e2x5Max, e5x5, r9, hcalDepth1OverEcal, hcalDepth2OverEcal, hcalDepth1OverEcalBc, hcalDepth2OverEcalBc, eLeft, eRight, eBottom, eTop, full5x5\_sigmaEtaEta},\\ \textit{full5x5\_sigmaIetaIeta, full5x5\_sigmaIphiIphi, full5x5\_sigmaIetaIphi, full5x5\_e1x5},\\ \textit{full5x5\_e2x5Max, full5x5\_e5x5, full5x5\_r9, full5x5\_hcalDepth1OverEcal},\\ \textit{full5x5\_hcalDepth2OverEcal, full5x5\_hcalDepth1OverEcalBc},\\ \textit{full5x5\_hcalDepth2OverEcalBc, full5x5\_eLeft, full5x5\_eRight},\\ \textit{full5x5\_eBottom, full5x5\_eTop, full5x5\_e2x5Left},\\ \textit{full5x5\_e2x5Right, full5x5\_e2x5Bottom, full5x5\_e2x5Top.}
	\end{itemize}
	\item RECO muons:
	\begin{itemize}
		\item Positional information (in $\eta-\phi$ plane): \textit{r, theta.}
		\item Particle properties:  \textit{rel\_pt, particle\_type (token identifier).}
		\item Calorimeter information: 	\textit{segmentCompatibility, caloCompatibility, pfEcalEnergy\_valid, rel\_pfEcalEnergy.}
		\item Track and impact parameter features: \textit{dxy, dxy\_sig, normalizedChi2\_valid, normalizedChi2, numberOfValidHits.}
		\item Muon chambers information: \textit{n\_matches\_DT\_\{1,2,3,4\}, n\_matches\_CSC\_\{1,2,3,4\}, n\_matches\_RPC\_\{1,2,3,4\}, n\_hits\_DT\_\{1,2,3,4\}, n\_hits\_CSC\_\{1,2,3,4\}, n\_hits\_RPC\_\{1,2,3,4\}}
	\end{itemize}
\end{itemize}
\newpage
\section{TaT ablation study: impact of modalities}\label{app:tat-add}

\begin{figure}[H]
    \centering
    \includegraphics[width=0.32\textwidth]{Figures/Tau/scaling_up_png/vs_e_ggH_DY_dm_00.png}
    \includegraphics[width=0.32\textwidth]{Figures/Tau/scaling_up_png/vs_e_ggH_DY_dm_11.png}
    \includegraphics[width=0.32\textwidth]{Figures/Tau/scaling_up_png/vs_e_ggH_DY_dm_1010.png}
    \includegraphics[width=0.32\textwidth]{Figures/Tau/scaling_up_png/vs_e_ggH_DY_dm_1111.png}
    \includegraphics[width=0.32\textwidth]{Figures/Tau/scaling_up_png/vs_e_ggH_DY_pt_20_100_eta_0_2.3_dm_0_1_2_10_11.png}
    \includegraphics[width=0.32\textwidth]{Figures/Tau/scaling_up_png/vs_e_ggH_DY_pt_20_100_eta_2.3_2.5_dm_0_1_2_10_11.png}
    \includegraphics[width=0.32\textwidth]{Figures/Tau/scaling_up_png/vs_e_ggH_DY_pt_100_1000_eta_0_2.3_dm_0_1_2_10_11.png}
    \includegraphics[width=0.32\textwidth]{Figures/Tau/scaling_up_png/vs_e_ggH_DY_pt_100_1000_eta_2.3_2.5_dm_0_1_2_10_11.png}
    \caption{Efficiency for electrons versus efficiency for genuine \tauh to pass the corresponding $D_\alpha$ discriminators for DeepTau v2.1, DeepTau v2.5 and TaT models for various \pt, $\eta$, and \tauh decay mode regions. For the TaT model, several configurations are trained with various modalities used as an input: PF candidates only on a binary classification task (dashed red, pale), PF candidates with RECO electrons/muons (PAT e/mu in the legend) on a binary classification task (dashed red, dark), PF candidates with RECO electrons/muons and global features on a binary classification task (solid red, pale), PF candidates with RECO electrons/muons and global features on a multiclass classification task (solid red, dark). Working points (grey dots) for DeepTau v2.1 are also shown, as derived in the original paper. The panel at the bottom of each figure shows the ratio of each of the ROC curves with respect to the one of the DeepTau v2.5 model.}
\end{figure}

\begin{figure}[H]
    \centering
    \includegraphics[width=0.32\textwidth]{Figures/Tau/scaling_up_png/vs_mu_ggH_DY_dm_00.png}
    \includegraphics[width=0.32\textwidth]{Figures/Tau/scaling_up_png/vs_mu_ggH_DY_dm_11.png}
    \includegraphics[width=0.32\textwidth]{Figures/Tau/scaling_up_png/vs_mu_ggH_DY_dm_1010.png}
    \includegraphics[width=0.32\textwidth]{Figures/Tau/scaling_up_png/vs_mu_ggH_DY_dm_1111.png}
    \includegraphics[width=0.32\textwidth]{Figures/Tau/scaling_up_png/vs_mu_ggH_DY_pt_20_100_eta_0_2.3_dm_0_1_2_10_11.png}
    \includegraphics[width=0.32\textwidth]{Figures/Tau/scaling_up_png/vs_mu_ggH_DY_pt_20_100_eta_2.3_2.5_dm_0_1_2_10_11.png}
    \includegraphics[width=0.32\textwidth]{Figures/Tau/scaling_up_png/vs_mu_ggH_DY_pt_100_1000_eta_0_2.3_dm_0_1_2_10_11.png}
    \includegraphics[width=0.32\textwidth]{Figures/Tau/scaling_up_png/vs_mu_ggH_DY_pt_100_1000_eta_2.3_2.5_dm_0_1_2_10_11.png}
    \caption{Efficiency for muons versus efficiency for genuine \tauh to pass the corresponding $D_\alpha$ discriminators for DeepTau v2.1, DeepTau v2.5 and TaT models for various \pt, $\eta$, and \tauh decay mode regions. For the TaT model, several configurations are trained with various modalities used as an input: PF candidates only on a binary classification task (dashed red, pale), PF candidates with RECO electrons/muons (PAT e/mu in the legend) on a binary classification task (dashed red, dark), PF candidates with RECO electrons/muons and global features on a binary classification task (solid red, pale), PF candidates with RECO electrons/muons and global features on a multiclass classification task (solid red, dark). Working points (grey dots) for DeepTau v2.1 are also shown, as derived in the original paper. The panel at the bottom of each figure shows the ratio of each of the ROC curves with respect to the one of the DeepTau v2.5 model.}
\end{figure}

\begin{figure}[H]
    \centering
    \includegraphics[width=0.32\textwidth]{Figures/Tau/scaling_up_png/vs_jet_ggH_TT_dm_00.png}
    \includegraphics[width=0.32\textwidth]{Figures/Tau/scaling_up_png/vs_jet_ggH_TT_dm_11.png}
    \includegraphics[width=0.32\textwidth]{Figures/Tau/scaling_up_png/vs_jet_ggH_TT_dm_1010.png}
    \includegraphics[width=0.32\textwidth]{Figures/Tau/scaling_up_png/vs_jet_ggH_TT_dm_1111.png}
    \includegraphics[width=0.32\textwidth]{Figures/Tau/scaling_up_png/vs_jet_ggH_TT_pt_20_100_eta_0_2.3_dm_0_1_2_10_11.png}
    \includegraphics[width=0.32\textwidth]{Figures/Tau/scaling_up_png/vs_jet_ggH_TT_pt_20_100_eta_2.3_2.5_dm_0_1_2_10_11.png}
    \includegraphics[width=0.32\textwidth]{Figures/Tau/scaling_up_png/vs_jet_ggH_TT_pt_100_1000_eta_0_2.3_dm_0_1_2_10_11.png}
    \includegraphics[width=0.32\textwidth]{Figures/Tau/scaling_up_png/vs_jet_ggH_TT_pt_100_1000_eta_2.3_2.5_dm_0_1_2_10_11.png}
    \caption{Efficiency for jets versus efficiency for genuine \tauh to pass the corresponding $D_\alpha$ discriminators for DeepTau v2.1, DeepTau v2.5 and TaT models for various \pt, $\eta$, and \tauh decay mode regions. For the TaT model, several configurations are trained with various modalities used as an input: PF candidates only on a binary classification task (dashed red, pale), PF candidates with RECO electrons/muons (PAT e/mu in the legend) on a binary classification task (dashed red, dark), PF candidates with RECO electrons/muons and global features on a binary classification task (solid red, pale), PF candidates with RECO electrons/muons and global features on a multiclass classification task (solid red, dark). Working points (grey dots) for DeepTau v2.1 are also shown, as derived in the original paper. The panel at the bottom of each figure shows the ratio of each of the ROC curves with respect to the one of the DeepTau v2.5 model.}
\end{figure}

\newpage
\section{Performance comparison for TaT and ParticleNet}\label{app:tat-bench}

\begin{figure}[H]
    \centering
    \includegraphics[width=0.32\textwidth]{Figures/Tau/benchmark_0_png/vs_e_ggH_DY_dz_dm_00.png}
    \includegraphics[width=0.32\textwidth]{Figures/Tau/benchmark_0_png/vs_e_ggH_DY_dz_dm_11.png}
    \includegraphics[width=0.32\textwidth]{Figures/Tau/benchmark_0_png/vs_e_ggH_DY_dz_dm_1010.png}
    \includegraphics[width=0.32\textwidth]{Figures/Tau/benchmark_0_png/vs_e_ggH_DY_dz_dm_1111.png}
    \includegraphics[width=0.32\textwidth]{Figures/Tau/benchmark_0_png/vs_e_ggH_DY_dz_pt_20_100_eta_0_2.3_dm_0_1_2_10_11.png}
    \includegraphics[width=0.32\textwidth]{Figures/Tau/benchmark_0_png/vs_e_ggH_DY_dz_pt_20_100_eta_2.3_2.5_dm_0_1_2_10_11.png}
    \includegraphics[width=0.32\textwidth]{Figures/Tau/benchmark_0_png/vs_e_ggH_DY_dz_pt_100_1000_eta_0_2.3_dm_0_1_2_10_11.png}
    \includegraphics[width=0.32\textwidth]{Figures/Tau/benchmark_0_png/vs_e_ggH_DY_dz_pt_100_1000_eta_2.3_2.5_dm_0_1_2_10_11.png}
    \caption{Efficiency for electrons versus efficiency for genuine \tauh to pass the corresponding $D_\alpha$ discriminators for DeepTau v2.1 (grey), DeepTau v2.5 (black), ParticleNet v0.1 (dark cyan) and TaT v0.2 (red) models. Working points (grey dots) for DeepTau v2.1 are also shown, as derived in the original paper. The panel at the bottom of each figure shows the ratio of each of the ROC curves with respect to the one of the DeepTau v2.5 model.}
\end{figure}

\begin{figure}[H]
    \centering
    \includegraphics[width=0.32\textwidth]{Figures/Tau/benchmark_0_png/vs_mu_ggH_DY_dz_dm_00.png}
    \includegraphics[width=0.32\textwidth]{Figures/Tau/benchmark_0_png/vs_mu_ggH_DY_dz_dm_11.png}
    \includegraphics[width=0.32\textwidth]{Figures/Tau/benchmark_0_png/vs_mu_ggH_DY_dz_dm_1010.png}
    \includegraphics[width=0.32\textwidth]{Figures/Tau/benchmark_0_png/vs_mu_ggH_DY_dz_dm_1111.png}
    \includegraphics[width=0.32\textwidth]{Figures/Tau/benchmark_0_png/vs_mu_ggH_DY_dz_pt_20_100_eta_0_2.3_dm_0_1_2_10_11.png}
    \includegraphics[width=0.32\textwidth]{Figures/Tau/benchmark_0_png/vs_mu_ggH_DY_dz_pt_20_100_eta_2.3_2.5_dm_0_1_2_10_11.png}
    \includegraphics[width=0.32\textwidth]{Figures/Tau/benchmark_0_png/vs_mu_ggH_DY_dz_pt_100_1000_eta_0_2.3_dm_0_1_2_10_11.png}
    \includegraphics[width=0.32\textwidth]{Figures/Tau/benchmark_0_png/vs_mu_ggH_DY_dz_pt_100_1000_eta_2.3_2.5_dm_0_1_2_10_11.png}
    \caption{Efficiency for muons versus efficiency for genuine \tauh to pass the corresponding $D_\alpha$ discriminators for DeepTau v2.1 (grey), DeepTau v2.5 (black), ParticleNet v0.1 (dark cyan) and TaT v0.2 (red) models. Working points (grey dots) for DeepTau v2.1 are also shown, as derived in the original paper. The panel at the bottom of each figure shows the ratio of each of the ROC curves with respect to the one of the DeepTau v2.5 model.}
\end{figure}

\begin{figure}[H]
    \centering
    \includegraphics[width=0.32\textwidth]{Figures/Tau/benchmark_0_png/vs_jet_ggH_TT_dz_dm_00.png}
    \includegraphics[width=0.32\textwidth]{Figures/Tau/benchmark_0_png/vs_jet_ggH_TT_dz_dm_11.png}
    \includegraphics[width=0.32\textwidth]{Figures/Tau/benchmark_0_png/vs_jet_ggH_TT_dz_dm_1010.png}
    \includegraphics[width=0.32\textwidth]{Figures/Tau/benchmark_0_png/vs_jet_ggH_TT_dz_dm_1111.png}
    \includegraphics[width=0.32\textwidth]{Figures/Tau/benchmark_0_png/vs_jet_ggH_TT_dz_pt_20_100_eta_0_2.3_dm_0_1_2_10_11.png}
    \includegraphics[width=0.32\textwidth]{Figures/Tau/benchmark_0_png/vs_jet_ggH_TT_dz_pt_20_100_eta_2.3_2.5_dm_0_1_2_10_11.png}
    \includegraphics[width=0.32\textwidth]{Figures/Tau/benchmark_0_png/vs_jet_ggH_TT_dz_pt_100_1000_eta_0_2.3_dm_0_1_2_10_11.png}
    \includegraphics[width=0.32\textwidth]{Figures/Tau/benchmark_0_png/vs_jet_ggH_TT_dz_pt_100_1000_eta_2.3_2.5_dm_0_1_2_10_11.png}
    \caption{Efficiency for jets versus efficiency for genuine \tauh to pass the corresponding $D_\alpha$ discriminators for DeepTau v2.1 (grey), DeepTau v2.5 (black), ParticleNet v0.1 (dark cyan) and TaT v0.2 (red) models. Working points (grey dots) for DeepTau v2.1 are also shown, as derived in the original paper. The panel at the bottom of each figure shows the ratio of each of the ROC curves with respect to the one of the DeepTau v2.5 model.}
\end{figure}

\newpage
\section{TaT ablation study: variation of the cone size}\label{app:tat-cone}

\begin{figure}[H]
    \centering
    \includegraphics[width=0.32\textwidth]{Figures/Tau/R_studies_png/vs_e_ggH_DY_dz_dm_00.png}
    \includegraphics[width=0.32\textwidth]{Figures/Tau/R_studies_png/vs_e_ggH_DY_dz_dm_11.png}
    \includegraphics[width=0.32\textwidth]{Figures/Tau/R_studies_png/vs_e_ggH_DY_dz_dm_1010.png}
    \includegraphics[width=0.32\textwidth]{Figures/Tau/R_studies_png/vs_e_ggH_DY_dz_dm_1111.png}
    \includegraphics[width=0.32\textwidth]{Figures/Tau/R_studies_png/vs_e_ggH_DY_dz_pt_20_100_eta_0_2.3_dm_0_1_2_10_11.png}
    \includegraphics[width=0.32\textwidth]{Figures/Tau/R_studies_png/vs_e_ggH_DY_dz_pt_20_100_eta_2.3_2.5_dm_0_1_2_10_11.png}
    \includegraphics[width=0.32\textwidth]{Figures/Tau/R_studies_png/vs_e_ggH_DY_dz_pt_100_1000_eta_0_2.3_dm_0_1_2_10_11.png}
    \includegraphics[width=0.32\textwidth]{Figures/Tau/R_studies_png/vs_e_ggH_DY_dz_pt_100_1000_eta_2.3_2.5_dm_0_1_2_10_11.png}
    \caption{Efficiency for electrons versus efficiency for genuine \tauh to pass the corresponding $D_\alpha$ discriminators for a TaT architecture with the various requirements on the cone distance between the directions of flight of constituents and \tauh candidate (R). The panel at the bottom of each figure shows the ratio of each of the ROC curves with respect to the model without any requirement on the cone distance (inclusive).}
\end{figure}


\begin{figure}[H]
    \centering
    \includegraphics[width=0.32\textwidth]{Figures/Tau/R_studies_png/vs_mu_ggH_DY_dz_dm_00.png}
    \includegraphics[width=0.32\textwidth]{Figures/Tau/R_studies_png/vs_mu_ggH_DY_dz_dm_11.png}
    \includegraphics[width=0.32\textwidth]{Figures/Tau/R_studies_png/vs_mu_ggH_DY_dz_dm_1010.png}
    \includegraphics[width=0.32\textwidth]{Figures/Tau/R_studies_png/vs_mu_ggH_DY_dz_dm_1111.png}
    \includegraphics[width=0.32\textwidth]{Figures/Tau/R_studies_png/vs_mu_ggH_DY_dz_pt_20_100_eta_0_2.3_dm_0_1_2_10_11.png}
    \includegraphics[width=0.32\textwidth]{Figures/Tau/R_studies_png/vs_mu_ggH_DY_dz_pt_20_100_eta_2.3_2.5_dm_0_1_2_10_11.png}
    \includegraphics[width=0.32\textwidth]{Figures/Tau/R_studies_png/vs_mu_ggH_DY_dz_pt_100_1000_eta_0_2.3_dm_0_1_2_10_11.png}
    \includegraphics[width=0.32\textwidth]{Figures/Tau/R_studies_png/vs_mu_ggH_DY_dz_pt_100_1000_eta_2.3_2.5_dm_0_1_2_10_11.png}
    \caption{Efficiency for muons versus efficiency for genuine \tauh to pass the corresponding $D_\alpha$ discriminators for a TaT architecture with the various requirements on the cone distance between the directions of flight of constituents and \tauh candidate (R). The panel at the bottom of each figure shows the ratio of each of the ROC curves with respect to the model without any requirement on the cone distance (inclusive).}
\end{figure}


\begin{figure}[H]
    \centering
    \includegraphics[width=0.32\textwidth]{Figures/Tau/R_studies_png/vs_jet_ggH_TT_dz_dm_00.png}
    \includegraphics[width=0.32\textwidth]{Figures/Tau/R_studies_png/vs_jet_ggH_TT_dz_dm_11.png}
    \includegraphics[width=0.32\textwidth]{Figures/Tau/R_studies_png/vs_jet_ggH_TT_dz_dm_1010.png}
    \includegraphics[width=0.32\textwidth]{Figures/Tau/R_studies_png/vs_jet_ggH_TT_dz_dm_1111.png}
    \includegraphics[width=0.32\textwidth]{Figures/Tau/R_studies_png/vs_jet_ggH_TT_dz_pt_20_100_eta_0_2.3_dm_0_1_2_10_11.png}
    \includegraphics[width=0.32\textwidth]{Figures/Tau/R_studies_png/vs_jet_ggH_TT_dz_pt_20_100_eta_2.3_2.5_dm_0_1_2_10_11.png}
    \includegraphics[width=0.32\textwidth]{Figures/Tau/R_studies_png/vs_jet_ggH_TT_dz_pt_100_1000_eta_0_2.3_dm_0_1_2_10_11.png}
    \includegraphics[width=0.32\textwidth]{Figures/Tau/R_studies_png/vs_jet_ggH_TT_dz_pt_100_1000_eta_2.3_2.5_dm_0_1_2_10_11.png}
    \caption{Efficiency for jets versus efficiency for genuine \tauh to pass the corresponding $D_\alpha$ discriminators for a TaT architecture with the various requirements on the cone distance between the directions of flight of constituents and \tauh candidate (R). The panel at the bottom of each figure shows the ratio of each of the ROC curves with respect to the model without any requirement on the cone distance (inclusive).}
\end{figure}

\newpage
\section{Control plots in the \et channel}\label{app:control-plots}
\begin{figure}[H]
    \centering
    \includegraphics[width=0.3\textwidth]{Figures/CP_etau/2016/1.png}
    \includegraphics[width=0.3\textwidth]{Figures/CP_etau/2016/2.png}
    \includegraphics[width=0.3\textwidth]{Figures/CP_etau/2016/3.png}
    % \includegraphics[width=0.31\textwidth]{Figures/CP_etau/mvis.png}
    \includegraphics[width=0.3\textwidth]{Figures/CP_etau/2016/12.png}
    \includegraphics[width=0.3\textwidth]{Figures/CP_etau/2016/5.png}
    \includegraphics[width=0.3\textwidth]{Figures/CP_etau/2016/6.png}
    \includegraphics[width=0.3\textwidth]{Figures/CP_etau/2016/7.png}
    \includegraphics[width=0.3\textwidth]{Figures/CP_etau/2016/8.png}
    \includegraphics[width=0.3\textwidth]{Figures/CP_etau/2016/9.png}
    \includegraphics[width=0.3\textwidth]{Figures/CP_etau/2016/10.png}
    \includegraphics[width=0.3\textwidth]{Figures/CP_etau/2016/11.png}
    \includegraphics[width=0.3\textwidth]{Figures/CP_etau/2016/4.png}
    \caption{Comparison of data with simulation for the 2016 data-taking period for the variables used in the neural network training, as described in Sec. \ref{sec:categ}.}
\end{figure}

\begin{figure}[H]
    \centering
    \includegraphics[width=0.3\textwidth]{Figures/CP_etau/2017/2.png}
    \includegraphics[width=0.3\textwidth]{Figures/CP_etau/2017/3.png}
    % \includegraphics[width=0.31\textwidth]{Figures/CP_etau/mvis.png}
    \includegraphics[width=0.3\textwidth]{Figures/CP_etau/2017/4.png}
    \includegraphics[width=0.3\textwidth]{Figures/CP_etau/2017/1.png}
    \includegraphics[width=0.3\textwidth]{Figures/CP_etau/2017/6.png}
    \includegraphics[width=0.3\textwidth]{Figures/CP_etau/2017/7.png}
    \includegraphics[width=0.3\textwidth]{Figures/CP_etau/2017/8.png}
    \includegraphics[width=0.3\textwidth]{Figures/CP_etau/2017/9.png}
    \includegraphics[width=0.3\textwidth]{Figures/CP_etau/2017/10.png}
    \includegraphics[width=0.3\textwidth]{Figures/CP_etau/2017/11.png}
    \includegraphics[width=0.3\textwidth]{Figures/CP_etau/2017/12.png}
    \includegraphics[width=0.3\textwidth]{Figures/CP_etau/2017/5.png}
    \caption{Comparison of data with simulation for the 2017 data-taking period for the variables used in the neural network training, as described in Sec. \ref{sec:categ}.}
\end{figure}

\newpage
\section{Pre-fit distributions in the \et channel}\label{app:pre-fit}

\begin{figure}[H]
    \centering
    \includegraphics[width=0.35\textwidth]{Figures/CP_etau/NN_score_embed_et_2016_higgs_prefit.png}
    \includegraphics[width=0.35\textwidth]{Figures/CP_etau/NN_score_fakes_et_2016_fakes_prefit.png}
    \includegraphics[width=0.35\textwidth]{Figures/CP_etau/NN_score_embed_et_2017_higgs_prefit.png}
    \includegraphics[width=0.35\textwidth]{Figures/CP_etau/NN_score_fakes_et_2017_fakes_prefit.png}
    \includegraphics[width=0.35\textwidth]{Figures/CP_etau/NN_score_embed_et_2018_higgs_prefit.png}
    \includegraphics[width=0.35\textwidth]{Figures/CP_etau/NN_score_fakes_et_2018_fakes_prefit.png}
    \caption{Pre-fit distribution of the NN score in the genuine $\tau$ (left) and fakes (right) background categories for the 2016 (top), 2017 (middle), and 2018 (bottom) data-taking periods.}
\end{figure}

\begin{figure}[H]
    \centering
    \includegraphics[width=0.42\textwidth]{Figures/CP_etau/Bin_number_e-0a1_et_2016_higgs_prefit.png}
    \includegraphics[width=0.42\textwidth]{Figures/CP_etau/Bin_number_e-a1_et_2016_higgs_prefit.png}
    \includegraphics[width=0.42\textwidth]{Figures/CP_etau/Bin_number_e-pi_et_2016_higgs_prefit.png}
    \includegraphics[width=0.42\textwidth]{Figures/CP_etau/Bin_number_e-rho_et_2016_higgs_prefit.png}
    \caption{Pre-fit distributions of the unrolled \phicp observable in bins of the NN score in the signal categories for the 2016 data-taking period.}
\end{figure}

\begin{figure}[H]
    \centering
    \includegraphics[width=0.42\textwidth]{Figures/CP_etau/Bin_number_e-0a1_et_2017_higgs_prefit.png}
    \includegraphics[width=0.42\textwidth]{Figures/CP_etau/Bin_number_e-a1_et_2017_higgs_prefit.png}
    \includegraphics[width=0.42\textwidth]{Figures/CP_etau/Bin_number_e-pi_et_2017_higgs_prefit.png}
    \includegraphics[width=0.42\textwidth]{Figures/CP_etau/Bin_number_e-rho_et_2017_higgs_prefit.png}
    \caption{Pre-fit distributions of the unrolled \phicp observable in bins of the NN score in the signal categories for the 2017 data-taking period.}
\end{figure}

\begin{figure}[H]
    \centering
    \includegraphics[width=0.42\textwidth]{Figures/CP_etau/Bin_number_e-0a1_et_2018_higgs_prefit.png}
    \includegraphics[width=0.42\textwidth]{Figures/CP_etau/Bin_number_e-a1_et_2018_higgs_prefit.png}
    \includegraphics[width=0.42\textwidth]{Figures/CP_etau/Bin_number_e-pi_et_2018_higgs_prefit.png}
    \includegraphics[width=0.42\textwidth]{Figures/CP_etau/Bin_number_e-rho_et_2018_higgs_prefit.png}
    \caption{Pre-fit distributions of the unrolled \phicp observable in bins of the NN score in the signal categories for the 2018 data-taking period.}
\end{figure}

\newpage
\section{Post-fit distributions in the \et channel}\label{app:post-fit}

\begin{figure}[H]
    \centering
    \includegraphics[width=0.35\textwidth]{Figures/CP_etau/NN_score_embed_et_2016_higgs_postfit.png}
    \includegraphics[width=0.35\textwidth]{Figures/CP_etau/NN_score_fakes_et_2016_fakes_postfit}
    \includegraphics[width=0.35\textwidth]{Figures/CP_etau/NN_score_embed_et_2017_higgs_postfit.png}
    \includegraphics[width=0.35\textwidth]{Figures/CP_etau/NN_score_fakes_et_2017_fakes_postfit.png}
    \includegraphics[width=0.35\textwidth]{Figures/CP_etau/NN_score_embed_et_2018_higgs_postfit.png}
    \includegraphics[width=0.35\textwidth]{Figures/CP_etau/NN_score_fakes_et_2018_fakes_postfit.png}
    \caption{Post-fit distribution of the NN score in the genuine $\tau$ (left) and fakes (right) background categories for the 2016 (top), 2017 (middle), and 2018 (bottom) data-taking periods.}
\end{figure}

\begin{figure}[H]
    \centering
    \includegraphics[width=0.42\textwidth]{Figures/CP_etau/Bin_number_e-0a1_et_2016_higgs_postfit.png}
    \includegraphics[width=0.42\textwidth]{Figures/CP_etau/Bin_number_e-a1_et_2016_higgs_postfit.png}
    \includegraphics[width=0.42\textwidth]{Figures/CP_etau/Bin_number_e-pi_et_2016_higgs_postfit.png}
    \includegraphics[width=0.42\textwidth]{Figures/CP_etau/Bin_number_e-rho_et_2016_higgs_postfit.png}
    \caption{Post-fit distributions of the unrolled \phicp observable in bins of the NN score in the signal categories for the 2016 data-taking period.}
\end{figure}

\begin{figure}[H]
    \centering
    \includegraphics[width=0.42\textwidth]{Figures/CP_etau/Bin_number_e-0a1_et_2017_higgs_postfit.png}
    \includegraphics[width=0.42\textwidth]{Figures/CP_etau/Bin_number_e-a1_et_2017_higgs_postfit.png}
    \includegraphics[width=0.42\textwidth]{Figures/CP_etau/Bin_number_e-pi_et_2017_higgs_postfit.png}
    \includegraphics[width=0.42\textwidth]{Figures/CP_etau/Bin_number_e-rho_et_2017_higgs_postfit.png}
    \caption{Post-fit distributions of the unrolled \phicp observable in bins of the NN score in the signal categories for the 2017 data-taking period.}
\end{figure}

\begin{figure}[H]
    \centering
    \includegraphics[width=0.42\textwidth]{Figures/CP_etau/Bin_number_e-0a1_et_2018_higgs_postfit.png}
    \includegraphics[width=0.42\textwidth]{Figures/CP_etau/Bin_number_e-a1_et_2018_higgs_postfit.png}
    \includegraphics[width=0.42\textwidth]{Figures/CP_etau/Bin_number_e-pi_et_2018_higgs_postfit.png}
    \includegraphics[width=0.42\textwidth]{Figures/CP_etau/Bin_number_e-rho_et_2018_higgs_postfit.png}
    \caption{Post-fit distributions of the unrolled \phicp observable in bins of the NN score in the signal categories for the 2018 data-taking period.}
\end{figure}

% \thispagestyle{empty}
\vspace{-3cm}
\section*{\centering Acknowledgements}

\thispagestyle{empty}


\newpage




\end{document}