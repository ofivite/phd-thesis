\chapter{Summary}\label{sec:outro}

This work presents the measurement of the CP properties of the Yukawa coupling between the Higgs boson and tau leptons with the CMS experiment at the LHC in the proton-proton collisions at $\sqrt{s} = 13$ TeV. The data set corresponds to the integrated luminosity of 137 \ifb collected in 2016, 2017, and 2018 years. The analysis was performed in the \et channel where one tau lepton decays into a single electron and the other hadronically via the following decay modes: $\pi^\pm$, $\rho \to \pi^\pm \pi^0$, $a_1^\pm \to \pi^\pm\pi^0\pi^0$, $a_1^\pm \to \pi^\pm \pi^\mp \pi^\pm$. 

The structure of the Yukawa coupling between the Higgs boson and tau leptons was parametrised in terms of the effective mixing angle \mixa between the pure CP-even and pure CP-odd contributions. The angle between the tau lepton decay planes in the zero-momentum frame \phicp was used as an observable to experimentally extract the \mixa value. For the \htt decays the distribution for \phicp represents modulations where \mixa enters as a phase shift with the $\mixa=0^\circ$ corresponding to the SM scenario of the pure CP-even coupling. Dedicated methods were developed and optimised to experimentally reconstruct the \phicp angle for each of the final states.  

Considering the \et channel only, the expected significance to reject the pure CP-odd hypothesis under the pure CP-even hypothesis was $0.99\sigma$ with $e\rho$ (expected significance $0.57\sigma$) and $e\pi$ (expected significance $0.54\sigma$) being the most sensitive final states. The observed (expected) value of the effective mixing angle was obtained to be:
\begin{equation}
    \mixa_{\tau_e\tauh} = -48^{+51^\circ}_{-42^\circ}(0 \pm 90^\circ) ~@68.3\% \text{ CL}.
\end{equation}

This result cannot lead to conclusive statements \textit{per se} since it is large limited by the available statistics and instrumental precision of reconstructing the \phicp observable. The combination of the measurements in the \et channel with those from the \mt and \tata channels was performed. The observed (expected) significance to reject the pure CP-even hypothesis was measured to be $3.0 (2.6)\sigma$. If one considers only \mt and \tata channels, the corresponding value was measured to be $3.2 (2.3)\sigma$. 

The observed (expected) value of the effective mixing angle for the combination of the \et, \mt, and \tata channels was obtained to be:
\begin{equation}
    \mixa = -1 \pm 19^\circ (0 \pm 21^\circ) ~@68.3\% \text{ CL}.
\end{equation}

This value is compatible with the SM prediction as well as with the measurement by the ATLAS experiment: $\mixa = 9 \pm 16^\circ (0 \pm 28^\circ) ~@68\% \text{ CL}$ \cite{ATLAS:2022phj}. The results presented in this work are published in the Journal of High Energy Physics \cite{CMS:2021sdq}.

Studies were performed to improve the tau lepton identification algorithms in CMS. The DeepTau model, extensively used by analysts during the Run 2 data-taking period, was retrained and optimised. This resulted in the consistently reduced misidentification rate against both electrons, muons, and jets by 10-50\% at a given efficiency across the phase space regions of interest. The dedicated adversarial fine-tuning provided improved agreement between data and simulation in the distribution of the $D_\text{jet}$ score in the region corresponding to the baseline \htt selection without significant decrease in the model performance. The resulting model will be used as a recommended algorithm for the tau lepton identification in CMS during the early Run 3 data-taking period. 

Several set-based architectures were developed and adapted to the tau lepton identification task. This includes the ParticleNet and a newly proposed Tau Transformer (TaT) architectures. The training was performed in the unified framework where the tau lepton representation was modelled as the combination of various modalities and the architectures enter as the feature extracting modules. Ablation studies were performed to investigate the contribution of various modalities and the cone size of \tauh candidates on the TaT model performance. Overall, compared to the retrained DeepTau baseline, the ParticleNet architecture achieves similarly or better performance on the \tauh classification against jets and muons, while performing sizeably worse against electrons. The Tau Transformer architecture consistently outperforms the retrained DeepTau baseline across \pt, $\eta$ and tau decay mode regions of interest providing a reduction in the misidentification rate at a given \tauh efficiency by up to 50\%. 

The measurement of the CP properties of the Higgs boson presented in this work will largely profit from more data. With that respect, the ongoing Run 3 and the upcoming HL-LHC periods of data-taking will allow for a more precise measurement of the structure of Yukawa coupling in the $\text{H}\tau\tau$ sector, which is currently sizeably limited by the statistical uncertainties. Furthermore, other scenarios of the coupling structure, e.g. non-Hermitian, can be investigated, as well as the combination with the measurements of the Higgs coupling with the top quark.  

From the experimental side, this is also a great opportunity to design and advance analysis techniques to account for specific features of the CP measurement and the underlying physics. Such ideas as the full ditau system reconstruction and direct \phicp angle regression are challenging and yet exciting to be explored. On the side of the tau lepton reconstruction, there is room for improvement to recover the inefficiencies of the HPS algorithm when it comes to the missing neutrino and $\pi^0$ reconstruction. Furthermore, better encoding of inductive biases into the model can be studied, e.g. in the positional encoding or multimodality structure of the input representation. Lastly, the study of self-supervised approaches to the model training will potentially open a new chapter in the object tagging by leveraging enormous amount of data collected by the LHC experiments. 

The Standard Model to date provides a very accurate description of various physics phenomena down to the particle scale, thus laying the very foundation of our understanding of nature. Despite this, several fundamental questions are not yet answered within the SM, which indicates that the knowledge puzzle is still missing some pieces. However, it seems reasonable to believe that human curiosity is stubborn enough to make it only a matter of time before the paradigm will broaden and a space for new questions will open up.