\chapter{The Standard Model}\label{sec:sm}

It is an indispensable part of human nature to wonder about phenomena happening in the world and to ask questions about their origin. It seems to be driven by inexplicable curiosity to understand, almost a demand of the reason to build logical structures, through centuries adding one brick after another to the temple of the humankind knowledge. 

Since people discovered the notion of \textit{space} and \textit{time} and formalised it, this became one of the most important frames of human reasoning. Another important notion to be realised and formulated was the one of \textit{matter}, a medium which fills the space, evolves in time, and sensually possesses heterogeneous properties. At this point it was natural to start asking questions about the \textit{structure} of matter, trying to split it into logical categories and find its place in some larger framework of reasoning. Starting from things which one can observe and experience in daily life, the reasoning went farther away to smaller or larger scales in space -- a natural frame of reference for reason to explore. It is probably this narrative which contributed to the emergence of \textit{particle physics}, a field of science describing phenomena at the very smallest scale. 

Most of the scientific experiments in natural sciences deal with \textit{interacting} objects, which in turn are essentially various representations of matter. These observed phenomena had to be necessarily explained using some toolkit with \textit{fundamental} blocks and categories which matter is made of. Step-by-step, this necessity led to the discovery of \textit{particles} -- and in fact a plethora of them. All the visible matter turned out to be composed of them organised in peculiar patterns. It became also possible to explain multiple phenomena in nature by formalising it in terms of dynamics, transformation, and interaction of particles. 

However, it is not sufficient to only formulate particles as a set of logical entities. There is always a need of reason to bring structure into it, to organise them in a \enquote{beautiful} and \enquote{meaningful} way. Furthermore, the concept of interaction and evolution necessarily motivates the composition of rules guiding them. These rules take up the form of \textit{laws}, which value is estimated not only by the precision to describe already explored phenomena but also by the power to predict yet undiscovered ones.

Bringing various pieces of the puzzle together, this is how the \textit{Standard Model} (SM) has been gradually built. It has always been expanding to incorporate newly discovered particles, interactions, and observed phenomena while also predicting new ones on the way. Admittedly, it is an \enquote{Absolutely Amazing Theory of Almost Everything} \cite{Mulders:2019vhb} as it describes the most fundamental blocks which constitute matter and the rules of its transformation. This Chapter gives a brief overview of its fundamental aspects. Sec. \ref{sec:particles} introduces the particles which have been discovered so far. In Sec. \ref{sec:lagr} the theoretical foundation of SM is laid out. A discovery of the Higgs boson in 2012 \cite{ATLAS:2012yve,CMS:2012qbp} was a great milestone bringing yet one more piece of the puzzle into the Standard Model. The corresponding motivation behind its importance is described in Sec. \ref{sec:break} and the properties as predicted theoretically and measured experimentally are outlined in Sec. \ref{sec:higgs}.

\section{Particle content}\label{sec:particles}

A particle, being a small localised in space object with certain properties, is called elementary (or fundamental) if there is no other particles which it is composed of. Three important particle properties should be mentioned as the starting point:
\begin{itemize}
    \item \textbf{Spin} -- a quantum number which might be informally associated to the internal angular momentum of the particle. Particles taking half-integer spin values are called \textit{fermions}, and particles with an integer spin are called \textit{bosons}.
    
    \item \textbf{Electric charge} -- a quantum number which defines the behaviour of the particle in the electromagnetic field. Taking the positive/negative integer values, it categorises particle to be charged (charge does not equal to 0) or neutral (charge equals to 0).
    
    \item \textbf{Mass} -- a property which measures the strength of the particle interaction via a gravitational force. In this work, the mass is measured in electronvolt (eV) units, where $1 \text{eV}/c^2 = 1.78266192 \cross 10^{-36} ~\text{kg}$ and $c = 299 792 458 ~m/s$ is a speed of light. Natural units are used throughout this work which set $c = h/(2\pi) = 1$, where $h$ is a Planck constant.  
\end{itemize}

\begin{figure}[h!]
    \centering
    \includegraphics[width=0.7\textwidth]{Figures/SM/sm.png}
    \caption{Elementary particles in the Standard Model and their properties \cite{SM:web-plot}.}
    \label{fig:sm}
\end{figure}

The Standard Model describes the following elementary fermions and bosons (Fig. \ref{fig:sm}):
\begin{itemize}
    \item \textbf{Fermions:} 6 quarks of different flavour ($q = u, \,d, \,c, \,s, \,t, \,b$), 3 charged leptons ($l = e, \,\mu, \,\tau$ ), 3 neutral leptons ($\nu_l= \nu_e, \,\nu_\mu, \,\nu_\tau$).
    \item \textbf{Bosons:} 8 gluons of different colour ($g$), $W^\pm/Z$ particles, photon ($\gamma$), the Higgs boson ($H$). 
\end{itemize}

Each of the particles is also accompanied by an \textit{antiparticle} which either coincides with the particle itself or represent a new particle. In the latter case it has the same mass and spin but the opposite electric charge (if different from 0). If the particle interacts with the corresponding antiparticle they annihilate and produce photons, quanta of the electromagnetic field. Fermions can be further categorised into \textit{quarks} and \textit{leptons}. The former can take part in the strong interaction (Sec. \ref{sec:lagr}) while the latter cannot. They are furthermore combined into three generations (three left columns on Fig. \ref{fig:sm}).

The bosons are considered to be \textit{force carriers}, i.e. particles which mediate interaction between other particles and hence guide the evolution of particle systems. The only exception is the Higgs boson, which is responsible for the generation of particle masses. At the moment, four fundamental forces are discovered: electromagnetic, weak, strong, gravitational. The former three are known to have the corresponding boson mediators. The mediator of the gravitational force still remains a mystery to be understood. 

The notion of interaction and force is at the very core of the Standard Model, and its formalisation is provided in the next Section. 

\section{Formulation}\label{sec:lagr}

\subsection{Introduction}\label{sec:lagr-intro}
The Standard Model is a relativistic Quantum Field Theory (QFT) describing dynamics of particles moving at the speed close to the speed of light. It is built on the notion of \textit{symmetry}, which in this particular case refers to the invariance of physical processes under a certain type of transformations. The transformations are formalised using the language of a group theory. The latter introduces a mathematical concept of a \textit{group} -- a set of elements $G$ and an operation on this set which maps two elements of the set to another element in the set. The group should also satisfy three axioms: associativity, existence of the identify element, and existence of the inverse element.

For example, one of the most fundamental symmetries which the Standard model should satisfy is the \textit{Lorentz symmetry}. In particular, the theory should be build in a Lorentz-covariant way, so that physical quantities transform accordingly under a given representation of the Lorentz group (indefinite orthogonal group O(1,3)) -- i.e. as quantities composed of scalars, four-vectors, four-tensors, and spinors. This allows to properly account for the relativistic nature of particles in a consistent with the special theory of relativity way. 

The main component of the SM is a \textit{Lagrangian density} $L$ which is the general starting point to derive laws describing evolution of a system. If classical Lagrangian mechanics uses Lagrangian to derive equations of motion for a system with finite number of degrees of freedom, field theory extends the approach to \textit{fields} having infinite number of degrees of freedom.

Equations of motion can be derived via the \textit{action principle}. One defines an action functional as: 
\begin{equation}
    A[\phi(x)] = \int d^4x ~L(\phi(x), \partial_\mu\phi) 
\end{equation}

where the integral is taken over the space-time coordinates $x \equiv x_\mu = (t, \vec{x})$ with a Minkowski metric $g_{\mu\nu}$ = diag$(1, -1, -1, -1)$ and for the sake of illustration the Lagrangian density is written for a scalar field $\phi(x)$. Requiring the variation of the functional for every infinitesimal 
shift of the field $\delta\phi(x)$ to be 0:

\begin{equation}\label{eq:var}
    A[\phi(x) + \delta\phi(x)] - A[\phi(x)] = \int d^4x \left[\left(\partial_\mu\dfrac{\partial L}{\partial\partial_\mu\phi} - \dfrac{\partial L}{\partial \phi}\right)\delta\phi + \partial_\mu\left(\dfrac{\partial L}{\partial\partial_\mu\phi}\right)\right] = 0
\end{equation}

one can obtain the equation of motion for the field $\phi(x)$ by plugging-in the known Lagrangian density of the field system.

A simple example would be a Lagrangian density of the free Dirac field describing the motion of a spin-1/2 particle with the mass $m$:
\begin{equation}\label{eq:lagr-dirac}
    L_0 = \bar{\psi}(i\hat{\partial} - m)\psi
\end{equation}

where $\hat{\partial} \equiv \gamma_\mu\partial^\mu$, $\gamma_\mu$ are the gamma matrices, $\psi(x)$ is a Dirac spinor, $\bar{\psi}(x) \equiv \psi^\dag(x)\gamma_0$ is a Dirac-conjugated spinor. Using Eq. \ref{eq:var} one can obtain the famous Dirac equation:

\begin{equation}
    (i\hat{\partial} - m)\psi(x) = 0
\end{equation}

One can notice that the Dirac Lagrangian density (\ref{eq:lagr-dirac}) is invariant under the \textit{global} transformation of the U(1) group which acts on the spinor $\psi(x)$ as:
\begin{equation}
    \psi(x) \mapsto \psi^\prime(x) = e^{ie\omega} \psi(x),
\end{equation}

where $e$ and $\omega$ are some constants. However, if one assumes the \textit{local} U(1) symmetry $\omega \to \omega(x)$, the Lagrangian is no longer invariant. This symmetry can be restored if an additional interaction of the fermion with a photon field $A_\mu$ is introduced by promoting the usual derivative to a \textit{covariant} one:
\begin{equation}
    \partial_\mu \to D_\mu \equiv \partial_\mu -ieA_\mu,
\end{equation}

and defining the transformation of the photon field as $A_\mu \to A^\prime_\mu = A_\mu - \partial_\mu\omega$. The Lagrangian which is invariant under the local U(1) symmetry corresponds to the Lagrangian of Quantum Electrodynamics (QED):
\begin{equation}\label{eq:lagr-qed}
    L_\text{QED} = \bar{\psi}(i\hat{D} - m)\psi -\dfrac{1}{4}F^2_{\mu\nu},
\end{equation}

where a field-strength tensor $F_{\mu\nu} = \partial_\mu A_\nu - \partial_\nu A_\mu$ is also introduced.

If there was no mass term in the Dirac Lagrangian (\ref{eq:lagr-dirac}), then a \textit{chiral symmetry} would also be conserved. Left and right \textit{chiral spinors} are defined as eigenvalues of the $\gamma_5 \equiv i\gamma_0\gamma_1\gamma_2\gamma_3$ operator:
\begin{equation}
    \gamma_5\psi_{L/R} = \mp\psi_{L/R}.
\end{equation}

Every spinor decomposes into a sum of the left and right spinors:
\begin{equation}
    \psi = \psi_L + \psi_R, ~\psi_{L/R} = \dfrac{1\mp\gamma_5}{2}\psi,
\end{equation}

and the Lagrangian (\ref{eq:lagr-dirac}) reads:
\begin{equation}\label{eq:lagr-dirac-2}
    L_0 = i(\bar{\psi}_L\hat{\partial}\psi_L + \bar{\psi}_R\hat{\partial}\psi_R) - m(\bar{\psi}_L\psi_R + \bar{\psi}_R\psi_L)
\end{equation}

If one considers a transformation $\psi \to \psi^\prime = e^{i\gamma_5\omega}$ associated with the chiral symmetry, the second term (mass term) in Eq.  \ref{eq:lagr-dirac-2} violates this symmetry.

\subsection{Lagrangian}

\subsubsection{Electroweak part}
The Lagrangian (\ref{eq:lagr-qed}) was the first theory which combined special relativity with quantum mechanics in a consistent way to describe observed electromagnetic phenomena. But while it produces very accurate predictions for such quantities as the anomalous magnetic moment of the electron, it does not incorporate such phenomena as, for example, $\beta$ decay of the neutron.

The Fermi's interaction turned out to be a good explanation of the $\beta$ decay proceeding via a \textit{weak interaction}. However, this was still a standalone piece of the puzzle which was not related to the electromagnetic phenomena. More importantly, it had a fundamental problem of \textit{unitarity violation} -- the cross section of the reaction was predicted to increase linearly with the energy. This was hinting that the Fermi's theory was an effective theory valid only in the limit of low energies. A more fundamental theory was yet to be understood.

It was Sheldon Glashow, Abdus Salam, and Steven Weinberg who proposed a theory which would describe in a unified way electromagnetic and weak interactions. It is build on the assumption that the Lagrangian should be invariant under the transformation of the group:
\begin{equation}
    \sutwo \cross \uone.
\end{equation}

The underlying symmetry (also referred to as \textit{gauge} symmetry) is composed of two gauge groups. The \textit{weak-isospin} group \sutwo comes with three gauge fields $W^a_\mu$, $a=1,2,3$. The \textit{weak-hypercharge} group \uone comes with one gauge field $B_\mu$, analogously to the QED case with the $A_\mu$ field. 

$W^a_\mu$ are allowed to interact only with the left fermions. This behaviour is introduced in order to account for the $(V-A)$ pattern of the Fermi's interaction. The latter is introduced in order to describe the experimental observation of the \textit{parity violation} in the Wu experiment \cite{PhysRev.105.1413}. Parity is another fundamental transformation which corresponds to the inversion of the spatial coordinates $x_\mu = (t, \vec{x}) \mapsto (t, -\vec{x})$. The $(V-A)$ pattern effectively models each of the interacting fermion currents in the Fermi's theory as a difference between the vector $j_\mu^V = \bar{\psi}\gamma_\mu\psi$ and axial $j_\mu^A = \bar{\psi}\gamma_\mu\gamma_5\psi$ currents, where the former is conserved and the latter is not. 

The covariant derivatives which act on the left and right fermion fields are:
\begin{align}
    &D^L_\mu = \partial_\mu - ig\dfrac{T_a}{2}W^a_\mu - ig^\prime\dfrac{Y_L^f}{2}B_\mu\\
    &D^R_\mu = \partial_\mu - ig^\prime\dfrac{Y^R_f}{2}B_\mu.
\end{align}

Here $g$ and $g^\prime$ are the gauge couplings, $T_a = \dfrac{\sigma_a}{2}$ are the generators of the \sutwo group with $\sigma_a$ being the Pauli matrices, $Y^f_{L/R}$ are the hypercharges introduced by the \uone group. 

As it was mentioned in Sec. \ref{sec:lagr-intro}, in the Standard Model one needs to incorporate in total 6 quark and 6 fermion fields. In the example of one generation, the corresponding spinors are grouped into doublets and singlets under the \sutwo group:
\begin{align}\label{eq:ferm-notation}
    &Q_j \equiv \left(\begin{matrix} u_L \\ d_L \end{matrix}\right)_j, ~u \in \{u, c, t\}, ~d \in \{d, s, b\}\\
    &L_j \equiv \left(\begin{matrix} \nu_L \\ l_L \end{matrix}\right)_j, \nu \in \{\nu_{e}, \nu_{\mu}, \nu_{\tau}\}, ~l \in \{e, \mu, \tau\}\\
    &u_j \equiv (u_R)_j\\
    &d_j \equiv (d_R)_j\\
\end{align}

The fermion part of the electroweak (EWK) Lagrangian then can be written by summing the terms of the Dirac Lagrangian (\ref{eq:lagr-dirac}) across the fermion generations $j = 1,2,3$ while taking into account the \sutwo singlet/doublet structure and the corresponding covariant derivatives:
\begin{equation}\label{eq:lagr-ewk-0}
    L_f^\text{EWK} = \sum_j i\bar{Q}_j\hat{D}^L Q_j + i\bar{u}_j\hat{D}^R u_j + i\bar{d}_j\hat{D}^R d_j + i\bar{L}_j\hat{D}^L L_j + i\bar{l}_j\hat{D}^R l_j
\end{equation}

Furthermore, kinetic terms for the $W_\mu$ and $B_\mu$ fields should also be included into the EWK Lagrangian:

\begin{align}
    % &L^\text{EWK} = L_f^\text{EWK} + L_k^\text{EWK}\\
    &L_k^\text{EWK} = -\dfrac{1}{4}W_a^{\mu\nu}W^a_{\mu\nu}-\dfrac{1}{4}B^{\mu\nu}B_{\mu\nu}\\
    &W^a_{\mu\nu} = \partial_\mu W^a_\nu - \partial_\nu W^a_\mu +gf^{abc}W^b_\mu W^c_\nu\\
    &B_{\mu\nu} = \partial_\mu B_\nu - \partial_\nu B_\mu,
\end{align}
where $f^{abc}$ are the structure constants of the SU(2) group.

One can notice that Eq. \ref{eq:lagr-ewk-0} doesn't contain terms with the right-handed neutrinos since experimental results indicate that the neutrinos are always left-handed. Furthermore, there is no fermion mass terms included since they mix left and right chiralities (Eq. \ref{eq:lagr-dirac-2}) and therefore violate \sutwo symmetry. This problem is solved by introducing the interaction with the Higgs field which generates the fermion masses after the symmetry breaking mechanism (Sec. \ref{sec:break}). 

In order to account for QED, one needs to obtain terms where the photon field couples to the fermions. This is not explicit yet in the Lagrangian (\ref{eq:lagr-ewk-0}) and the desired interaction can be achieved by rotating the initial gauge fields $W^a_\mu, ~B_\mu$ to a physical basis:
\begin{align}\label{eq:ewk-rotate}
    &W^\pm_\mu = \dfrac{1}{\sqrt{2}}(W^1_\mu \pm iW^2_\mu),\\
    &\left(\begin{matrix} Z_\mu \\ A_\mu \end{matrix}\right) = \left(\begin{matrix} \cos\theta_W & -\sin\theta_W \\ \sin\theta_W & \cos\theta_W \end{matrix}\right) \left(\begin{matrix} W^3_\mu \\ B_\mu \end{matrix}\right).
\end{align}

Here the \textit{Weinberg angle} $\theta_W$ is introduced. The requirement that $A_\mu$ in Eq. \ref{eq:ewk-rotate} corresponds to the photon field adds further constraints on the hypercharges $Y_L^f$. Noting also that $Y_L^u = Y_L^d \equiv Y_L^Q$, $Y_L^\nu = Y_L^l \equiv Y_L^L$ and fixing the normalisation $Y_L^Q = 1/3$ one can obtain the following relations between the electric charge $e$ and gauge couplings $g, g^\prime$:
\begin{align}
    &e = g\sin\theta_W,\\
    &e = g^\prime\cos\theta_W,
\end{align}

and the following relation in terms of the eletromagnetic and $\sutwo\cross\uone$ charge operators:
\begin{equation}
    Q^f = (T_3^f + \dfrac{Y^f}{2}),
\end{equation}

where $T_3^f$ return $\pm1/2$ for the left up/down fermions and $0$ for the right fermions. Finally, the fermionic part of the Lagrangian takes the following form:
\begin{align}\label{eq:lagr-ewk-1}
    &L_f^\text{EWK} = L_{CC} + L_{NC}\\ 
    &L_{CC} = \dfrac{g}{\sqrt{2}}(J_\mu^+W^{+\mu} + J_\mu^-W^{-\mu}), \quad J_\mu^+ = \sum_f\bar{f}_u\gamma_\mu\dfrac{1-\gamma_5}{2}f_d\\
    &L_{NC} = e J_\mu^A A^\mu + \dfrac{g}{\cos\theta_W}J_\mu^Z Z^\mu, \quad J_\mu^A = \sum_f Q^f \bar{f}\gamma_\mu f, \quad J^Z_\mu = \sum_f \bar{f}\gamma_\mu(v^f - a^f\gamma_5)f\\
    &v^f = \dfrac{T_3^f}{2} - Q^f \sin^2\theta_W, \quad a^f = \dfrac{T_3^f}{2}  
\end{align}

One can observe that the (V-A) structure of the Fermi's interaction is present in the CC component of Eq. \ref{eq:lagr-ewk-1}. Moreover, now the theory predict one new particle -- the neutral $Z$ boson. It was a great success of the EWK theory when the Gargamelle experiment at CERN reported in 1973 the observation of the neutral current, a first indication of the $Z$ boson existence.   

Later on, both $W^\pm$ and $Z$ boson properties were measured at the Large Electron Proton (LEP) Collider and  at the Standford Linear Collider (SLAC). One of these properties is their mass, which is now measured to be $m_W = 80.377 \pm 0.012$ GeV, $m_Z = 91.1876 \pm 0.0021$ GeV \cite{ParticleDataGroup:2020ssz}. However, similarly to fermions the EWK theory does not have explicit mass terms for the boson since they would violate the underlying gauge symmetry. The minimalistic approach of leaving them as massless particles in the theory would contradict the experimental observations. Furthermore, one can show that the unitarity principle is violated in the gauge boson scattering amplitudes, which hence leaves the theory with issues to be solved.   

\subsubsection{QCD part}

One more piece of the puzzle which has not been mentioned so far is the theory describing the strong interaction. This theory is called \textit{Quantum Chromodynamics} (QCD) and it describes the interaction between quarks. The motivation for a dedicated theoretical foundation is that experimentally the quarks have not been found in a free state contrary to leptons. Instead, they are confined within bound states of quark-antiquark pairs $\bar{q_i}q_j$ (called \textit{mesons}) and quark triplets $q_iq_jq_k$ (caled \textit{barions}), where $i,j,k$ refers to different quark flavours. Mesons and barions are collectively called hadrons. These states are colour singlets, where the colour $C$ is a charge introduced by the \suthr group. As of now, six quarks have been discovered: up ($u$), down ($d$), charm ($c$), strange ($s$), top ($t$), bottom ($b$).

Another consequence of the theory, similarly to the EWK case, was the appearance of the gauge bosons associated to the group -- gluons. These are the mediators of the strong force which is responsible for formation of quark bound states. Experimentally, gluons were discovered at the electron-positron collider PETRA at DESY in the three-jet events \cite{Barber:1979yr}. The jet is a cone of particles produced after the hadronisation (hadron formation) of a quark or a gluon. In this search, two jets would be initiated by two quarks appearing from the lepton annihilation, while one jet would be initiated by the gluon radiated from one of the quarks.  

Theoretically, gluons are introduced by the covariant derivative corresponding to the \suthr group:
\begin{equation}
    D_\mu = \partial_\mu - ig_s\dfrac{\lambda_a}{2}G_\mu^a
\end{equation}

Here, $G_\mu^a$ are the gluon fields with the index $a=1..8$, $\lambda_a$ are the Gell-Mann matrices, $g_s$ is the gauge constant of the \suthr group. These terms from the covariant derivative are then added to the covariant derivatives in the quark terms of the EWK Lagrangian (\ref{eq:lagr-ewk-0}). Lastly, a kinetic term for the gluons also have to be introduced:

\begin{align}
    &L_k^\text{QCD} = -\dfrac{1}{4}G_{\mu\nu}^aG^{\mu\nu}_a\\
    &G_{\mu\nu}^a = \partial_\mu G_\nu^a - \partial_\nu G_\mu^a + g_s f^{abc}G_\mu^b G_\nu^c,
\end{align}

where $f^{abc}$ are the structure constants of the \suthr group.

\subsection{Spontaneous symmetry breaking}\label{sec:break}

As discussed up to this point, the SM Lagrangian has the following form, invariant under the local $\sutwo \cross \uone \cross \suthr$ gauge symmetry:
\begin{equation}
    L_0^\text{SM} = L_f^\text{EWK+QCD} + L_k^\text{EWK} + L_k^\text{QCD} + L_\text{Gauge-fixing} + L_\text{Ghosts},
\end{equation}

where the terms corresponding to the gauge fixing and Fadeev-Popov ghosts are added as required by the theory quantisation and renormalisation procedures. The theory thus assumes that the fermions and the gauge bosons are massless, since explicit mass terms violate the underlying symmetries. This clearly contradicts the experimental observations, which poses a question of completeness of the SM in such formulation.

It was a Brout-Englert-Higgs-Hagen-Guralnik-Kibble
mechanism which was proposed in 1964 by the authors as the solution to the mass generation problem. It describes what is commonly referred to as \textit{\enquote{spontaneous symmetry breaking}} and it proceeds as follows. A doublet $\Phi$ of complex scalar fields under the \sutwo group -- referred to as the \textit{Higgs field} -- is introduced:

\begin{equation}
    \Phi = \left(\begin{matrix} \phi^+ \\ \phi^0 \end{matrix}\right)  = \dfrac{1}{\sqrt{2}} \left(\begin{matrix} \phi_1 + i\phi_2 \\ \phi_3 + i\phi_4 \end{matrix}\right),
\end{equation}

with the corresponding Lagrangian consisting of the kinetic and potential terms:
\begin{equation}\label{eq:lagr-h}
    L^\text{Higgs} = \partial_\mu\Phi^\dag \partial^\mu\Phi - \text{V}(\Phi^\dag\Phi).
\end{equation}

where the potential reads:
\begin{equation}\label{eq:higgs-pot}
    \text{V}(\Phi^\dag\Phi) = \mu^2\Phi^\dag\Phi + \lambda (\Phi^\dag\Phi)^2
\end{equation}

\begin{figure}[t!]
    \centering
    \includegraphics[width=0.8\textwidth]{Figures/SM/mexican.png}
    \caption{Illustration of the Higgs potential (Eq. \ref{eq:higgs-pot}) in case of a single complex scalar field $\phi$ for two scenarios: $\mu^2 > 0$ (left) and $\mu^2 < 0$ (right) \cite{Mulders:2019vhb}.}
    \label{fig:mexican}
\end{figure}

Considering a simpler example of only one complex scalar field $\phi$ instead of a \sutwo doublet $\Phi$, one can distinguish either a scenario $\mu^2 > 0$ with a single minimum of the potential at $\phi_0=0$ or a scenario $\mu^2 > 0$ with a valley of degenerate minima at $\phi_0 \neq 0$ (Fig. \ref{fig:mexican}). In the latter case the non-trivial minima correspond to the value of the field:
\begin{equation}\label{eq:vev-0}
    \phi_0 \equiv \langle 0 | \phi(x) | 0 \rangle = \dfrac{v}{\sqrt{2}}e^{i\beta}, \quad v = \dfrac{\mu}{\sqrt{\lambda}}.
\end{equation}

$\phi_0$ is interpreted as the \textit{vacuum expectation value}, a characteristic of the vacuum state. One observes that $\phi_0$ transforms under $\text{U(1)}$ symmetry, however this symmetry is \textit{implicitly broken} when one selects a particular minimum:
\begin{equation}
    \phi_0 \stackrel{\beta=0}{=} \dfrac{v}{\sqrt{2}} 
\end{equation}

Considering again the $\Phi$ doublet, one can parametrise the Higgs field around the vacuum expectation value similarly to Eq. \ref{eq:vev-0} as:
\begin{equation}\label{eq:sym-break}
    \Phi(x) = \dfrac{1}{\sqrt{2}}\exp\left(i\dfrac{\zeta_j(x)\sigma^j}{2v}\right)\left(\begin{matrix}0 \\ v + h(x)\end{matrix}\right), \quad \Phi_0 \equiv \langle 0 | \Phi(x) | 0 \rangle = \dfrac{1}{\sqrt{2}}\left(\begin{matrix}0 \\ v \end{matrix}\right)
\end{equation}

Here, $\zeta_j$ fields are the Nambu-Goldstone bosons, massless scalar particles necessarily appearing in theories where a continuous symmetry is spontaneously broken, i.e. the ground state is not invariant under the action of the underlying group. $h$ is the physical state which corresponds to the \textit{Higgs boson}.

In order to link the Lagrangian (\ref{eq:lagr-h}) with the SM, one firstly introduces the covariant derivative $\partial_\mu \to D_\mu$ subject to the $\sutwo \cross \uone$ group of the SM. This adds interaction between the Higgs field and the gauge bosons. One can further show that after the symmetry breaking (\ref{eq:sym-break}) $W^\pm$ and $Z$ fields can be redefined to include $\partial_\mu\zeta_j$ terms. This is interpreted as the \enquote{absorption} of the Goldstone bosons resulting in the longitudinal polarisation of the $W^\pm$ and $Z$ bosons.

Moreover, the following terms appear after the symmetry breaking (\ref{eq:sym-break}):
\begin{align}
    &L^\text{SM} \supset M_W^2W_\mu^+ W^{\mu-}, \quad M_W^2 = \dfrac{g^2v^2}{4}\\
    &L^\text{SM} \supset \dfrac{1}{2}M_Z^2 Z_\mu Z^{\mu}, \quad M_Z^2 = \dfrac{(g+g^\prime)^2v^2}{4}
\end{align}

These are naturally interpreted as the mass term for the $W^\pm$ and $Z$ gauge bosons. The requirement that the photon has to remain massless leads to the following condition:
\begin{equation}
    g\sin\theta_W - g^\prime\cos\theta_W = 0,
\end{equation}

and the mass of the Higgs boson $h$ itself from the corresponding $m_Hhh$ term is found to be:
\begin{equation}
    m_H = \sqrt{2\lambda}v
\end{equation}

Given that in the limit of small energies one should obtain the Fermi's interaction, the vacuum expectation value can be directly linked to the Fermi constant:
\begin{equation}
    \dfrac{G_F}{\sqrt{2}} = \dfrac{g^2}{8M_W^2} \Rightarrow v^2 = \dfrac{1}{\sqrt{2}G_F} \simeq 246 ~\text{GeV}.
\end{equation}

One therefore concludes that the spontaneous symmetry breaking mechanism $\sutwo \cross \uone \to \text{U(1)}_\text{em}$ and the added Higgs potential $\text{V}(\Phi^\dag\Phi)$ solve the issue with the generation of the gauge boson mass terms. One can also show, that due to the terms with trilinear and quartic coupling between $h$ and $W^\pm/Z$, the unitarity violation in the scattering amplitudes vanishes. Finally, it predicts the existence of the new scalar Higgs boson. A particle which so far meets the required properties to a very good precision was discovered in 2012 by the CMS and ATLAS experiments, as described in more detail in Sec. \ref{sec:higgs}.  

There is one more aspect which has not been covered yet -- the mass generation for the fermions. While for neutrinos the process of acquiring mass is still not solved, for the other fermions it proceeds with the addition of \textit{Yukawa terms}, preserving the $\sutwo \cross \uone$ symmetry:
\begin{equation}\label{eq:lagr-yuk}
    L^\text{Yukawa} = - \sum_{ij} \left(Y^{ij}_d \bar{Q}^i \Phi d^j + Y^{ij}_u \bar{Q}^i \Phi^c u^j + Y^{ij}_l \bar{L}^i \Phi l^j\right) + h.c.
\end{equation}

The notation here is identical to the one introduced in (\ref{eq:ferm-notation}) and the sum is taken over the three fermion generations. The charge conjugate of the Higgs field is defined as $\Phi^c = i\sigma_2\Phi^*$. $Y^{ij}_u$, $Y^{ij}_d$, and $Y^{ij}_l$ are the general complex $3\cross 3$ matrices. The expression (\ref{eq:lagr-yuk}) is written in the flavour basis and one can make the corresponding transformation to the mass basis by rotating the fields with a unitary matrix:
\begin{equation}
    Q \to V_Q Q, \quad u \to V_u u, \quad d \to V_d d
\end{equation}

so that:
\begin{equation}
    Y_d = \text{diag}(y_d, y_s, y_b), \quad Y_u = V_\text{CKM}^\dag\text{diag}(y_u, y_c, y_t),
\end{equation}

where $V_\text{CKM}$ is the Cabibbo-Kobayashi-Maskawa (CKM) matrix. After performing this diagonalisation and the symmetry breaking (\ref{eq:sym-break}) one obtains the following terms describing the quark interaction with the gauge bosons and the Higgs field:
\begin{equation}
    L^\text{SM} \supset L_{NC} + \dfrac{g}{\sqrt{2}}\bar{u}^i_L \hat{W}^+ \dfrac{1-\gamma_5}{2}V_\text{CKM}^{ij}d^j_L + \dfrac{y_{u_i}v}{\sqrt{2}}\bar{u}^i_L u^i_R\left(1+\dfrac{h}{v}\right) + \dfrac{y_{d_i}v}{\sqrt{2}}\bar{d}^i_L d^i_R\left(1+\dfrac{h}{v}\right) + h.c., 
\end{equation}

where $L_{NC}$ is defined in Eq. \ref{eq:lagr-ewk-1}. One can observe that now quarks acquired the mass:
\begin{equation}
    m_{u_i/d_i} = \dfrac{y_{u_i/d_i}v}{\sqrt{2}},
\end{equation}

and they couple to the Higgs boson with the strength which is linearly proportional to the quark mass. The CKM matrix further modifies the interaction of the quarks with the $W$ boson comparing with (\ref{eq:lagr-ewk-1}) by accounting for the mixing across the generations.

It should be noted that the same diagonalisation from the flavour to the mass basis can be performed in the lepton sector, yielding the same mass generation $m_{l_i} = \dfrac{y_{l_i} v}{\sqrt{2}}$ and Higgs coupling structure as for the quarks. In the basis where the Yukawa couplings of the charged leptons are diagonal, flavour states of neutrinos are linked to the mass states by the rotation with the Pontecorvo–Maki–Nakagawa–Sakata (PMNS) matrix. The PMNS matrix enters the charged current for the neutrino in place of the CKM matrix in the quark case. Furthermore, it plays an important role in the description of neutrino oscillations.

Lastly, it should be noted that it is the Yukawa terms (\ref{eq:lagr-yuk}) which violate the \textit{CP symmetry} in the fermion sector of the SM. This is the symmetry under the combined charge conjugation and parity transformations. The former corresponds to the change of particle to their corresponding antiparticles, while the latter corresponds to the change of the sign of the spatial coordinates. These discrete symmetries are not explicitly introduced into the theory as the foundation, but are observed in nature and also implicitly manifest itself in the SM Lagrangian. Due to the fact that there is exactly three generations of fermions, $Y_{ij}^* \neq Y_{ij}$ and there is one complex phase parameter in the CKM matrix and, in case of the Dirac nature of neutrinos, also in the PMNS matrix. In nature, this manifests itself as direct/indirect CP violation, for example, in the kaon \cite{KTeV:1999kad}, B-meson \cite{LHCb:2013syl}, and D-meson sectors \cite{LHCb:2019hro}.    

This work continues the investigation of the Yukawa coupling sector from the perspective of searching CP anomalous effects. In particular, the structure of the Yukawa coupling between tau leptons and the Higgs boson is experimentally studied, as described in Sec. \ref{sec:cp-etau}.

\section{Higgs boson properties}\label{sec:higgs}

In 2012 the CMS and ATLAS experiments at the Large Hadron Collider (Sec. \ref{sec:lhc}) announced an observation of a new particle at a mass of 125 GeV \cite{ATLAS:2012yve, CMS:2012qbp}. In the CMS experiment the search was performed in $\gamma\gamma$, $ZZ$, $WW$, $\tau\tau$, and $b\bar{b}$ final states using the proton-proton collision at the center-of-mass energy $\sqrt{s} = 7$ and $8$ TeV. A local observed significance of the signal combined for all the decay modes was 5.0 standard deviations ($\sigma$). In the ATLAS experiment, using the same decay modes the local observed significance of $5.9\sigma$ was reached. 

Although the properties of the newly observed particle were reasonably consistent with the SM expectations of the Higgs boson, more studies had to be performed to better understand the nature of the new particle. Since 10 years of its discovery, multiple measurements have been done in order to understand deeper its properties \cite{CMS:2022dwd, ATLAS:2022vkf}.  

\begin{figure}[!h]
    \centering
    \includegraphics[width=0.99\textwidth]{Figures/SM/h-int.png}
    \caption{Feynman diagrams for the leading Higgs boson interactions: single Higgs boson production modes (top left), Higgs boson decay channels (top right), and Higgs boson pair production modes (bottom) \cite{Mulders:2019vhb}.}
    \label{fig:h-int}
\end{figure}

From the SM perspective, the Higgs boson is predicted to be dominantly produced in the following processes (Fig. \ref{fig:h-int}, top left panel): 
\begin{itemize}
    \item Gluon-gluon fusion (ggH),
    \item Vector boson fusion (VBF),
    \item Associated production with a W or Z boson (VH or Higgsstrahlung),
    \item Associated production with top (ttH) or bottom (bbH) quarks.
\end{itemize}

\begin{figure}[!ht]
    \centering
    \includegraphics[width=0.56\textwidth]{Figures/SM/h-prod.pdf}
    \includegraphics[width=0.42\textwidth]{Figures/SM/h-br.eps}
    \includegraphics[width=0.45\textwidth]{Figures/SM/h-prod-exp.png}
    \includegraphics[width=0.45\textwidth]{Figures/SM/h-br-exp.png}
    \caption{(Top left panel) Predicted Higgs boson production cross sections for various production modes as a function of the Higgs boson mass in proton-proton collisions at $\sqrt{s} = 13$ TeV \cite{LHCHiggsCrossSectionWorkingGroup:2016ypw}. (Top right panel) Predicted Higgs boson branching fractions for various decay modes as a function of the Higgs boson mass in proton-proton collisions at $\sqrt{s} = 13$ TeV \cite{LHCHiggsCrossSectionWorkingGroup:2013rie}. (Bottom left panel) Signal strength modifiers for each of the Higgs production modes as measured by the CMS experiment \cite{CMS:2022dwd}. (Bottom right panel) Signal strength modifiers for each of the Higgs decay modes as measured by the CMS experiment \cite{CMS:2022dwd}.}
    \label{fig:h-prod}
\end{figure}

In the proton-proton collisions at $\sqrt{s}=13$ TeV, for the mass of the Higgs boson $m_\text{H} = 125.38$ GeV the most dominant production mode is ggH with the predicted cross section $\sigma_\text{ggH} = 48.31 \pm 2.44$ pb (Fig. \ref{fig:h-prod}, top left panel). This accounts for approximately 87\% of the total predicted production cross section $\sigma_\text{tot} = 55.4 \pm 2.6$ pb \cite{LHCHiggsCrossSectionWorkingGroup:2016ypw}. The next most important contributions come from the VBF mode (7\%, $\sigma_\text{VBF} = 3.771 \pm 0.807$ pb), the VH mode (4\%, $\sigma_\text{WH} = 1.359 \pm 0.028$ pb, $\sigma_\text{ZH} = 0.877 \pm 0.036$ pb). Experimentally, in CMS these values were confirmed to a large degree as measured by the signal strength modifiers $\mu_i$ (Fig. \ref{fig:h-prod}, bottom left panel). The latter corresponds to the ratio of experimentally observed signal yields to those predicted by the SM. The signal strength obtained after the fit to all the production modes and decay channels  of all Run 2 data (Sec. \ref{sec:lhc}) with a common single parameter results in $\mu = 1.002 \pm 0.057$, which is in excellent agreement with the SM prediction.

The mass of the Higgs boson is measured to be $m_\text{H} = 125.25 \pm 0.17$ GeV as a combination of the CMS and ATLAS results in the most precisely reconstructed $\gamma\gamma$ and $ZZ \to 4l$ final states \cite{ParticleDataGroup:2020ssz}. The width for the Higgs boson with such mass is predicted to be $\Gamma_\text{H} = 4.14 \pm 0.02$ MeV \cite{LHCHiggsCrossSectionWorkingGroup:2016ypw}. Using off-mass-shell and on-mass-shell production, the CMS experiment measured the width to be $\Gamma_\text{H} = 3.2^{+2.4}_{-1.7}$ MeV \cite{CMS:2022ley}, which is in agreement with the SM expectation. 

As discussed in Sec. \ref{sec:break}, the Higgs field and the corresponding Higgs boson, appearing after the spontaneous symmetry breaking, couple to fermions (via the Yukawa terms) and gauge bosons $W^\pm/Z$ (via the covariant derivative terms). Therefore it is expected that the SM Higgs boson decays into the corresponding particle-antiparticle pairs as well as into the massless bosons (photons and gluons) via loops at the quantum level (Fig. \ref{fig:h-int}, top right panel). Furthermore, the Higgs coupling strength is proportional to the mass (mass squared) of the fermion (vector boson). This sets the hierarchy of the expected decay probabilities with the third fermion generation being more preferred comparing to the second and first generations (Fig. \ref{fig:h-prod}, top right panel). 

The Higgs boson decays which have been observed so far are:
\begin{itemize}
    \item Bosons: $\text{H} \to \gamma\gamma$, $\text{H} \to ZZ$, $\text{H} \to WW$
    \item 3rd generation fermions: $\text{H} \to \tau\tau$, $\text{H} \to \text{bb}$
    \item 2rd generation fermions: $\text{H} \to \mu\mu$
\end{itemize}

\begin{figure}[!ht]
    \centering
    \includegraphics[width=0.6\textwidth]{Figures/SM/h-coup-exp.png}
    \caption{The measured $\kappa$-coupling modifiers for various Higgs boson decay modes as a function of the corresponding fermion/gauge boson mass \cite{CMS:2022dwd}.}
    \label{fig:h-coup}
\end{figure}

To date, the signal strength modifiers associated to each of the decay modes also agree well with the SM predicted values within uncertainties (Fig. \ref{fig:h-prod}, bottom right panel). One can also perform the analysis in the $\kappa$-framework \cite{LHCHiggsCrossSectionWorkingGroup:2013rie}, which introduces $\kappa$ parameters scaling the interaction of the Higgs boson with a given particle both in production (affecting the cross section) and decay (affecting the decay width). The $\kappa$ value of one corresponds to the SM scenario. Experimental results for the $\kappa$-parametrised couplings of the Higgs boson with the fermions and gauge bosons indicate perfect agreement with the SM expectation over three orders of magnitude of mass (Fig. \ref{fig:h-coup}).  

Due to the mentioned mass-dependant coupling of the Higgs boson significantly decreasing the production cross section/branching fraction, it is experimentally challenging to reach the coupling of the Higgs boson to the second and first generations fermions. However, it is of high importance to observe and measure these processes, either in production or in decay,  to further verify that the observed particle with the mass $m_\text{H} \simeq 125$ GeV is indeed the SM Higgs boson. Recent evidence for the $\text{H} \to \mu\mu$ decay \cite{CMS:2020xwi} and search for the $\text{VH} \to cc$ decays \cite{CMS:2022psv} have made a promising step in this direction. Furthermore, the parameters of the Higgs potential (\ref{eq:higgs-pot}) are also yet to be measured experimentally. This is where the ongoing quest to discover the Higgs pair production (Fig. \ref{fig:h-int}, top right panel) will shed more light on the Higgs boson nature \cite{CMS:2022dwd}. 

The last item to be covered is the CP properties of the Higgs boson. The SM predicts the Higgs boson to be even under the charge-parity inversion, i.e. to have the quantum numbers of the pure CP scalar particle $\text{J}^\text{CP} = 0^{++}$. Experimentally, the hypotheses of a pure pseudoscalar as well as a spin-1 and spin-2 particle were excluded \cite{CMS:2013fjq, ATLAS:2015zhl} at the confidence level of more than $3\sigma$. However, it is possible that the observed Higgs boson is a mixture of scalar and pseudoscalar hypotheses. Such anomalies in the CP sector can be searched either in the Higgs coupling to the fermions or to the vector bosons. For the former, the Higgs interaction with the top quark and the tau lepton plays the dominant role. Pure CP-odd hypothesis was rejected at the confidence level of more than $3\sigma$ and upper limits were set on the anomalous Yukawa couplings to the top quark in the studies of tH and ttH processes \cite{CMS:2020cga, ATLAS:2020ior, CMS:2022dbt,ATLAS:2022ngt}. The coupling with the vector bosons has also been probed in the VBF and VH production modes in the $\text{H}\to ZZ, \gamma\gamma, \tau\tau$ decays \cite{CMS:2021nnc, Collaboration:2022mlq,ATLAS:2022tan, ATLAS:2022fnp}. Overall, the results are compatible with the SM expectation of the pure CP scalar hypothesis. This work complements these studies by exploring the Yukawa coupling of the Higgs boson with tau leptons.