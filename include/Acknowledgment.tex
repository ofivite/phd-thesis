\thispagestyle{empty}
\vspace{-3cm}
\section*{\centering Acknowledgements}

I should be honest: the time of my PhD has not been an easy ride. Now I am thinking that maybe I should have thought several times before committing to such an experience if I had known how this would unfold. But luckily I did not, so for me, this was a step into the unknown with an open heart and no expectations whatsoever.

The only metaphor of this time which persistently swirls in my mind is the one of light shining through the darkness. I guess it is the reflection of my personality: as my mother told me once, \enquote{you are a careful pessimist}. If I were to make a metaphoric image of this world, it would be a calm grey water surface at twilight with light ripples fading away; when it is not clear if it is a sunset or a sunrise, if it calls for a storm or contemplates its end.

It was not really the PhD studies which added up to the dark component of this (very naive and misleading!) dual depiction of reality. Of course, pushing the analysis further and trying to understand its nuances was not a piece of cake -- rather a marathon towards deeper satisfaction of the Eureka moment. It was more about the world turning upside down and downside up back again. It is probably its natural state, but in any case, it felt hard to recover from the loss of ground and push myself away from dissolving in the dark. The harmony felt so distant as it was stormy all along.

Despite this, I would like to acknowledge the whole experience of my time in Hamburg: as it was, with all the ups and downs, struggles, suffering, moments of enlightenment, excitement, and enthusiasm. I view the world as a vibration, and retrospectively I think the rhythm of these times has been unique and special to be experienced.

This experience would be empty and lacking depth without all the wonderful people with whom I have had a chance to share it. I would certainly say that they have been one of the sources of light in this metaphorical darkness. Thank you so much:
\begin{itemize}
	\item Alexei Raspereza \& Elisabetta Gallo for babysitting me, giving space for exploration and supporting my growth as a researcher. I cannot think of the supervisor-student relationship differently than those of parents and kids. Of course, I exaggerate a bit here -- but still, the PhD step for me is reminiscent of late adolescence: there is little babysitting involved, which is why it is easy to get lost as one is establishing their research values. That is why I am very thankful for your guidance and wisdom as I was navigating my way through the scientific landscape. I definitely could not ask for better supervisors.
	
	\item My parents Viktor \& Tatiana and brother Andrey for unconditional support and love. I realised this during the past years, as I was growing older, what it is actually like to be a parent (in particular of myself) and what it actually means to love someone. There is enormous work behind it, but at its core, it goes easily and naturally from the depth of the heart. You helped me a lot when I felt broken and scared, and I am deeply grateful for all your kind words and advice.
	
	\item DeepTau team, in particular Konstantin and Mykyta. I remember the (absolutely random) moment when I joined the team, and I love such turning points. There is no decisive moment \textit{per se}, only following the flow as it leads you where one should be. It has been a pleasure to work with you as we were developing DeepTau v2.5 and beyond. It was the time when I learned so much about how things should be written and designed in code. Moreover, the leadership style and the expertise standards of Konstantin are something which I personally admire, and I am grateful for the chance to learn it from you.
	
	\item All the nice DESY people with whom I worked together, in particular Aliya, Andrea, Daina, Mareike, Maryam, Sam, Sandra, Teresa, Valeria, and Yiwen. It is the people who create the environment, and I am very happy that I was a part of the welcoming and supportive DESY research space. I think it is crucial for every researcher to have peers to share and exchange ideas with. And so it is good that at DESY we have got ideas flowing and new interactions being explored.
	
	\item My close friends: Alyona, Bettina, Denis, Dima, Eldar, Gosha, Leo, Micha\l, Nastya, Valentina, Vlad (1), Vlad (2), Vova, Stepan, Yana. I am happy that with you I could not care less about being serious: it was in the natural order of things to make ridiculous (meta)ironic jokes, form a self-proclaimed beer party, ride a bike with a bunch of balloons, and in general behave like kids. And yet we could sit on the floor of our WG kitchen and talk about the depth of the world, or discuss the implications of ancient Greek philosophy, or argue what it is to be a scientist. We were exploring together the landscapes of music in the most intimate ways (i.e. jamming with each other) and the landscapes of the world through the lenses of film cameras (with occasional stops in craft beer shops). We were drinking wine on Elbstrand and talking about art or were slurping lots of coffee in Surf to dial in that sweat spot. You are all deeply kind and loving in your heart, and this is the precious thing I cherish.
\end{itemize}

It is funny that after writing this, I truly realised how lucky I am to have such people close by. There is a saying that it is easy to overlook and not notice the value of things one possesses until they are lost. In that respect, I would like to finally acknowledge these acknowledgements for an opportunity to contemplate what I have experienced during the years of my PhD and what I value as important to stick to before it is lost.

In the end, I would like to conclude this work with a reminder for my future self and a curious reader: there is a bright side, and it shines within oneself.

\thispagestyle{empty}

\vfill
\centering$\sim\sim\sim$

