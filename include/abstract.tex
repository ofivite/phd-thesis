\thispagestyle{empty}
\vspace{-3cm}
\section*{\centering Abstract}

\noindent

The measurement of the CP properties of the Yukawa coupling of the Higgs boson to $\tau$ leptons is presented. The data set used for the analysis is collected by the CMS experiment at the LHC during the Run 2 data-taking period in proton-proton collisions at $\sqrt{s}=13$ TeV and corresponds to an integrated luminosity of 137 \ifb. The Yukawa coupling between the Higgs boson and $\tau$ leptons is parametrised in terms of the effective mixing angle \mixa, where the value $\mixa = 0^\circ (90^\circ)$ corresponds to the SM scenario of the pure CP-even (CP-odd) $\text{H}\tau\tau$ coupling. 

The angle between the decay planes of the $\tau$ leptons is used as the observable encoding the CP nature of the Higgs boson. The measurement is performed in the \et channel where one $\tau$ lepton decays into a single electron and the other hadronically. The results are combined with the measurement in the \mt and \tata channels. The observed (expected) value of the effective mixing angle for the combination is measured to be:

\begin{equation}
    \mixa = -1 \pm 19^\circ (0 \pm 21^\circ) ~@68.3\% \text{ CL}.
\end{equation}

The results are compatible with the SM expectation and the pure CP-odd hypothesis is rejected at the observed (expected) significance level of $3.0 (2.6)$ standard deviations.

The improvements to the $\tau$ lepton identification in CMS in the context of the Run 3 preparation are described. Retraining and optimisation of the DeepTau algorithm with addition of the adversarial fine-tuning procedure is performed. The resulting model improves upon the previous DeepTau model in terms of the background rejection by 10-50\% and has a better description of data with simulation in the \htt selection region. 

A new algorithm called Tau Transformer (TaT) is proposed to overcome the limitations of the DeepTau architecture. The TaT core is based on self-attention layers and features the embedding module allowing for the multimodality treatment of the input representation. Comparison of the TaT model with the retrained DeepTau model and a comparable ParticleNet-based architecture shows consistently improved performance by up to 50\% in the misidentification rate across the \pt, $\eta$, and decay mode ranges of interest.   

\newpage
\thispagestyle{empty}
\mbox{}

